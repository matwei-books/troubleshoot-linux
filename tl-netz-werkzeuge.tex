%% tl-netz-werkzeuge.tex
\chapter{Werkzeuge zur Netzwerkfehlersuche}
\label{cha:netz-werkzeuge}

\begin{abstractsec}
  Verschiedene Werkzeuge eignen sich zur Fehlersuche im Netz. Hier stelle ich
  diejenigen vor, die ich in den nächsten Kapiteln einsetzen will.
\end{abstractsec}

\begin{notes}
\item[mii-tools, eth-tools] für Hardware-Ethernet-Schnittstellen
\item[quagga] zur Diagnose und Manipulation von Routingprotokollen und
  -tabellen
\item[smbclient] für Windows-Namensdienste
\item[traceroute] zur Pfadkontrolle
\end{notes}

\section{netstat}
%\label{sec:netz-werkzeuge-netstat}

\begin{abstractsec}
  Mit netstat ist es möglich essentielle Informationen zur
  Netzwerkkonfiguration und zum aktuellen Zustand der Netzwerksockets und
  Verbindungen eines Rechners zu erfahren.
\end{abstractsec}
\begin{normaltext}
  Ein Werkzeug, dass ich auch bei der Analyse von lokalen Rechnerproblemen
  einsetze und dort bereits vorgestellt habe, ist netstat. Im Bereich
  Netzwerkprobleme kann es viele Informationen liefern, die mir, je nach
  Problemfall weiterhelfen können.

  \subsection{Sockets}
  \label{sec:netz-werkzeuge-netstat-sockets}

  Rufe ich \verb?netstat? ohne Argumente auf, liefert es mir eine Liste der
  offenen und aktiven Sockets aller konfigurierten Adressfamilien, das
  heisst, der bestehenden Verbindungen.

  Meist interessieren mich nicht alle Adressfamilien, sondern nur ganz
  bestimmte. Dann kann ich diese zum Beispiel mit der Option
  \verb?--protocol=$familie? einschränken. Für \verb?$familie? kan ich in
  einer kommaseparierten Liste die folgenden angeben: \verb?unix?,
  \verb?inet?, \verb?ipx?, \verb?ax25?, \verb?netrom?, \verb?ddp?. Alternativ
  kann ich jeden gewünschten Familiennamen einzeln als Option übergeben:
  \verb?--unix?, \verb?--inet?, \ldots

  In diesem Teil der Buches interessiert mich vor allem die Familie
  \verb?--inet?. Diese kann ich weiter eingrenzen. Mit der Option \verb?-4?
  beziehungsweise \verb?-6? beschränke ich die Ausgabe auf die entsprechende
  Version des Internet Protokolls.

  Außerdem verwende ich
  \begin{itemize}
    \item[--raw oder -w] wenn ich an Raw-Sockets interessiert bin,
    \item[--tcp oder -t] für TCP-Sockets, und
    \item[--udp oder -u] für UDP-Sockets.
  \end{itemize}

  Bin ich nur daran interessiert, ob überhaupt ein Prozess an einem bestimmten
  Socket wartet, verwende ich die Option \verb?--listening? beziehungsweise
  \verb?-l?. Diese werden bei der normalen Ausgabe weggelassen. Will ich
  hingegen sowohl die aktiven als auch die lauschenden Sockets erfassen,
  verwende ich die Option \verb?--all? beziehungsweise \verb?-a?.

  \subsection{Routen}
  \label{sec:netz-werkzeuge-netstat-routen}

  Wenn ich statt an den Sockets eher an den Routen interessiert bin, verwende
  ich die Option \verb?--routes? beziehungsweise \verb?-r?. Damit bekomme ich
  die gleiche Ausgabe, wei mit dem Befehl \verb?route -e?. Auch hier kann ich
  mit \verb?-4? oder \verb?-6? die Protokollversion einschränken.

  Füge ich die Option \verb?-C? hinzu, bekomme ich Informationen aus dem
  Routencache, mit der Option \verb?-F? stattdessen aus der Forwarding
  Information Base (der Routentabelle), aber das ist sowieso die
  Voreinstellung.

  \subsection{Interfaces}
  \label{sec:netz-werkzeuge-netstat-interfaces}

  Mit der Option \verb?--interfaces? oder \verb?-i? kann ich Informationen
  über die Netzwerk-Interfaces bekommen.

  Ein einfaches \verb?netstat -i? liefert mir in einer übersichtlichen Tabelle
  zu jedem aktiven Interface unter anderem die MTU, die Anzahl der gesendeten
  und empfangenen Datenpakete sowie die Anzahl der Sende- beziehungweise
  Empfangsfehler.

  Kombiniere ich das mit \verb?-e?, bekomme ich die gleiche Ausgabe wie vom
  Program \verb?ifconfig?. Kombiniert mit \verb?-a? werden auch Interfaces
  angezeigt, die nicht im Zustand \verb?UP? sind.

  \subsection{Multicast-Gruppen}
  \label{sec:netz-werkzeuge-netstat-groups}
  Die Option \verb?--groups? beziehungsweise \verb?-g? liefert mir
  Informationen zur Mitgliedschaft des Rechners in Multicast-Gruppen.

  Auch diese kann ich mit \verb?-4? oder \verb?-6? einschränken.

  \subsection{Statistiken}
  \label{sec:netz-werkzeuge-netstat-statistics}

  Mit der Option \verb?--statistics? beziehungsweise \verb?-s? zeigt netstat
  zusammengefasste Statistiken für alle Protokolle.

  \subsection{allgemeine Optionen}
  \label{sec:netz-werkzeuge-netstat-allgemein}

  Abschließen möchte ich diese kleine Vorstellung von netstat mit ein paar
  allgemeinen Optionen, mit denen ich die Ausgabe modifizieren kann.

  Die von mir wohl am meisten eingesetzte Option ist \verb?--numeric?, kurz
  \verb?-n?. Mit dieser Option zeigt netstat numerische statt symbolischer
  Informationen an und das ist insbesondere bei Netzwerkadressen ein immenser
  Geschwindigkeitsvorteil, da sonst unter Umständen etliche DNS-Anfragen mit
  den entsprechenden Verzögerungen gestellt werden, bevor die Ausgabe
  angezeigt werden kann. Natürlich kann man das auch selektiv einstellen mit
  \verb?--numeric-hosts?, \verb?--numeric-ports? und \verb?--numeric-users?.

  Mit der Option \verb?--verbose? oder \verb?-v? kann ich mehr Informationen
  bekommen, insbesondere zu nicht konfigurierten Adressfamilien.

  Ähnliches bietet die Option \verb?--extend? oder \verb?-e?, die zusätzliche
  Informationen zum Beispiel bei Interfaces liefert.

  Gebe ich die Option \verb?--continuos? oder \verb?-c? an, bekomme ich die
  Informationen aller Sekunde ausgegeben.
\end{normaltext}

%\subsection{Seiteneffekte}
%\label{sec:seiteneffekte}

%\subsection{Einschränkung der Umsetzung}
%\label{sec:einschr-der-umsetz}

%\section{Schnittstellen nach außen}
%\label{sec:schn-nach-au3en}

%\subsection{Schnittstelle zu A}
%\label{sec:schnittstelle-zu}

%\begin{notes}
%\item Syntax
%\item Semantik
%\end{notes}

%\subsection{Schnittstelle zu B}
%\label{sec:schnittstelle-zu-b}

%\begin{notes}
%\item Syntax
%\item Semantik
%\end{notes}


%%% Local Variables: 
%%% mode: latex
%%% TeX-master: "troubleshoot-linux"
%%% End: 
%%% vim: set sw=2 ts=2 tw=78 et si:
