%% tl-netz-werkzeuge.tex
\chapter{Werkzeuge zur Netzwerkfehlersuche}
\label{cha:netz-werkzeuge}

\begin{abstractsec}
  Verschiedene Werkzeuge eignen sich zur Fehlersuche im Netz. Hier stelle ich
  diejenigen vor, die ich in den nächsten Kapiteln einsetzen will.
\end{abstractsec}

\begin{notes}
\item[mii-tools, eth-tools] für Hardware-Ethernet-Schnittstellen
\item[smbclient] für Windows-Namensdienste
\end{notes}

%\section{Konzept der Umsetzung}
%\label{sec:konz-der-umsetz}

%\begin{abstractsec}
%  Die Aufteilung der Umsetzung wird hier gegliedert in verschiedene Aspekte
%\end{abstractsec}
%\begin{normaltext}
%  Blablabla
%\end{normaltext}

%\subsection{Seiteneffekte}
%\label{sec:seiteneffekte}

%\subsection{Einschränkung der Umsetzung}
%\label{sec:einschr-der-umsetz}

%\section{Schnittstellen nach außen}
%\label{sec:schn-nach-au3en}

%\subsection{Schnittstelle zu A}
%\label{sec:schnittstelle-zu}

%\begin{notes}
%\item Syntax
%\item Semantik
%\end{notes}

%\subsection{Schnittstelle zu B}
%\label{sec:schnittstelle-zu-b}

%\begin{notes}
%\item Syntax
%\item Semantik
%\end{notes}


%%% Local Variables: 
%%% mode: latex
%%% TeX-master: "troubleshoot-linux"
%%% End: 
%%% vim: set sw=2 ts=2 tw=78 et si:
