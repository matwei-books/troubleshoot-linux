%% tl-netz-werkzeuge.tex
\chapter{Werkzeuge zur Netzwerkfehlersuche}
\label{cha:netz-werkzeuge}

\begin{abstractsec}
  Verschiedene Werkzeuge eignen sich zur Fehlersuche im Netz. Hier stelle ich
  diejenigen vor, die ich in den nächsten Kapiteln einsetzen will.
\end{abstractsec}

\begin{notes}
\item[iproute, bridge-utils, netstat, route] zur Abfrage,
  Manipulation der (Kernel-)Netzwerkschnittstellen und Kernelrouten
\item[mii-tools, eth-tools] für Hardware-Ethernet-Schnittstellen
\item[ping, traceroute, nmap] zur ersten Diagnose von Verbindungsproblemen
\item[tcpdump, wireshark, libtrace-tools] zur Diagnose von hartnäckigen
  Verbindungsproblemen
\item[quagga] zur Diagnose und Manipulation von Routingprotokollen und
  -tabellen
\item[telnet, nc, openssl, smbclient] zum Testen von Anwendungsprotokollen
\item[iperf, ttcp, nuttcp] zur Diagnose von Performanceproblemen
\end{notes}

\section{arp}
\label{sec:netz-werkzeuge-arp}
\begin{abstractsec}
  Das Programm arp dient der Anzeige und Manipulation des ARP-Caches des
  Kernels und wird bei Problemen in direkt angeschlossenen Netzwerksegmenten
  eingesetzt.
\end{abstractsec}
\begin{normaltext}
  Das Programm arp dient der Anzeige und Manipulation des ARP-Caches des
  Kernels und wird bei Problemen in direkt angeschlossenen Netzwerksegmenten
  eingesetzt.

  ARP, das (Ethernet) Address Resolution Protocol, dient der Zuordnung von
  Internetadressen zu Ethernet-Adressen. Es wurde in RFC826 beschrieben und in
  RFC5227 und RFC5494 aktualisiert.

  Der prinzipielle Aufruf ist:
  \begin{verbatim}
# arp [Optionen] [Rechnername]
  \end{verbatim}

  Bei Aufruf ohne Optionen zeigt arp die MAC-Adressen an, die Rechnername
  zugeordnet sind. Fehlt Rechnername, werden alle bekannten Adresszuordnungen
  angezeigt. Rechnername kann dabei ein Hostname sein, ein FQDN oder eine
  IP-Adresse.

  Die für die Fehlersuche wichtigsten Optionen sind:
  \begin{description}
    \item[-n] damit unterlässt das Programm die Namensauflösung bei der
      Anzeige
    \item[-d Rechnername] entfernt alle Einträge für Rechnername
    \item[-s Rechnername Hardwareadresse] setzt den Eintrag für Rechnername
      auf Hardwareadresse
  \end{description}
  Für weitergehende Informationen verweise ich wieder auf die Handbuchseite.
\end{normaltext}

\section{ifconfig}
\label{sec:netz-werkzeuge-ifconfig}
\begin{abstractsec}
  Das Programm ifconfig dient traditionell der Konfiguration der
  Netzwerkschnittstellen. Außerdem liefert es Informationen über den aktuellen
  Zustand und die Konfiguration der Netzwerkschnittstelle.
\end{abstractsec}
\begin{normaltext}
  Das Programm ifconfig dient traditionell der Konfiguration der
  Netzwerkschnittstellen. Außerdem liefert es Informationen über den aktuellen
  Zustand und die Konfiguration der Netzwerkschnittstelle.

  Es gibt drei prinzipielle Aufrufmöglichkeiten für ifconfig. Mit
  \begin{verbatim}
# ifconfig [-a]
  \end{verbatim}
  zeigt das Programm den Status der aktiven Schnittstellen (ohne \verb?-a?)
  beziehungsweise aller (aktiven und nichtaktiven) Schnittstellen (mit
  \verb?-a?) an.

  Durch Aufruf von
  \begin{verbatim}
# ifconfig $schnittstellenname
  \end{verbatim}
  zeigt das Programm den Zustand genau der angegebenen Schnittstelle an.

  In der dritten Form,
  \begin{verbatim}
# ifconfig $schnittstellenname $optionen
  \end{verbatim}
  wird die Schnittstelle konfiguriert.

  Die wichtigsten Optionen beim Troubleshooting sind
  \begin{description}
    \item[up] zum Aktivieren von Schnittstellen
    \item[down] zum Deaktivieren
    \item[mtn N] zum Setzen der Maximum Transfer Unit
    \item[netmask M] zum Setzen der Netzmaske
    \item[hw A] zum Setzen der Hardware-Adresse
  \end{description}

  Weitere Informationen liefert die Handbuchseite.

  Es ist möglich, an eine Netzwerkschnittstelle mehrere IP-Adressen zu binden.
  Das Programm ifconfig arbeitet jedoch mit einer Adresse pro Schnittstelle.
  Um weitere Adressen an diese Schnittstelle zu binden, fügt man an den
  Schnittstellennamen einen Doppelpunkt und eine Zahl an.

  In letzter Zeit werden die Netzwerkschnittstellen auch mit dem Programm ip
  vom Paket iproute2 konfiguriert, welches in Sektion
  \ref{sec:netz-werkzeuge-iproute} beschrieben ist.
  Für den schnellen Überblick über die aktuelle Schnittstellenkonfiguration
  gefällt mir die Ausgabe von \verb?ip addr show? besser, da sie kompakter
  ist, was sich insbesondere dann auszahlt, wenn ein Rechner mehrere
  Netzwerkschnittstellen hat, oder mehrere Adressen auf einer Schnittstelle.

  Hier habe ich die Ausgabe für einen Rechner:
  \begin{verbatim}
$ /sbin/ifconfig -a
eth0      Link encap:Ethernet  HWaddr 00:01:6c:6f:c5:d6  
          inet addr:192.0.2.5  Bcast:192.0.2.255  Mask:255.255.255.0
          inet6 addr: 2001:db8::201:6cff:fe6f:c5d6/64 Scope:Global
          inet6 addr: fe80::201:6cff:fe6f:c5d6/64 Scope:Link
          UP BROADCAST RUNNING MULTICAST  MTU:1500  Metric:1
          RX packets:12259377 errors:0 dropped:0 overruns:0 frame:0
          TX packets:13256939 errors:0 dropped:0 overruns:0 carrier:1
          collisions:0 txqueuelen:1000 
          RX bytes:3104599402 (2.8 GiB)  TX bytes:1370040423 (1.2 GiB)
          Interrupt:30 

eth0:0    Link encap:Ethernet  HWaddr 00:01:6c:6f:c5:d6  
          inet addr:192.0.2.246  Bcast:192.0.2.255  Mask:255.255.255.0
          UP BROADCAST RUNNING MULTICAST  MTU:1500  Metric:1
          Interrupt:30 

lo        Link encap:Local Loopback  
          inet addr:127.0.0.1  Mask:255.0.0.0
          inet6 addr: ::1/128 Scope:Host
          UP LOOPBACK RUNNING  MTU:16436  Metric:1
          RX packets:5386162 errors:0 dropped:0 overruns:0 frame:0
          TX packets:5386162 errors:0 dropped:0 overruns:0 carrier:0
          collisions:0 txqueuelen:0 
          RX bytes:668745964 (637.7 MiB)  TX bytes:668745964 (637.7 MiB)

wlan0     Link encap:Ethernet  HWaddr 0c:60:76:7c:3e:25  
          BROADCAST MULTICAST  MTU:1500  Metric:1
          RX packets:0 errors:0 dropped:0 overruns:0 frame:0
          TX packets:0 errors:0 dropped:0 overruns:0 carrier:0
          collisions:0 txqueuelen:1000 
          RX bytes:0 (0.0 B)  TX bytes:0 (0.0 B)
  \end{verbatim}
  Die Ausgabe von \verb?ip? ist demgegenüber wesentlich kompakter.
  \begin{verbatim}
$ ip addr show
1: lo: <LOOPBACK,UP,LOWER_UP> mtu 16436 qdisc noqueue state UNKNOWN 
    link/loopback 00:00:00:00:00:00 brd 00:00:00:00:00:00
    inet 127.0.0.1/8 scope host lo
    inet6 ::1/128 scope host 
       valid_lft forever preferred_lft forever
2: eth0: <BROADCAST,MULTICAST,UP,LOWER_UP> mtu 1500 qdisc pfifo_fast state UP qlen 1000
    link/ether 00:01:6c:6f:c5:d6 brd ff:ff:ff:ff:ff:ff
    inet 192.0.2.246/24 brd 192.0.2.255 scope global eth0:0
    inet 192.0.2.5/24 brd 192.0.2.255 scope global secondary eth0
    inet6 2001:db8:201:6cff:fe6f:c5d6/64 scope global dynamic 
       valid_lft 86345sec preferred_lft 14345sec
    inet6 fe80::201:6cff:fe6f:c5d6/64 scope link 
       valid_lft forever preferred_lft forever
3: wlan0: <BROADCAST,MULTICAST> mtu 1500 qdisc noop state DOWN qlen 1000
    link/ether 0c:60:76:7c:3e:25 brd ff:ff:ff:ff:ff:ff
  \end{verbatim}

\end{normaltext}

\section{iproute}
\label{sec:netz-werkzeuge-iproute}
\begin{abstractsec}
  Die Programme der iproute-Suite sind moderne Werkzeuge zur Anzeige und
  Konfiguration von Netzwerkschnittstellen, -routen und -verkehrskontrolle.

  Sie bieten gegenüber den Programmen der net-tools-Suite erweiterte
  Möglichkeiten.
\end{abstractsec}
\begin{normaltext}
  Die Programme der iproute-Suite sind moderne Werkzeuge zur Anzeige und
  Konfiguration von Netzwerkschnittstellen, -routen und -verkehrskontrolle.

  Sie bieten gegenüber den Programmen der net-tools-Suite erweiterte
  Möglichkeiten.

  Mit dem Kernel kommunizieren sie über die (rt)netlink-Schnittstelle.

  Die iproute-Suite umfasst unter anderem die folgenden Programme:
  \begin{description}
    \item[ip] zum Anzeigen und Konfigurieren von Netzwerkschnittstellen,
      -adressen, -routen, Policy-Regeln, ARP- und NDISC-Einträgen, IP-Tunneln
      und Multicast
    \item[ss] zur Anzeige von Socketstatistiken
    \item[tc] zur Anzeige und Konfiguration von Netzwerkverkehrskontrolle
      (traffic control)
    \item[arpd] einen Userspace-ARP-Daemon
    \item[*stat, rtacct] verschiedene Statistikwerkzeuge
  \end{description}

  Weitere und genauere Informationen stehen in den betreffenden Handbuchseiten
  und der zugehörigen Dokumentation.

  \subsection*{ip}

  Das Program ip dient zum Anzeigen / Manipulieren von Routen, Schnittstellen,
  Policy-Routing und Tunneln. Es ist das Programm aus der Suite, das ich am
  häufigsten interaktiv aufrufe.

  Allgemein sieht der Aufruf des Programms wie folgt aus:
  \begin{verbatim}
  ip [ Optionen ] Objekt [ Befehl [ Argumente ] ]
  \end{verbatim}

  Die Optionen sind allgemeine Modifikatoren, die das Verhalten des Programms
  ändern, wie zum Beispiel \verb?-4? und \verb?-6?, die die Adressfamilie auf
  IPv4 oder IPv6 einschränken.

  Das Objekt gibt an, worüber ich Informationen wünsche, beziehungsweise, was
  ich manipulieren will. Mögliche Objekte sind unter anderem:
  \begin{description}
    \item[link] Netzwerkschnittstellen
    \item[address] Protokoll-Adressen an einer Schnittstelle
    \item[neighbour] ARP- oder NDISC-Einträge
    \item[route] Einträge in der Kernel-Routingtabelle
    \item[rule] Regeln in der Policydatenbank
    \item[maddress] Multicast-Adressen
    \item[mroute] Multicast-Routingeinträge
    \item[tunnel] IP-Tunnel
    \item[monitor] Nachrichten auf der netlink-Schnittstelle des Kernels
  \end{description}

  Der Befehl schließlich gibt die Aktion an, die wir ausführen wollen und wird
  gegebenenfalls von passenden Argumenten gefolgt. Befehl und Argumente sind
  spezifisch für die entsprechenden Objekte.
\end{normaltext}

  \subsection*{ss}

  Das Programm ss (socket statistics) liefert Informationen und Statistiken zu
  Sockets. Ähnliche Informationen kann ich zum Beispiel auch mit netstat
  bekommen, jedoch habe ich bei ss mehr Möglichkeiten zur Filterung, was mir
  insbesondere bei Servern mit vielen Verbindungen zugute kommt.

  Der Aufruf ist wie folgt:
  \begin{verbatim}
  ss [ Optionen ] [ Filter ]
  \end{verbatim}

  Debei zeigt ss ohne Optionen die verbundenen TCP-Sockets an.

  Die wichtigsten Optionen sind unter anderen:
  \begin{description}
    \item[-a] um alle Sockets anzuzeigen
    \item[-l] um nur die Listening Sockets anzuzeigen (diese werden per
      Default ausgelassen)
    \item[-p] um die zugehörigen Prozesse anzuzeigen (dafür benötige ich
      Rootrechte)
    \item[-t | -u | -w | -x] um TCP-, UDP-, Raw- oder Unix-Sockets auszuwählen
  \end{description}

  Der Filter beim Aufruf von ss hat die allgemeine Form:
  \begin{verbatim}
  [ TCP-Status ] [ Ausdruck ]
  \end{verbatim}

  Um nach TCP-Status zu filtern, gebe ich das Schlüsselwort \verb?state? oder
  \verb?exclude? gefolgt von einem der Standard-TCP-Zustandsnamen oder einem
  der folgenden an:
  \begin{description}
    \item[all] für alle Zustände
    \item[bucket] für TCP-Minisockets (TIME-WAIT|SYN-RECV)
    \item[big] für alle außer den Minisockets
    \item[connected] für alle nicht geschlossenen und nicht lauschenden
      Sockets
    \item[synchronized] für alle verbundenen Sockets, die nicht im Zustand
      SYN-SENT sind
  \end{description}

  Falls weder ein \verb?state? noch ein \verb?exclude? Statement vorhanden
  ist, ist die Voreinstellung \verb?all? bei Option \verb?-a? und ansonsten
  alle außer \verb?listening?, \verb?syn-recv?, \verb?time-wait? und
  \verb?closed?.

  Mit dem (booleschen) Ausdruck kann ich nach Adressen und Ports filtern.

  Weitere Optionen und ausführlichere Informationen zur Filterung gibt es in
  der Handbuchseite und im Artikel ``SS Utility: Quick Intro'', der sich bei
  der Dokumentation des iproute-Pakets befindet.
\section{netstat}
\label{sec:netz-werkzeuge-netstat}
\begin{abstractsec}
  Mit netstat ist es möglich essentielle Informationen zur
  Netzwerkkonfiguration und zum aktuellen Zustand der Netzwerksockets und
  Verbindungen eines Rechners zu erfahren.
\end{abstractsec}
\begin{normaltext}
  Ein Werkzeug, dass ich auch bei der Analyse von lokalen Rechnerproblemen
  einsetze und dort bereits vorgestellt habe, ist netstat. Im Bereich
  Netzwerkprobleme kann es viele Informationen liefern, die mir, je nach
  Problemfall weiterhelfen können.

  \subsection*{Sockets}
  \label{sec:netz-werkzeuge-netstat-sockets}

  Rufe ich \verb?netstat? ohne Argumente auf, liefert es mir eine Liste der
  offenen und aktiven Sockets aller konfigurierten Adressfamilien, das
  heisst, der bestehenden Verbindungen.

  Meist interessieren mich nicht alle Adressfamilien, sondern nur ganz
  bestimmte. Dann kann ich diese zum Beispiel mit der Option
  \verb?--protocol=$familie? einschränken. Für \verb?$familie? kan ich in
  einer kommaseparierten Liste die folgenden angeben: \verb?unix?,
  \verb?inet?, \verb?ipx?, \verb?ax25?, \verb?netrom?, \verb?ddp?. Alternativ
  kann ich jeden gewünschten Familiennamen einzeln als Option übergeben:
  \verb?--unix?, \verb?--inet?, \ldots

  In diesem Teil der Buches interessiert mich vor allem die Familie
  \verb?--inet?. Diese kann ich weiter eingrenzen. Mit der Option \verb?-4?
  beziehungsweise \verb?-6? beschränke ich die Ausgabe auf die entsprechende
  Version des Internet Protokolls.

  Außerdem verwende ich
  \begin{description}
    \item[--raw oder -w] wenn ich an Raw-Sockets interessiert bin,
    \item[--tcp oder -t] für TCP-Sockets, und
    \item[--udp oder -u] für UDP-Sockets.
  \end{description}

  Bin ich nur daran interessiert, ob überhaupt ein Prozess an einem bestimmten
  Socket wartet, verwende ich die Option \verb?--listening? beziehungsweise
  \verb?-l?. Diese werden bei der normalen Ausgabe weggelassen. Will ich
  hingegen sowohl die aktiven als auch die lauschenden Sockets erfassen,
  verwende ich die Option \verb?--all? beziehungsweise \verb?-a?.

  \subsection*{Routen}
  \label{sec:netz-werkzeuge-netstat-routen}

  Wenn ich statt an den Sockets eher an den Routen interessiert bin, verwende
  ich die Option \verb?--routes? beziehungsweise \verb?-r?. Damit bekomme ich
  die gleiche Ausgabe, wei mit dem Befehl \verb?route -e?. Auch hier kann ich
  mit \verb?-4? oder \verb?-6? die Protokollversion einschränken.

  Füge ich die Option \verb?-C? hinzu, bekomme ich Informationen aus dem
  Routencache, mit der Option \verb?-F? stattdessen aus der Forwarding
  Information Base (der Routentabelle), aber das ist sowieso die
  Voreinstellung.

  \subsection*{Interfaces}
  \label{sec:netz-werkzeuge-netstat-interfaces}

  Mit der Option \verb?--interfaces? oder \verb?-i? kann ich Informationen
  über die Netzwerk-Interfaces bekommen.

  Ein einfaches \verb?netstat -i? liefert mir in einer übersichtlichen Tabelle
  zu jedem aktiven Interface unter anderem die MTU, die Anzahl der gesendeten
  und empfangenen Datenpakete sowie die Anzahl der Sende- beziehungweise
  Empfangsfehler.

  Kombiniere ich das mit \verb?-e?, bekomme ich die gleiche Ausgabe wie vom
  Program \verb?ifconfig?. Kombiniert mit \verb?-a? werden auch Interfaces
  angezeigt, die nicht im Zustand \verb?UP? sind.

  \subsection*{Multicast-Gruppen}
  \label{sec:netz-werkzeuge-netstat-groups}
  Die Option \verb?--groups? beziehungsweise \verb?-g? liefert mir
  Informationen zur Mitgliedschaft des Rechners in Multicast-Gruppen.

  Auch diese kann ich mit \verb?-4? oder \verb?-6? einschränken.

  \subsection*{Statistiken}
  \label{sec:netz-werkzeuge-netstat-statistics}

  Mit der Option \verb?--statistics? beziehungsweise \verb?-s? zeigt netstat
  zusammengefasste Statistiken für alle Protokolle.

  \subsection*{allgemeine Optionen}
  \label{sec:netz-werkzeuge-netstat-allgemein}

  Abschließen möchte ich diese kleine Vorstellung von netstat mit ein paar
  allgemeinen Optionen, mit denen ich die Ausgabe modifizieren kann.

  Die von mir wohl am meisten eingesetzte Option ist \verb?--numeric?, kurz
  \verb?-n?. Mit dieser Option zeigt netstat numerische statt symbolischer
  Informationen an und das ist insbesondere bei Netzwerkadressen ein immenser
  Geschwindigkeitsvorteil, da sonst unter Umständen etliche DNS-Anfragen mit
  den entsprechenden Verzögerungen gestellt werden, bevor die Ausgabe
  angezeigt werden kann. Natürlich kann man das auch selektiv einstellen mit
  \verb?--numeric-hosts?, \verb?--numeric-ports? und \verb?--numeric-users?.

  Mit der Option \verb?--verbose? oder \verb?-v? kann ich mehr Informationen
  bekommen, insbesondere zu nicht konfigurierten Adressfamilien.

  Ähnliches bietet die Option \verb?--extend? oder \verb?-e?, die zusätzliche
  Informationen zum Beispiel bei Interfaces liefert.

  Gebe ich die Option \verb?--continuos? oder \verb?-c? an, bekomme ich die
  Informationen aller Sekunde ausgegeben.
\end{normaltext}

\section{perl}
\label{sec:lokal-werkzeuge-perl}
\begin{abstractsec}
  Für knifflige Probleme, die ich mit den spezialisierten Werkzeugen nicht zu
  fassen kriege und denen mit einfacher Shell-Programmierung auch nicht
  beizukommen ist, benötige ich eine Programmiersprache, die mächtiger als die
  Shell ist, mit der ich aber trotzdem mit wenig Aufwand ein passendes
  Programm schreiben kann. Insbesondere durch die vielen verfügbaren Module
  auf CPAN kann ich damit relativ schnell eine Speziallösung für vertrackte
  Probleme zusammenbauen.
\end{abstractsec}
\begin{normaltext}
  Perl hatte ich als Werkzeug bereits im Abschnitt über die Werkzeuge zur
  lokalen Fehlersuche beschrieben. Durch die vielen einfach verfügbaren und
  meist sehr gut getesteten und dokumentierten Module auf CPAN ist Perl auch
  ein unentbehrliches Werkzeug für die Fehlersuche bei Netzwerkproblemen.

  Das Perl Kochbuch \cite{ChristiansenTorkington04de} hatte ich bereits
  erwähnt. Mit dessen Hilfe und den darin beschriebenen Modulen von CPAN war
  es mir zum Beispiel möglich ein Testprogramm für ein Timing-Problem bei
  einem Webservice zu schreiben.

  \subsection*{HTTP Injector}

  Vor einiger Zeit hatte ich ein Problem, bei dem 502-Fehler von
  einem Webservice abhängig waren von der Zeit für die
  Anfrage. Der Betreiber des Webservices stritt das ab und um das
  Problem zu verifizieren benötigte ich die Möglichkeit HTTP-Anfragen gezielt
  zu verzögern.
  
  Ich kam mit Hilfe des Kochbuches zu folgendem Programm:
  \begin{verbatim}
01:#!/usr/bin/perl
02:use Getopt::Long;
03:use IO::Socket;
04:use Time::HiRes qw(sleep);
05:
06:my %opt = ( delay => 0 );
07:
08:GetOptions( \%opt, 'delay=i');
09:
10:my $server = shift;
11:my $port   = shift || 80;
12:
13:my $socket = IO::Socket::INET->new(PeerAddr => $server,
14:                                   PeerPort => $port,
15:                                   Proto    => 'tcp',
16:                                   Type     => SOCK_STREAM);
17:
18:my @in = <>;
19:my $del = $opt{delay} / ( 1.0 + scalar @in );
20:foreach (@in) {
21:    s/[\r\n]+$//;
22:    sleep $del;
23:    print $socket $_, "\r\n";
24:}
25:sleep $del;
26:print $socket "\r\n";
27:
28:while (my $line = <$socket>) {
29:    print $line;
30:}
  \end{verbatim}
  In den Zeilen 2-4 lade ich die benötigten Module. \verb?Getopt::Long? ist
  für die Verarbeitung der Kommandozeilenoptionen und sichert ab, dass ich mit
  \verb?--delay? einen Integerwert angebe. \verb?IO::Socket? stellt die
  Socketfunktionalität bereit, so dass ich diesen Socket wie eine Datei
  verwenden kann. \verb?Time::HiRes? stellt mir eine verbesserte
  \verb?sleep()? Funktion bereit, die mit Gleitkommazahlen zurechtkommt.

  In Zeile 6 stelle ich die Option \verb?--delay? auf den Wert 0 ein, falls
  sie nicht explizit angegeben wird. In Zeile 8 werden die Optionen
  eingelesen.

  Zeile 10 und 11 entnehmen den Server und gegebenenfalls den Port der
  Kommandozeile und in Zeile 13 öffne ich mit diesen Angaben den Socket.

  In Zeile 18 lese ich die gesamte Eingabe in ein Array ein. Dies benötige
  ich, da ich die Anzahl der Zeilen wissen muss, denn ich verzögere das Senden
  zeilenweise um jeweils einen Bruchteil der Gesamtverzögerung. Die Zeilen
  20-25 schließlich bereiten die Zeilenenden auf und senden die modifizierten
  Zeilen verzögert über den Socket. Zeile 26 schickt die Leerzeile, nach der
  der Server antwortet.

  In Zeile 28-30 liest das Skript die Antwort des Servers vom Socket und
  schreibt sie zur Standardausgabe.

  Dieses Skript kann ich nun wie folgt aufrufen:
  \begin{verbatim}
time ./http-injector.pl --delay 5 localhost 80 < request > reply

real  0m5.072s
user  0m0.056s
sys   0m0.012s
  \end{verbatim}
  Dabei steht in der Datei request die HTTP-Anfrage, die ich an den Server
  sende.
  Nach fünf Sekunden ist die Anfrage beim Server, und die Antwort landet in
  der Datei reply.

  Damit konnte ich nachweisen, dass dieselbe Anfrage einen Fehler
  lieferte, wenn sie mehr als drei Sekunden zur Übertragung brauchte und
  fehlerfrei beantwortet wurde, wenn sie weniger als drei Sekunden brauchte.
\end{normaltext}

%%% Local Variables: 
%%% mode: latex
%%% TeX-master: "troubleshoot-linux"
%%% End: 
%%% vim: set sw=2 ts=2 tw=78 et si:
