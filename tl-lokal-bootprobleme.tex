%% tl-lokal-bootprobleme.tex
\chapter{Bootprobleme}
\label{cha:bootprobleme}

\begin{abstractsec}
  Probleme beim Booten des Rechners betrachte ich als lokalen Totalausfall.
\end{abstractsec}

\begin{notes}
\item Partitionen von Festplattenimage einhängen
%\item geht auch nicht
%\item und auch das
\end{notes}

\section{Bootprobleme bei virtuellen Maschinen}
\label{sec:lokal-bootprobleme-vm}

\subsection*{Partitionen von Festplattenimages einhängen}
\label{sec:mount-hdimage-partition}

Bei Problemen mit dem Start von virtuellen Maschinen ist es oft notwendig,
wenigstens die Bootpartition der betroffenen VM zu untersuchen.

Manchmal kann
ich dazu das Festplattenimage der defekten VM einfach einer anderen VM
zuordnen und es von dieser aus untersuchen. Das erscheint zumindest bei
grafischen Oberflächen für die Administration als der einfachste Weg.
Allerdings muss ich dann immer zwischen dem Hostsystem beziehungsweise der
Administrationsoberfläche und der VM, mit der ich das Festplattenimage
untersuche wechseln und verliere viel Zeit beim Zuordnen, Ein- und Aushängen
und den Startversuchen.

Besser ist in meinen Augen, die Partitionen des betroffenen Images gleich im
Hostsystem für die Analyse und Störungsbeseitigung einzubinden und die
Vorteile der Shell für zügiges Arbeiten zu nutzen.

Nun habe ich aber das Problem, dass ich auf dem Hostsystem nicht, wie bei den
eigenen Festplatten, die Partitionen direkt zur Verfügung habe, sondern nur
das Komplettimage der Festplatte für die VM. Und dieses beginnt nicht mit dem
Dateisystem sondern mit der Partitionstabelle. Um die gewünschte Partition
einzubinden, muss ich dem Mount-Befehl den Offset der Partition mitgeben.

Den Offset kann ich mit dem Programm fdisk bestimmen. Dieses listet mit der
Option \verb?-l? die Partitionen und deren Offsets auf. Da wir letztere genau
bestimmen müssen, verwenden wir zusätzlich die Option \verb?-u?, damit fdisk
die Offsets als Anzahl von Sektoren zu je 512 Byte ausgibt:

\begin{verbatim}
# fdisk -l -u /dev/camion/ssh3 

Disk /dev/camion/ssh3: 4294 MB, 4294967296 bytes
255 heads, 63 sectors/track, 522 cylinders, total 8388608 sectors
Units = sectors of 1 * 512 = 512 bytes
Sector size (logical/physical): 512 bytes / 4096 bytes
I/O size (minimum/optimal): 4096 bytes / 4096 bytes
Disk identifier: 0x000c48f7

            Device Boot      Start         End      Blocks   Id  System
/dev/camion/ssh3p1            2048     7706623     3852288   83  Linux
/dev/camion/ssh3p2         7708670     8386559      338945    5  Extended
Partition 2 does not start on physical sector boundary.
/dev/camion/ssh3p5         7708672     8386559      338944   82  Linux swap
\end{verbatim}

Der Offset für die Systempartition \verb?ssh3p1? ist $512 * 2048$, also
1048576. Damit kann ich diese Partition im Hostsystem wie folgt einhängen:

\begin{verbatim}
# mount /dev/camion/ssh3 /tmp/mnt -o loop,offset=1048576
\end{verbatim}

Wenn ich fertig bin, hänge ich die Partition normal mit umount wieder aus.

Wichtig ist, dass ich die Partition im Hostsystem nur einhänge, wenn die VM
nicht läuft. Das ist beim Untersuchen von Bootproblemen meist gegeben. Bei
laufenden VMs habe ich mit LVM die Möglichkeit, einen Snapshot anzufertigen
und diesen Snapshot nur-lesend einzuhängen. Dabei muss ich aber bedenken, dass
Dateien, die in der VM geöffnet waren, eventuell in einem inkonsistenten
Zustand sind. Für Backups kann ich jedoch die VM kurz runterfahren, den
Snapshot anlegen und die VM gleich wieder starten.

%\begin{abstractsec}
%  Die Aufteilung der Umsetzung wird hier gegliedert in verschiedene Aspekte
%\end{abstractsec}
%\begin{normaltext}
%  Blablabla
%\end{normaltext}

%\subsection{Einschränkung der Umsetzung}
%\label{sec:einschr-der-umsetz}

%\section{Schnittstellen nach außen}
%\label{sec:schn-nach-au3en}

%\subsection{Schnittstelle zu A}
%\label{sec:schnittstelle-zu}

%\begin{notes}
%\item Syntax
%\item Semantik
%\end{notes}

%\subsection{Schnittstelle zu B}
%\label{sec:schnittstelle-zu-b}

%\begin{notes}
%\item Syntax
%\item Semantik
%\end{notes}


%%% Local Variables: 
%%% mode: latex
%%% TeX-master: "arbeit-hauptdatei"
%%% End: 
%%% vim: set sw=2 ts=2 tw=78 et si:
