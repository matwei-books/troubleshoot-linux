%% tl-lokal-programmfehler.tex
\chapter{Partielle Ausfälle}
\label{cha:partielle-ausfaelle}

\begin{abstractsec}
  Der Rechner läuft, aber ein Dienst will um's Verrecken nicht starten. Ein
  Programm bricht immer mit Fehlermeldung ab. Ein Verzeichnis läßt sich nicht
  einhängen. Das sind Beispiele für partielle Ausfälle auf dem lokalen System,
  dem Linux-Rechner.
\end{abstractsec}

\section{Mount-Fehler}
\label{sec:mount-fehler}

\begin{abstractsec}
  Eine Klasse für sich sind Fehler beim Ein- und Aushängen von Dateisystemen.
  Seien das Dateisysteme auf mobilen Datenträgern oder über das Netz bezogene
  Dateisysteme
\end{abstractsec}
%\begin{normaltext}
%  Blablabla
%\end{normaltext}

\subsection{mount: / is busy}
\label{sec:mount-is-busy}
\begin{normaltext}
  Dieser Fehler tritt mitunter bei umount auf, aber auch, wenn ich ein
  Dateisystem von read-write auf read-only umhängen will.
  Die Meldung \verb?is busy? deutet es schon an, der Kernel, genauer gesagt
  das Dateisystem ist (noch) beschäftigt. Wenn ich das Dateisystem aushängen
  will, reicht ein Prozess, der auf irgendeine Art auf das Dateisystem
  zugreift. Will ich es read-only umhängen, muss ich nach Prozessen suchen,
  die Dateien zum Schreiben geöffnet haben. In \cite{weidner12:linuxkopflos}
  habe ich einen solchen Fall detailliert beschrieben, hier gehe ich nur kurz
  auf die Schritte ein um die betreffenden Prozesse und die geöffneten Dateien
  zu finden. Die nächsten Schritte sind dann vom konkreten Fall abhängig.

  \begin{Exkursbox}{Offene gelöschte Dateien}
    Übrigens, auch Dateilöschungen sind Schreibzugriffe. Wenn
    Systembibliotheken aktualisiert werden, dann wird die alte Datei mit dem
    Systembefehl \verb?unlink()? gelöscht und die neue Datei mit dem gleichen
    Namen gespeichert. Neu gestartete Prozesse verwenden dann die neue
    Bibliothek. Prozesse, die bereits vor der Aktualisierung liefen, arbeiten
    weiter mit der alten Bibliothek, weil sie diese ja vor der Aktualisierung
    geöffnet hatten.
    
    Abgesehen von den möglichen Sicherheitsproblemen, die die alte Bibliothek
    vielleicht noch hat und damit auch die alten Prozesse, kann ich nun das
    Dateisystem mit der alten Bibliothek nicht read-only umhängen. Denn die
    alte Bibliothek ist zwas nicht mehr im Dateisystem gelinkt, die
    betreffenden Blöcke können aber erst freigegeben werden, wenn der letzte
    Prozess, der sie verwendet, beendet ist. Und erst danach kann ich das
    Dateisystem read-only umhängen.

    Das gleiche gilt für Dateien, die geöffnet werden und unmittelbar darauf
    im Dateisystem gelöscht. Diese sind damit nur noch für den Prozess, der
    sie geöffnet hat, verwendbar und die zugehörigen Speicherblöcke werden
    erst mit Beenden dieses Prozesses im Dateisystem freigegeben.
  \end{Exkursbox}

  Prozesse, die in irgendeiner Weise auf ein Dateisystem zugreifen, finde ich
  am schnellsten mit:
\begin{verbatim}
# fuser -vm $mntpnt
\end{verbatim}
  Diesen Befehl rufe ich mit Superuserprivilegien auf, um alle Prozesse
  angezeigt zu bekommen.
  Dabei interessieren mich neben der PID und dem Prozessnamen (COMMAND) vor
  allem die Angaben unter ACCESS. Insbesondere Prozesse mit \verb?F? und
  \verb?m?, da diese es sind, die das read-only Umhängen des Dateisystems
  verhindern.
  Ich könnte diese Prozesse mit Option \verb?-k? gleich von fuser beenden
  lassen, aber besser ist es, erst mit \verb?lsof -p $pid? nachzuschauen,
  welcher Prozess das genau ist und welche Dateien er konkret offen hält.
  Auch kann ich erstmal mit \verb?pstree -p? abschätzen, ob ich vielleicht
  vitale Systemfunktionen beende, wenn ich den Prozess einfach abschieße.
  Handelt es sich um einen Systemdienst, wie SSH, dann ist meist der
  Listening-Daemon bereits neu gestartet und nur die Instanz, über die ich
  angemeldet bin, verwendet noch die alte Bibliothek. Dann reicht es mitunter,
  wenn ich mich ein zweites Mal anmelde und danach die alte Verbindung trenne.
\end{normaltext}

%\subsection{Einschränkung der Umsetzung}
%\label{sec:einschr-der-umsetz}

%\begin{notes}
%\item Was nicht geht
%\item geht auch nicht
%\item und auch das
%\end{notes}

%\section{Schnittstellen nach außen}
%\label{sec:schn-nach-au3en}

%\subsection{Schnittstelle zu A}
%\label{sec:schnittstelle-zu}

%\begin{notes}
%\item Syntax
%\item Semantik
%\end{notes}

%\subsection{Schnittstelle zu B}
%\label{sec:schnittstelle-zu-b}

%\begin{notes}
%\item Syntax
%\item Semantik
%\end{notes}


%%% Local Variables: 
%%% mode: latex
%%% TeX-master: "troubleshoot-linux"
%%% End: 
%%% vim: set sw=2 ts=2 tw=78 et si:
