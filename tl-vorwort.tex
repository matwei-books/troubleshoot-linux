%% tl-vorwort.tex
%% vim: set sw=2 ts=2 tw=78 et si:
\chapter{Vorwort}
\label{cha:vorwort}

\begin{abstractsec}
  Hier wird die Einführung stehen.
\end{abstractsec}
\begin{normaltext}
  In diesem Buch lege ich meine über die Jahre gesammelten Erfahrungen mit
  der Fehlersuche bei Linux-Servern und in IP-Netzwerken nieder.

  Ich habe diese beiden Themen kombiniert, weil zum Einen Linux-Server fast
  immer über IP-Netzwerke angesprochen werden und daher auf ein gut
  funktionierendes Netzwerk angewiesen sind und zum Anderen Linux-Rechner
  (da gehen auch Arbeitsstationen) für mich ideal zur Analyse von
  Netzwerkproblemen geeignet sind.

  Zwar benutze ich fast täglich Linux auf dem Desktop, allerdings sind hier
  die am häufigsten verwendeten Programme ein Terminal und ein Webbrowser, die
  in den meisten Fällen vorzüglich funktionieren, so dass ich kaum Gelegenheit
  hatte Erfahrungen mit der Fehlersuche am Linux-Desktop zu sammeln. Aus
  diesem Grund bleibt dieser Bereich aussen vor.
\end{normaltext}
\begin{notes}
\item Motivation
\item Aufbau
\item Danksagung
\end{notes}

\section*{Für wen ist dieses Buch}
\label{sec:fuerwen}

\begin{normaltext}
\end{normaltext}

\section*{Übersicht}
\label{sec:ubersicht}

\begin{normaltext}
  Dieses Buch ist in vier Teile gegliedert:
  \begin{itemize}
  \item Teil 1 beschäftigt sich mit dem grundlegenden Vorgehen bei der
    Fehleranalyse.
  \item In Teil 1 geht es vorwiegend um lokale Probleme an einem
    Linux-Rechner.
  \item Teil 3 befasst sich mit Netzwerkproblemen.
  \item In Teil 4 geht es um die Nachbereitung eines gelösten und die
    Vorbereitung auf das nächste Problems.
  \end{itemize}
\end{normaltext}

\section*{Danksagung}
\label{sec:problemstellung}

\begin{abstractsec}
  Wer alles geholfen hat.
\end{abstractsec}
\begin{normaltext}
  Diese haben alle geholfen.
\end{normaltext}

%%% Local Variables: 
%%% mode: latex
%%% TeX-master: "troubleshoot-linux"
%%% End: 
