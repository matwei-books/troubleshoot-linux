%% tl-vorwort.tex
%% vim: set sw=2 ts=2 tw=78 et si:
\chapter{Vorwort}
\label{cha:vorwort}

\begin{abstractsec}
  Hier wird die Einführung stehen.
\end{abstractsec}
\begin{normaltext}
  Ich habe bereits viele Jahre mit Fehlersuche verbracht.
  Sei es als Kind beim Basteln, wenn etwas nicht so funktionierte, wie ich
  dachte. In meinem ersten Beruf bei der Reparatur von Telefonen und
  Telefonanlagen, die damals noch mit Relais funktionierten, so dass man ihnen
  im wahrsten Sinne des Wortes bei der Arbeit zusehen konnte. Nach dem
  Studium, als Softwareentwickler, beim Debuggen meiner oder fremder
  Programme. Oder später dann als Systemadministrator für UNIX und Netzwerke.

  Dabe war ich vor allem auf einige Dinge angewiesen um zum Erfolg zu kommen.
  Das erste und wichtigste ist ein zumindest grundlegendes Verständnis der
  Materie und die Fähigkeit und Möglichkeit, weitere nötige Kenntnisse zu
  erwerben.
  
  Das zweite ist eine einigermaßen strukturierte Vorgehensweise, die
  mich zumindest in den meisten Fällen schneller zum Ziel führt als andere
  Vorgehensweisen.

  \begin{Exkursbox}{Wissenschaftliches versus strukturiertes Vorgehen}
  Ich unterscheide hier zwischen dem wissenschaftlichen Vorgehen, einem
  methodischen Untersuchen aller Möglichkeiten zwecks Erkenntnisgewinn und dem
  zielgerichteten strukturierten Vorgehen, bei dem ich zunächst die
  Möglichkeiten untersuche, die vermutlich am ehesten zum Erfolg führen zwecks
  schnellstmöglicher Problembeseitigung.
  \end{Exkursbox}

  Eine dritte Voraussetzung ist die entsprechende geistige Einstellung. Ich
  musste sehr oft und werde vermutlich noch oft die Erfahrung machen, dass ich
  mir selbst bei der Lösung eines Problems im Weg stehe:

  \begin{itemize}
    \item Sei es, dass ich mir vor der Kenntnis aller relevanten Tatsachen
      bereits eine (dann eventuell falsche) Meinung gebildet hatte.
    \item Sei es, dass ich durch Druck von außen oder selbst erzeugten Druck
      besonders schnell sein wollte und dann beim oberflächlichen Hinsehen
      wichtige Details übersah.
    \item Sei es, dass ich angebotene Hilfe nicht annahm, weil ich es selbst
      schaffen wollte oder so sehr im Problem gefangen war, das ich das
      Hilfsangebot nicht wahrnahm.
    \item Sei es, dass ich kurz vor dem Ziel aufgab und einfach nicht den
      letzten Meter gegangen war.
  \end{itemize}

  In diesem Buch lege ich meine über die Jahre gesammelten Erfahrungen mit
  der Fehlersuche bei Linux-Servern und in IP-Netzwerken nieder.

  Ich habe diese beiden Themen kombiniert, weil zum Einen Linux-Server fast
  immer über IP-Netzwerke angesprochen werden und daher auf ein gut
  funktionierendes Netzwerk angewiesen sind und zum Anderen Linux-Rechner
  (da gehen auch Arbeitsstationen) für mich ideal zur Analyse von
  Netzwerkproblemen geeignet sind.

  Zwar benutze ich fast täglich Linux auf dem Desktop, allerdings sind hier
  die am häufigsten verwendeten Programme ein Terminal und ein Webbrowser, die
  in den meisten Fällen vorzüglich funktionieren, so dass ich kaum Gelegenheit
  hatte Erfahrungen mit der Fehlersuche am Linux-Desktop zu sammeln. Aus
  diesem Grund bleibt dieser Bereich aussen vor.
\end{normaltext}
\begin{notes}
\item Meine Motivation ist das mitunter erschreckende Unverständnis einiger
  meiner Kollegen beim einfachen Fehlersuchen. Ich extrapoliere, dass es auch
  anderen so gehen könnte.
\item Die Zielgruppe ist angehende (und gestandene?) Systemadministratoren für
  Linux und/oder IP-Netzwerke. Für diese muss ich die meisten Gedanken
  transparent machen und entsprechende Beispiele bringen.
\item Beim Text und Stil brauche ich mir keine persönlichen und
  institutionellen Beschränkungen auferlegen. Der Stil sollte dennoch
  weitgehend neutral sein, in den Beispielen ist die erste (und dritte)
  Person angemessen.
\item Ich werde grundlegende (strategische) Methoden, die ich bei der
  Fehlersuche verwende, vorstellen.
\item Ich werde einige (taktische) Programme vorstellen, die mir bei der
  Fehlersuche helfen.
\item Ich werde die operative Anwendung der grundlegenden Methoden und der
  Programme in einigen konkreten Fehlersituationen vorstellen.
\item Ich werde mich auf die Gebiete Linux-Server und (IPv4) Netzwerke
  beschränken. Grafische Oberflächen bleiben (zunächst) außen vor.
\item Danksagung
\end{notes}

\section*{Für wen ist dieses Buch}
\label{sec:fuerwen}

\begin{normaltext}
\end{normaltext}

\section*{Übersicht}
\label{sec:ubersicht}

\begin{normaltext}
  Dieses Buch ist in vier Teile gegliedert:
  \begin{itemize}
  \item Teil 1 beschäftigt sich mit dem grundlegenden Vorgehen bei der
    Fehleranalyse.
  \item In Teil 1 geht es vorwiegend um lokale Probleme an einem
    Linux-Rechner.
  \item Teil 3 befasst sich mit Netzwerkproblemen.
  \item In Teil 4 geht es um die Nachbereitung eines gelösten und die
    Vorbereitung auf das nächste Problems.
  \end{itemize}
\end{normaltext}

\section*{Danksagung}
\label{sec:problemstellung}

\begin{abstractsec}
  Wer alles geholfen hat.
\end{abstractsec}
\begin{normaltext}
  Diese haben alle geholfen.
\end{normaltext}

%%% Local Variables: 
%%% mode: latex
%%% TeX-master: "troubleshoot-linux"
%%% End: 
