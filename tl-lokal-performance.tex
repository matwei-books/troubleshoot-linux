%% tl-lokal-performance.tex
\chapter{Performance des Rechner}
\label{cha:lokal-performance}

\begin{abstractsec}
  Der Rechner funktioniert zwar, aber alles erscheint zäh und in Zeitlupe
  abzulaufen. Jobs brauchen eine gefühlte Ewigkeit, die Tastatur scheint mit
  8 Baud angeschlossen. Irgendwo sind Engpässe, aber wo?
\end{abstractsec}

\begin{notes}
\item CPU?
\item Speicher?
\item Input/Output?
\item ALIX performance with an internal disk => Tests HD Performancemessung
  mit hdparm
\end{notes}

\section{Input/Output-Performance}
\label{sec:input-output-performance}

%\begin{abstractsec}
%  Die Aufteilung der Umsetzung wird hier gegliedert in verschiedene Aspekte
%\end{abstractsec}
%\begin{normaltext}
%  Blablabla
%\end{normaltext}

\subsection{Festplatten}
\label{sec:performance-festplatten}

\begin{notes}
\item HD Performancemessung mit hdparm (ALIX performance with an
  internal disk)
\end{notes}

%\subsection{Einschränkung der Umsetzung}
%\label{sec:einschr-der-umsetz}

%\section{Schnittstellen nach außen}
%\label{sec:schn-nach-au3en}

%\subsection{Schnittstelle zu A}
%\label{sec:schnittstelle-zu}

%\subsection{Schnittstelle zu B}
%\label{sec:schnittstelle-zu-b}

%\begin{notes}
%\item Syntax
%\item Semantik
%\end{notes}


%%% Local Variables: 
%%% mode: latex
%%% TeX-master: "troubleshoot-linux"
%%% End: 
%%% vim: set sw=2 ts=2 tw=78 et si:
