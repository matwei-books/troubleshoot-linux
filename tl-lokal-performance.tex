%% tl-lokal-performance.tex
\chapter{Performance des Rechner}
\label{cha:lokal-performance}

\begin{abstractsec}
  Der Rechner funktioniert zwar, aber alles erscheint zäh und in Zeitlupe
  abzulaufen. Jobs brauchen eine gefühlte Ewigkeit, die Tastatur scheint mit
  8 Baud angeschlossen. Irgendwo sind Engpässe, aber wo?
\end{abstractsec}

\begin{notes}
\item CPU?
\item Speicher?
\item Input/Output?
\item ALIX performance with an internal disk => Tests HD Performancemessung
  mit hdparm
\end{notes}

\section{Input/Output-Performance}
\label{sec:input-output-performance}

%\begin{abstractsec}
%  Die Aufteilung der Umsetzung wird hier gegliedert in verschiedene Aspekte
%\end{abstractsec}
%\begin{normaltext}
%  Blablabla
%\end{normaltext}

\subsection*{Festplatten}
\label{sec:performance-festplatten}

\begin{normaltext}
  Bei vermuteten Performanceproblemen mit Festplatten kann man mit hdparm
  zumindest die Lesegeschwindigkeit kontrollieren. Mit dem Befehl
  \begin{verbatim}
# hdparm -tT /dev/sda
  \end{verbatim}
  bekomme ich einen Überblick über die Lesegeschwindigkeit der ersten
  Festplatte.
  Sinnvollerweise rufe ich den Befehl auf, wenn
  das System sonst unbelastet ist und wiederhole ihn noch ein oder zwei mal.

  Mit der Option \verb?-t? ermittelt hdparm die Lesegeschwindigkeit der Platte
  durch die Buffercaches, ohne dass letztere vorher gefüllt werden. Das zeigt
  an, wie schnell die Festplatte die sequentiell Daten unter Linux liefern
  kann.

  Mit der Option \verb?-T? zeigt hdparm, wie schnell Linux die Daten aus dem
  Buffercache lesen kann, ohne auf die Festplatte zuzugreifen.

  Will ich einen Überblick über die Schreibgeschwindigkeit, kann ich zum
  Beispiel bonnie++ verwenden.

  Bei allen diesen Benchmark gilt, dass sie nur relativ zu anderen Systemen
  sinnvoll sind. Das bedeutet, das man am besten bereits im Vorfeld Benchmarks
  auf verschiedenen Systemen macht und die Rahmenbedingungen so genau wie
  möglich notiert (Betriebssystemversion, CPU, Speicher, Plattenparameter) um
  im Falle eines Problems die Ausgaben interpretieren zu können.
\end{normaltext}

\begin{notes}
\item HD Performancemessung mit hdparm (ALIX performance with an
  internal disk)
\end{notes}

%\subsection{Einschränkung der Umsetzung}
%\label{sec:einschr-der-umsetz}

%\section{Schnittstellen nach außen}
%\label{sec:schn-nach-au3en}

%\subsection{Schnittstelle zu A}
%\label{sec:schnittstelle-zu}

%\subsection{Schnittstelle zu B}
%\label{sec:schnittstelle-zu-b}

%\begin{notes}
%\item Syntax
%\item Semantik
%\end{notes}


%%% Local Variables: 
%%% mode: latex
%%% TeX-master: "troubleshoot-linux"
%%% End: 
%%% vim: set sw=2 ts=2 tw=78 et si:
