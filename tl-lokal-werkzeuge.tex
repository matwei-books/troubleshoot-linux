%% tl-lokal-werkzeuge.tex
\chapter{Werkzeuge}
\label{cha:lokal-werkzeuge}

\begin{abstractsec}
  Verschiedene Werkzeuge helfen mir lokale Probleme einzugrenzen. Hier stelle
  ich die Werkzeuge kurz vor, die ich in den n�chsten drei Kapiteln zur
  Fehlersuche einsetze.
\end{abstractsec}

\begin{notes}
\item Shell
\item Perl
\item strace
\item gdb
\item fuser
\item lsof
\item netstat
\item ifconfig
\item iproute
\end{notes}

%\section{Konzept der Umsetzung}
%\label{sec:konz-der-umsetz}

%\begin{abstractsec}
%  Die Aufteilung der Umsetzung wird hier gegliedert in verschiedene Aspekte
%\end{abstractsec}
%\begin{normaltext}
%  Blablabla
%\end{normaltext}

%\subsection{Seiteneffekte}
%\label{sec:seiteneffekte}

%\subsection{Einschr�nkung der Umsetzung}
%\label{sec:einschr-der-umsetz}

%\begin{notes}
%\item Was nicht geht
%\item geht auch nicht
%\item und auch das
%\end{notes}

%\section{Schnittstellen nach au�en}
%\label{sec:schn-nach-au3en}

%\subsection{Schnittstelle zu A}
%\label{sec:schnittstelle-zu}

%\subsection{Schnittstelle zu B}
%\label{sec:schnittstelle-zu-b}

%\begin{notes}
%\item Syntax
%\item Semantik
%\end{notes}


%%% Local Variables: 
%%% mode: latex
%%% TeX-master: "arbeit-hauptdatei"
%%% End: 
%%% vim: set sw=2 ts=2 tw=78 et si:
