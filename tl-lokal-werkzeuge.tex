%% tl-lokal-werkzeuge.tex
\chapter{Werkzeuge}
\label{cha:lokal-werkzeuge}

\begin{abstractsec}
  Verschiedene Werkzeuge helfen mir lokale Probleme einzugrenzen. Hier stelle
  ich die Werkzeuge kurz vor, die ich in den nächsten drei Kapiteln zur
  Fehlersuche einsetze.
\end{abstractsec}

\begin{normaltext}
  Linux stellt mir eine Unmenge von Werkzeugen für die Fehlersuche zur
  Verfügung. Etliche davon kommen mir sowohl bei Total- oder Partialausfällen
  zu gute. Andere bei Performanceproblemen. Einige sind so nützlich, dass sie
  immer wieder bei den unterschiedlichsten Problemen zum Einsatz kommen. Da
  es mir schwerfällt, die einzelnen Werkzeuge bestimmten Kategorien
  zuzuordnen, stelle ich diese nachfolgend in alphabetischer Reihenfolge vor.
  Das erleichtert zumindest das Wiederfinden, wenn man mal eben etwas schnell
  nachschlagen möchte.
\end{normaltext}

\begin{notes}
\item hdparm
\item fuser
\item gdb
\item ifconfig
\item iproute
\item lsof
\item netstat
\item perl
\item shell
\item strace
\end{notes}

%\section{Konzept der Umsetzung}
%\label{sec:konz-der-umsetz}

%\begin{abstractsec}
%  Die Aufteilung der Umsetzung wird hier gegliedert in verschiedene Aspekte
%\end{abstractsec}
%\begin{normaltext}
%  Blablabla
%\end{normaltext}

%\subsection{Seiteneffekte}
%\label{sec:seiteneffekte}

%\subsection{Einschränkung der Umsetzung}
%\label{sec:einschr-der-umsetz}

%\begin{notes}
%\item Was nicht geht
%\item geht auch nicht
%\item und auch das
%\end{notes}

%\section{Schnittstellen nach außen}
%\label{sec:schn-nach-au3en}

%\subsection{Schnittstelle zu A}
%\label{sec:schnittstelle-zu}

%\subsection{Schnittstelle zu B}
%\label{sec:schnittstelle-zu-b}

%\begin{notes}
%\item Syntax
%\item Semantik
%\end{notes}


%%% Local Variables: 
%%% mode: latex
%%% TeX-master: "arbeit-hauptdatei"
%%% End: 
%%% vim: set sw=2 ts=2 tw=78 et si:
