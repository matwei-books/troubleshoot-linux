%% tl-herangehen.tex
\chapter{Partieller Ausfall im Netz}
\label{cha:netz-teilausfall}

\begin{abstractsec}
  Wenn nur ein oder mehrere Dienste eines Servers nicht funktionieren, oder
  ein Server in einem ansonsten gut erreichbaren Netzsegment, handelt es sich
  um einen partiellen Ausfall im Netz.
\end{abstractsec}

\section{instabile Routen}
\label{sec:instabile-routen}

\begin{notes}
\item Routersoftware Quagga
\item OSPF-BDR hatte falschen Neighbor (openbsd, GeNUScreen) -> Reboot
\end{notes}

%\begin{abstractsec}
%  Die Aufteilung der Umsetzung wird hier gegliedert in verschiedene Aspekte
%\end{abstractsec}
%\begin{normaltext}
%  Blablabla
%\end{normaltext}

%\subsection{Seiteneffekte}
%\label{sec:seiteneffekte}

%\subsection{Einschränkung der Umsetzung}
%\label{sec:einschr-der-umsetz}

%\section{Schnittstellen nach außen}
%\label{sec:schn-nach-au3en}

%\subsection{Schnittstelle zu A}
%\label{sec:schnittstelle-zu}

%\begin{notes}
%\item Syntax
%\item Semantik
%\end{notes}

%\subsection{Schnittstelle zu B}
%\label{sec:schnittstelle-zu-b}

%\begin{notes}
%\item Syntax
%\item Semantik
%\end{notes}


%%% Local Variables: 
%%% mode: latex
%%% TeX-master: "arbeit-hauptdatei"
%%% End: 
%%% vim: set sw=2 ts=2 tw=78 et si:
