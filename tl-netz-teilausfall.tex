%% tl-herangehen.tex
\chapter{Partieller Ausfall im Netz}
\label{cha:netz-teilausfall}

\begin{abstractsec}
  Wenn nur ein oder mehrere Dienste eines Servers nicht funktionieren, oder
  ein Server in einem ansonsten gut erreichbaren Netzsegment, handelt es sich
  um einen partiellen Ausfall im Netz.
\end{abstractsec}

\begin{normaltext}
  Wenn ich einen Teilausfall eines Netzes bemerke beziehungsweise mitgeteilt
  bekomme, interessiert mich als erstes, ob ich in dem betreffenden
  Netzsegment bin oder aussen. In dem Segment bin ich zum Beispiel, wenn der
  erste verfügbare Rechner an das betreffende Segment angeschlossen ist. In
  diesem Fall kann ich meist so vorgehen, prüfe ich als erstes, wie in
  Abschnitt \ref{sec:gar-kein-netz} beschrieben, ob das problematische
  Netzsegment in sich funktioniert. Habe ich mich von der Funktionisfähigkeit
  des Netzsegmentes überzeugt oder es funktionsfähig gemacht, kommt als
  nächstes die Verbindung zu anderen Netzen dran.

  Außerhalb des betreffenden Netzsegmentes bin ich, wenn der nächste Rechner
  in diesem so weit weg ist, dass ich zunächst auf alle Möglichen Arten
  versuche über das Netzwerk an das Netzsegment heran zu kommen, bevor ich
  mich auf den Weg mache. Nun wird ein Ausfall eines Netzsegmentes im
  Allgemeinen dadurch gekennzeichnet, dass das Segment nicht via Netz
  erreichbar ist. Was ich aber tun kann und sollte, ist, mich zu überzeugen,
  dass das auch wirklich so ist und nicht nur so scheint. Das kann ich, indem
  ich mir einen Überblick beschaffe, ob in den Routingprotokollen der Gateways
  eine Übersicht über die bekannten Routen im Netzwerk verschaffe. Taucht die
  Route zu dem Netz dort auf, habe ich eventuell gute Chancen, zumindest das
  Gateway dort zu erreichen und von diesem aus das Netzsegment zu analysieren.
  Taucht die Route dort nicht auf, ist das zwar ein schlechtes Zeichen, aber
  ich kann immer noch versuchen, von einem Router zum nächsten zu hangeln um
  so nah wie möglich an das Netz heranzukommen. Dazu brauche ich aber eine
  genaue Kenntnis des Gesamtnetzwerkes. Dabei kann ich mir unterwegs die Sicht
  jedes einzelnen Gateways auf das Gesamtnetz und seine Routen ansehen und
  eventuell mögliche Routingprobleme finden. Das letzte erreichbare Gateway
  vor dem ausgefallene Netz kann bereits Hinweise auf die Ursache des Ausfalls
  liefern.

  Habe ich mich über das Netzwerk bis zum Gateway an das ausgefallene Netzwerk
  vorgearbeitet, dann prüfe ich mit diesem die grundlegende Funktionsweise des
  Netzsegments, wie in Abschnitt \ref{sec:gar-kein-netz} beschrieben.
  Funktioniert das Netzsegment grundsätzlich, arbeite ich mich  nach oben und
  überprüfe die verschiedenen Netzdienste, wie DHCP, Namensauflösung,
  Zeitserver, Netzanmeldung, soweit vorhanden und eingesetzt.

  \begin{Exkursbox}{Loopback-Adressen}
  \end{Exkursbox}
\end{normaltext}

\section{Instabile Routen}
\label{sec:instabile-routen}

\begin{notes}
\item Routersoftware Quagga
\item OSPF-BDR hatte falschen Neighbor (openbsd, GeNUScreen) -> Reboot
\item Fehlende Rückroute
\end{notes}

%\begin{abstractsec}
%  Die Aufteilung der Umsetzung wird hier gegliedert in verschiedene Aspekte
%\end{abstractsec}

\subsection{OSPF}
\label{sec:ospf}

\begin{normaltext}
  \begin{Beispiel}{statische Route in OSPF}
    Das ist ein Beispiel für ein selbstgemachtes Problem.

    % Hier Bild vom Netzwerk

    In dem angedeuteten Netzwerk wurde der Router C umgeschwenkt vom direkten
    Anschluß im Core-Netz auf einen Anschluß via Router D. Im Core-Netz wurde
    OSPF für das Routing verwendet, zwischen D und C sowie zwischen B und D
    wurde RIP verwendet.

    Später wurde das Core-Netz erweitert und Router D in die Area mit
    aufgenommen. D bekam die Route von C immer noch via RIP. Trotzdem hatte D
    in seiner Routingtabelle B als Router zu C. B hingegen hatte gar keine
    Route zu C.

    Erst ein Blick in die OSPF Route Database (\verb?show ip ospf database?)
    zeigte, dass Router A eine statische Route zu C via B einspeiste.
    Da die Übergabe der Route von RIP nach OSPF zunächst nicht
    richtig funktionierte, wurde auf Router A eine statische Route zu C
    eingespeist, die B als Router auswies. Das funktionierte, weil zu dem
    Zeitpunkt B Borderrouter war und die externe Route über ihn ging. Mit dem
    Ausweiten der Area war B aber nicht mehr Borderrouter und die Notlösung
    mit der statischen Route verursachte den Fehler.

    Insgesamt lassen sich folgende Ursachen, die in diesem Fall zusammen den
    Fehler auftreten ließen, angeben:
    \begin{enumerate}
      \item Das Einspeisen der statischen Route war per se ein Fehler. Besser
        wäre die genaue Diagnose des ursprünglichen Routingproblems. Dieses
        Problem wurde kaschiert, weil B Borderrouter war und die korrekte
        Route auf anderem Weg erfuhr. Erschwerend kam hinzu, dass die
        statische Route nicht auf dem Gateway selbst, sondern auf einem
        anderen Router (der als zentraler Punkt für das Einspeisen von
        statischen Routen gedacht war) erfolgte.
      \item Durch das Erweitern der OSPF-Area war B auf einmal kein
        Borderrouter mehr und konnte die korrekte Route durch den Wegfall von
        RIP auch nicht mehr auf anderem Weg lernen.
      \item Die OSPF-Route hatte auf D eine höhere Priorität als die
        RIP-Route. Darum stellte D sein Routing zu C in Richtung B um und
        konnte das Netz hinter C nicht mehr erreichen.
    \end{enumerate}
    Merke: Bei der Verwendung von Routingprotokollen, sollten statische Routen
    nur in wohlbegründeten Ausnahmefällen eingespeist werden und vor allem -
    wenn möglich - auf dem Router, der für die Route zuständig ist.
  \end{Beispiel}
\end{normaltext}

%\subsection{Einschränkung der Umsetzung}
%\label{sec:einschr-der-umsetz}

%\section{Schnittstellen nach außen}
%\label{sec:schn-nach-au3en}

%\subsection{Schnittstelle zu A}
%\label{sec:schnittstelle-zu}

%\begin{notes}
%\item Syntax
%\item Semantik
%\end{notes}

%\subsection{Schnittstelle zu B}
%\label{sec:schnittstelle-zu-b}

%\begin{notes}
%\item Syntax
%\item Semantik
%\end{notes}


%%% Local Variables: 
%%% mode: latex
%%% TeX-master: "arbeit-hauptdatei"
%%% End: 
%%% vim: set sw=2 ts=2 tw=78 et si:
