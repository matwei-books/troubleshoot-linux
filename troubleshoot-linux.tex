%% Hauptdatei für das Buch Troubleshooting Linux Servers
%%
\documentclass[%
%draft,                        %% um später den Zeilenumbruch zu
%kontrollieren
%a4paper,                       %% DIN A4-Papier
a5paper,                       %% DIN A5-Papier
BCOR4mm,                       %% Bindekorrektur 4 mm
DIVcalc,                       %% Satzspiegel berechnen
11pt,                          %% Schriftgröße
bibtotoc,                      %% Literaturverzeichnis ins Inhaltsverzeichnis
idxtotoc,                      %% Index ins Inhaltsverzeichnis
tablecaptionabove,             %% Tabellenüber``= statt unterschriften
]{scrbook}                     %% KOMA-Skript Book als Klasse

\usepackage{versions}          %% bedingte Kompilierung
\usepackage{ifthen}            %% für \ifthenelse
\usepackage[utf8]{inputenc}    %% UTF8 muss sein

%% bedingte Kompilierung via Makefile
\newcommand{\enableversions}[4]{
  \ifthenelse{\equal{#1}{on}}{
    \newenvironment{abstractsec}{\begin{quotation}\noindent\textbf{Zusammenfassung:}}{\end{quotation}}
  }{\excludeversion{abstractsec}}
  \ifthenelse{\equal{#2}{on}}{\includeversion{normaltext}
  }{\excludeversion{normaltext}}
  \ifthenelse{\equal{#3}{on}}{
    \newenvironment{notes}{\par\hrulefill{}\emph{Ungeordnete partielle
        Ideensammlung (nicht im Druck):}\hrulefill{}\begin{itemize*}}{\end{itemize*}}
  }{\excludeversion{notes}}
  \ifthenelse{\equal{#4}{on}}{\def\svnfinal{final}\includeversion{finalversion}
  }{\def\svnfinal{}\excludeversion{finalversion}}
}
\newcommand{\versionabstracts}{\enableversions{on}{off}{off}{off}}
\newcommand{\versionnotes}{\enableversions{off}{off}{on}{off}}
\newcommand{\versionabstractstext}{\enableversions{on}{on}{off}{off}}
\newcommand{\versiontext}{\enableversions{off}{on}{off}{off}}
\newcommand{\versiontextnotes}{\enableversions{off}{on}{on}{off}}
\newcommand{\versionfinal}{\enableversions{off}{on}{off}{on}}

%% Wählen, was erzeugt werden soll:
\providecommand{\selectversion}{
%\versionabstracts
%\versionnotes
%\versionabstractstext
%\versiontext
%\versiontextnotes
\versionfinal
}

\AtBeginDocument{%
%  \selectlanguage{ngerman}%
%  \renewcommand*{\freffigname}{\figurename}
%  \renewcommand*{\freftabname}{\tablename}
  \selectversion{}
}

%% Hier beginnt das eigentliche Dokument
\begin{document}

%% Die Eckdaten des Dokuments, wegen Babel erst nach \begin{document}
\author{Mathias Weidner}
\title{Troubleshooting Linux Servers}

\maketitle

%% tl-vorwort.tex
%% vim: set sw=2 ts=2 tw=78 et si:
\chapter{Vorwort}
\label{cha:vorwort}

\begin{abstractsec}
  Hier wird die Einführung stehen.
\end{abstractsec}
\begin{normaltext}
  Ich habe bereits viele Jahre mit Fehlersuche verbracht.
  Sei es als Kind beim Basteln, wenn etwas nicht so funktionierte, wie ich
  dachte. In meinem ersten Beruf bei der Reparatur von Telefonen und
  Telefonanlagen, die damals noch mit Relais funktionierten, so dass man ihnen
  im wahrsten Sinne des Wortes bei der Arbeit zusehen konnte. Nach dem
  Studium, als Softwareentwickler, beim Debuggen meiner oder fremder
  Programme. Oder später dann als Systemadministrator für UNIX und Netzwerke.

  Dabe war ich vor allem auf einige Dinge angewiesen um zum Erfolg zu kommen.
  Das erste und wichtigste ist ein zumindest grundlegendes Verständnis der
  Materie und die Fähigkeit und Möglichkeit, weitere nötige Kenntnisse zu
  erwerben.
  
  Das zweite ist eine einigermaßen strukturierte Vorgehensweise, die
  mich zumindest in den meisten Fällen schneller zum Ziel führt als andere
  Vorgehensweisen.

  \begin{Exkursbox}{Wissenschaftliches versus strukturiertes Vorgehen}
  Ich unterscheide hier zwischen dem wissenschaftlichen Vorgehen, einem
  methodischen Untersuchen aller Möglichkeiten zwecks Erkenntnisgewinn und dem
  zielgerichteten strukturierten Vorgehen, bei dem ich zunächst die
  Möglichkeiten untersuche, die vermutlich am ehesten zum Erfolg führen zwecks
  schnellstmöglicher Problembeseitigung.
  \end{Exkursbox}

  Eine dritte Voraussetzung ist die entsprechende geistige Einstellung. Ich
  musste sehr oft und werde vermutlich noch oft die Erfahrung machen, dass ich
  mir selbst bei der Lösung eines Problems im Weg stehe:

  \begin{itemize}
    \item Sei es, dass ich mir vor der Kenntnis aller relevanten Tatsachen
      bereits eine (dann eventuell falsche) Meinung gebildet hatte.
    \item Sei es, dass ich durch Druck von außen oder selbst erzeugten Druck
      besonders schnell sein wollte und dann beim oberflächlichen Hinsehen
      wichtige Details übersah.
    \item Sei es, dass ich angebotene Hilfe nicht annahm, weil ich es selbst
      schaffen wollte oder so sehr im Problem gefangen war, das ich das
      Hilfsangebot nicht wahrnahm.
    \item Sei es, dass ich kurz vor dem Ziel aufgab und einfach nicht den
      letzten Meter gegangen war.
  \end{itemize}

  In diesem Buch lege ich meine über die Jahre gesammelten Erfahrungen mit
  der Fehlersuche bei Linux-Servern und in IP-Netzwerken nieder.

  Ich habe diese beiden Themen kombiniert, weil zum Einen Linux-Server fast
  immer über IP-Netzwerke angesprochen werden und daher auf ein gut
  funktionierendes Netzwerk angewiesen sind und zum Anderen Linux-Rechner
  (da gehen auch Arbeitsstationen) für mich ideal zur Analyse von
  Netzwerkproblemen geeignet sind.

  Zwar benutze ich fast täglich Linux auf dem Desktop, allerdings sind hier
  die am häufigsten verwendeten Programme ein Terminal und ein Webbrowser, die
  in den meisten Fällen vorzüglich funktionieren, so dass ich kaum Gelegenheit
  hatte Erfahrungen mit der Fehlersuche am Linux-Desktop zu sammeln. Aus
  diesem Grund bleibt dieser Bereich aussen vor.
\end{normaltext}
\begin{notes}
\item Meine Motivation ist das mitunter erschreckende Unverständnis einiger
  meiner Kollegen beim einfachen Fehlersuchen. Ich extrapoliere, dass es auch
  anderen so gehen könnte.
\item Die Zielgruppe ist angehende (und gestandene?) Systemadministratoren für
  Linux und/oder IP-Netzwerke. Für diese muss ich die meisten Gedanken
  transparent machen und entsprechende Beispiele bringen.
\item Beim Text und Stil brauche ich mir keine persönlichen und
  institutionellen Beschränkungen auferlegen. Der Stil sollte dennoch
  weitgehend neutral sein, in den Beispielen ist die erste (und dritte)
  Person angemessen.
\item Ich werde grundlegende (strategische) Methoden, die ich bei der
  Fehlersuche verwende, vorstellen.
\item Ich werde einige (taktische) Programme vorstellen, die mir bei der
  Fehlersuche helfen.
\item Ich werde die operative Anwendung der grundlegenden Methoden und der
  Programme in einigen konkreten Fehlersituationen vorstellen.
\item Ich werde mich auf die Gebiete Linux-Server und (IPv4) Netzwerke
  beschränken. Grafische Oberflächen bleiben (zunächst) außen vor.
\item Danksagung
\end{notes}

\section*{Für wen ist dieses Buch}
\label{sec:fuerwen}

\begin{normaltext}
\end{normaltext}

\section*{Übersicht}
\label{sec:ubersicht}

\begin{normaltext}
  Dieses Buch ist in vier Teile gegliedert:
  \begin{itemize}
  \item Teil 1 beschäftigt sich mit dem grundlegenden Vorgehen bei der
    Fehleranalyse.
  \item In Teil 1 geht es vorwiegend um lokale Probleme an einem
    Linux-Rechner.
  \item Teil 3 befasst sich mit Netzwerkproblemen.
  \item In Teil 4 geht es um die Nachbereitung eines gelösten und die
    Vorbereitung auf das nächste Problems.
  \end{itemize}
\end{normaltext}

\section*{Danksagung}
\label{sec:problemstellung}

\begin{abstractsec}
  Wer alles geholfen hat.
\end{abstractsec}
\begin{normaltext}
  Diese haben alle geholfen.
\end{normaltext}

%%% Local Variables: 
%%% mode: latex
%%% TeX-master: "troubleshoot-linux"
%%% End: 

%% allgemeiner Teil
\include{tl-wie-vorgehen}
\include{tl-dokumentation}
%% lokale Probleme
\include{tl-klassifikation-lokal}
\include{tl-programmprobleme}
\include{tl-mount-probleme}
\include{tl-dateiprobleme}
%% Netzwerkprobleme
\include{tl-klassifikation-netzwerk}
\include{tl-totalausfall}
\include{tl-teilweiser-ausfall}
%% tl-netz-performance.tex
\chapter{Netzwerkperformance}
\label{cha:netz-performance}

\begin{abstractsec}
  Manchmal scheint alles in Ordnung zu sein und trotzdem sind die Kunden nicht
  zufrieden. Performance ist ein heikles Thema, weil jeder seine eigene
  Vorstellung davon hat, was ausreichende Performance ist.
\end{abstractsec}

\begin{notes}
\item Baseline, ein Ausgangspunkt, der gemessen werden kann und als Referenz
  dient.
\item Bufferbloat
\item Netalyzr
\end{notes}

\section{Netzwerkbandbreite}
\label{sec:netz-performance-bandbreite}
\begin{abstractsec}
  Die Bandbreite einzelner Netzsegmente bestimme ich normalerweise nicht erst
  im Fehlerfall, sondern vorher, am Besten gleich nach Inbetriebnahme eines
  Abschnitts. Dann habe ich bei Problemen einen Referenzwert, der mir bei der
  Eingrenzung des Problems nützlich sein kann.
\end{abstractsec}
\begin{normaltext}
  Die Bandbreite einzelner Netzsegmente bestimme ich normalerweise nicht erst
  im Fehlerfall, sondern vorher, am Besten gleich nach Inbetriebnahme eines
  Abschnitts. Dann habe ich bei Problemen einen Referenzwert, der mir bei der
  Eingrenzung des Problems nützlich sein kann.

  \subsection{Bestimmung der Bandbreite mit ping}
  Wie man mit Ping die Bandbreite bestimmen kann, ist sehr gut in
  \cite{sloan2001} Kapitel 4. Path Characteristics beschrieben.

  Das geht für beliebige Streckenabschnitte einer Verbindung und zwar wie
  folgt:
  \begin{enumerate}
    \item Mit Ping bestimmt man die RTT zum vorderen und hinteren Ende des
      Netzsegments. Die Differenzen zwischen den beiden Zeiten eliminieren die
      Einflüsse der anderen Netzkomponenten.
    \item Jetzt bestimmt man die RTT zu den beiden Enden mit größeren
      Datenpaketen und bestimmt wieder die Differenz.
    \item Die Differenz zwischen den beiden Zeitdifferenzen aus den ersten
      beiden Schritten ist die Zeit, die für die zusätzlichen Daten benötigt
      wird.
    \item Die Bitrate ist 16 mal der Differenz der Paketgrößen geteilt
      durch die Differenzzeit aus Schritt 3 (die magische Zahl 16 kommt daher,
      dass wir mit 8 Bit pro Byte rechnen und die die Messungen die doppelte
      Zeit, nämlich für Hin- und Rückweg, enthalten, aber nur die einfache
      Bitrate bestimmen wollen)
  \end{enumerate}
  Wenn ich mehrere Pingpakete pro Einzelmessung sende, dann verwende ich
  jeweils die geringste gemessene RTT, da diese vermutlich die geringsten
  Störeinflüsse enthält.
\end{normaltext}

\section{Lasttests}
\label{sec:netz-performance-lasttests}
\begin{abstractsec}
  Die reine Kenntnis von Bandbreite und Latenz von Netzwerken reicht nicht
  aus, um das Verhalten unter realen Bedingungen zu Beschreiben. Insbesondere
  durch Pufferung von Datenpaketen, die nicht gleich gesendet werden können,
  ändert sich die Latenz einer Übertragungsstrecke erheblich. Eine
  Möglichkeit, derartige Probleme zu diagnostizieren sind Lasttests.
\end{abstractsec}
\begin{normaltext}
  Zusätzlich zu den Latenzzeiten, die durch die Datenübertragung und das
  reine Umkopieren in den Routern und Switches entstehen, gibt es
  Verzögerungen, die durch die Pufferung von Datenpaketen verursacht werden.
  Diese entstehen dadurch, dass ein Netzwerkgerät auf der Sendeseite die Daten
  nicht so schnell los wird, wie sie auf der Empfangsseite ankommen. Das
  passiert meist beim Übergang von schnellen auf langsamere Medien, aber auch,
  wenn Datenpakete aus mehreren Richtungen ankommen und in die selbe Richtung
  abgehen. Tritt dieser Effekt nur kurzzeitig auf, dann wirken die Puffer
  positiv, da keines der Datenpaket verloren geht, sondern nur etwas später
  ankommt. Kommen jedoch ständig mehr Daten an, als gesendet werden können,
  dann werden erst die Puffer gefüllt, bevor Daten verworfen werden. Ohne
  aktives Puffermangement hat dann jedes ankommende Datenpaket so viele andere
  Datenpakete vor sich, wie in den Puffer passen. Die durch den Puffer
  verursachte Latenz beträgt dann Puffergröße geteilt durch Sendebandbreite.
  Leider verhindert die durch den Puffer erhöhte Latenzzeit, dass sich das
  TCP-Protokoll an die gerade mögliche Bandbreite anpassen kann. Die
  wirksamste  Abhilfe ist Adaptive Queue Management, ein Notbehelf kann
  Trafficshaping sein, wobei dieses den Nachteil hat, dass man als
  Netzadministrator die Sendebandbreite kennen muss um den Traffic manuell auf
  die entsprechende Rate zu begrenzen.

  Die reine Kenntnis von Bandbreite und Latenz von Netzwerken reicht also nicht
  aus, um das Verhalten unter realen Bedingungen zu Beschreiben. Insbesondere
  durch Pufferung von Datenpaketen, die nicht gleich gesendet werden können,
  ändert sich die Latenz einer Übertragungsstrecke erheblich. Eine
  Möglichkeit, derartige Probleme zu diagnostizieren sind Lasttests.

  Das heisst, ich erzeuge künstlichen einen starken Datenverkehr und
  beobachte, wie sich die anderen Netzparameter verhalten. Das betrifft zum
  einen die Latenz und zum anderen die verfügbare Bandbreite.
  \subsection{Lasttest mit Ping}
  Mit dem Befehl
  \begin{verbatim}
# ping -f rechnername
  \end{verbatim}
  sendet Ping Datenpakete so schnell es geht zum Zielrechner. Dazu benötige
  ich Superuserrechte.

  Diesen Aufruf verwende ich, um auf einem Segment Netzwerklast zu erzeugen.
  Zusätzlich zur normalen Statistik (min/avg/max/mdev) zeigt Ping am Ende zwei
  Werte: IPG und EWMA.

  IPG (Inter Packet Gap) ist die Zeit zwischen dem Senden zweier Datenpakete.
  Für Ethernet ist die Minimalzeit auf die Zeit festgelegt, in der 96 Bit
  übertragen werden. Das sind 9,6 µs für 10 MBit/s Ethernet und 9,6 ns für 10
  GBit/s. Diese Zeit wird automatisch vom Ethernetadapter an jedes Datenpaket
  angehängt. Das angezeigte IPG kann ich als Maß verwenden, um abzuschätzen,
  wie effizient die Kombination Betriebssystem, Netzwerkkarte, Netzwerk Daten
  senden kann.

  EWMA steht für Exponential Weighted Moving Average. Bei diesem Durchschnitt
  werden die letzten  RTT-Zeiten höher gewichtet als ältere. Im Normalfall
  sollte dieser gleich dem Mittelwert für RTT sein. Weicht er signifikant ab,
  deutet das auf einen Trend hin. Dafür benötigt man aber einen länger
  laufenden Ping und da EWMA am Ende ausgegeben wird, wird der Trend erst am
  Ende dieser Messung offenbar.
\end{normaltext}


%\begin{abstractsec}
%  Die Aufteilung der Umsetzung wird hier gegliedert in verschiedene Aspekte
%\end{abstractsec}
%\begin{normaltext}
%  Blablabla
%\end{normaltext}

%\section{Schnittstellen nach außen}
%\label{sec:schn-nach-au3en}

%\subsection{Schnittstelle zu A}
%\label{sec:schnittstelle-zu}

%\begin{notes}
%\item Syntax
%\item Semantik
%\end{notes}

%\subsection{Schnittstelle zu B}
%\label{sec:schnittstelle-zu-b}

%\begin{notes}
%\item Syntax
%\item Semantik
%\end{notes}


%%% Local Variables: 
%%% mode: latex
%%% TeX-master: "arbeit-hauptdatei"
%%% End: 
%%% vim: set sw=2 ts=2 tw=78 et si:

\include{tl-netz-hilfsmittel}
%% Vorbereitung auf das nächste Problem
\include{tl-lernen}
\include{tl-syslog}
\include{tl-monitoring}

\bibliography{literatur}

\end{document}

%%% Local Variables: 
%%% mode: latex
%%% TeX-master: t
%%% End: 
