%% Beispielgerüst für große Dokumente, für den Artikel aus DTK 2/2007
%% basierend auf:
%% Vorlage einer längeren wissenschaftlichen Arbeit für das Buch
%% "Wissenschaftliche Arbeiten schreiben mit LaTeX"
%% von Joachim Schlosser, 2006
%%
%% Sie dürfen dieses Dokument frei verteilen, verändern und verwenden.
%%
%% Dies ist nur eine Vorlage. 
%% Zwei Kommentarzeichen (%%) bedeuten, dass eine Erläuterung folgt.
%% Ein Kommentarzeichen heisst, dass die Zeile deaktiviert wurde
%% Die meisten Pakete sind deaktiviert, Sie aktivieren Sie, indem Sie
%% das Kommentarzeichen am Beginn der Zeile entfernen.
%% Aktivieren Sie nur, was Sie wirklich brauchen!
%% Lesen Sie in l2tabu nach, ob nicht einige Pakete mittlerweile
%% veraltet sind, wenn Sie diese Vorlage erst 2008 oder später verwenden!

\RequirePackage{fix-cm}        %% Korrektur der Computer Modern Schriftfamilien
\documentclass[%
%draft,                        %% um später den Zeilenumbruch zu kontrollieren
%a5paper,                       %% DIN A4-Papier
a4paper,                       %% DIN A4-Papier
BCOR4mm,                       %% Bindekorrektur 4 mm
DIVcalc,                       %% Satzspiegel berechnen
11pt,                          %% Schriftgröße
%bibtotoc,                      %% Literaturverzeichnis ins Inhaltsverzeichnis
%idxtotoc,                      %% Index ins Inhaltsverzeichnis
tablecaptionabove,             %% Tabellenüber"= statt unterschriften
%]{scrbook}                    %% KOMA-Skript Buch für das Endprodukt
]{scrreprt}                    %% KOMA-Skript Report als Klasse
\usepackage{geometry}          %% damit auch das Papierformat passt
\usepackage[english,german,ngerman]{babel}  %% Deutsch (neue Rechtschreibung) als Hauptsprache
\input{dehyphtex.tex}          %% zusätzliche deutsche Trennmuster
%\usepackage{microtype}         %% Optischer Randausgleich mit pdfLaTeX
\usepackage[T1]{fontenc}       %% Schriftkodierung, die nativ Umlaute unterstützt
\usepackage[utf8]{inputenc}    %% Umlaute direkt eingeben
% \usepackage[osf]{mathpazo}    %% Palatino als Brotschrift, mit Minuskelziffern
% \linespread{1.08}             %% Palatino braucht größeren Zeilenabstand
% \usepackage[scaled=.87]{luximono} %% anderer Monotype-Font
\usepackage{textcomp}          %% besondere Textzeichen
\usepackage{amssymb}           %% Mathesymbole
\usepackage{amsmath}           %% erweiterte Mathemakros
%\usepackage{paralist}          %% Aufzählungen im Fließtext
\usepackage{mdwlist}           %% engere Aufzählungen und Nummerierungen
%\usepackage{filecontents}      %% eingebundene Dateien (für ltxtable)
%\usepackage{ltxtable}          %% lange Tabellen mit flexibler Spaltenbreite
%\usepackage{tabularx}          %% Tabellen mit flexibler Spaltenbreite 
\usepackage{booktabs}          %% schönere Tabellenlinien
\usepackage{graphicx}          %% Graphiken einbinden
\providecommand\pstpdfinactive{inactive}%inactive
\usepackage[\pstpdfinactive]{pst-pdf}
\usepackage{tikz}              %% TikZ/pgf zum Schreiben von Grafiken
% \usepackage{ifpdf}            %% direktes Einbinden von Metapost-Graphiken
% \ifpdf 
% \DeclareGraphicsRule{*}{mps}{*}{} 
% \fi

\usepackage{versions}          %% bedingte Kompilierung
%% Literaturverzeichnisse: Hier ist Babelbib gewählt. Wählen Sie, was
%% Sie brauchen.
\usepackage{babelbib}          %% Mehrsprachiges Literaturverzeichnis
\usepackage[german,vario]{fancyref} %% "plain" für einfachere Querverweise, ",vario" für Seitenzahl
%% Anpassung für Querverweiserkennung: Kapitel durch "cha:" anstatt "chap:"
\fancyrefchangeprefix{\fancyrefchaplabelprefix}{cha}
%\fancyrefaddcaptions{german}{%
\newcommand*{\Freflstname}{Listing}%
\newcommand*{\freflstname}{\Freflstname}%
%}
\newcommand*{\fancyreflstlabelprefix}{lst}
\frefformat{plain}{\fancyreflstlabelprefix}{\freflstname\fancyrefdefaultspacing#1}
\frefformat{vario}{\fancyreflstlabelprefix}{\freflstname\fancyrefdefaultspacing#1#3}

%\usepackage{index}             %% Index vorbereiten
%\makeindex                     %% make index

\usepackage[%
  bookmarks,                   %% PDF-Lesezeichen
  bookmarksopen=true,          %% Lesezeichenbaum aufgeklappt...
  bookmarksopenlevel=1,        %% ...um eine Ebene
  bookmarksnumbered=true,      %% Lesezeichen numerieren
%  pagebackref,                 %% Links von Literatur in Text
  pdfusetitle,                 %% LaTeX-Titelei als Metainfo nehmen
  pdfstartpage={1},            %% mit welcher Seite das PDF öffnen
  pdfstartview={FitH},         %% Zoom auf Seitenbreite
  pdfkeywords={},              %% Stichwörter fürs PDF, kommagetrennt
  pdfsubject={},               %% Themenbeschreibung kurz
  pdfcreator={LaTeX with KOMA-Script and hyperref package},
%   pdfpagelabels,               %% Seitenzahlen wie im Dokument...
%   plainpages=false,            %% ...und nicht wie sonst
  hyperfootnotes=true,         %% Links auf Fußnoten
  hyperindex=true,             %% Indexeinträge verweisen auf Text
  linkbordercolor={0 1 1},     %% Rahmenfarbe interne Links
  menubordercolor={0 1 1},     %% Rahmenfarbe Literaturlinks
  urlbordercolor={1 0 0}       %% Rahmenfarbe externe Links
]{hyperref}                    %% Hyperlinks und Lesezeichen in PDF 
%\usepackage[figure]{hypcap}    %% Springt bei Links auf Abb. an die richtige Stelle

\pagestyle{headings}           %% lebende Kolumnentitel

%% Makro für Randnotizen
\newcommand{\notiz}[1]{\marginpar{\raggedright\footnotesize #1}}


%% Alles für die bedingte Kompilierung via Makefile
\newcommand{\enableversions}[4]{
  \ifthenelse{\equal{#1}{on}}{
    \newenvironment{abstractsec}{\begin{quotation}\noindent\textbf{Zusammenfassung:}}{\end{quotation}}
  }{\excludeversion{abstractsec}}
  \ifthenelse{\equal{#2}{on}}{\includeversion{normaltext}
  }{\excludeversion{normaltext}}
  \ifthenelse{\equal{#3}{on}}{
    \newenvironment{notes}{\par\hrulefill{}\emph{Ungeordnete partielle
        Ideensammlung (nicht im Druck):}\hrulefill{}\begin{itemize*}}{\end{itemize*}}
  }{\excludeversion{notes}}
  \ifthenelse{\equal{#4}{on}}{\def\svnfinal{final}\includeversion{finalversion}
  }{\def\svnfinal{}\excludeversion{finalversion}}
}

%% include-Makro umdefinieren, um auch mit diff zu harmonieren
\providecommand{\includesuffix}{}
\let\includeorig=\include
\renewcommand{\include}[1]{\includeorig{#1\includesuffix}}

\newcommand{\versionabstracts}{\enableversions{on}{off}{off}{off}
  \renewcommand{\include}[1]{\input{##1}}
}
\newcommand{\versionnotes}{\enableversions{off}{off}{on}{off}}
\newcommand{\versionabstractstext}{\enableversions{on}{on}{off}{off}}
\newcommand{\versiontext}{\enableversions{off}{on}{off}{off}}
\newcommand{\versiontextnotes}{\enableversions{off}{on}{on}{off}}
\newcommand{\versionfinal}{\enableversions{off}{on}{off}{on}
  \renewcommand{\include}[1]{\input{##1}}
  \renewcommand{\notiz}[1]{}
}

%% Wählen, was erzeugt werden soll:
\providecommand{\selectversion}{
%\versionabstracts
%\versionnotes
%\versionabstractstext
%\versiontext
%\versiontextnotes
\versionfinal
}

\AtBeginDocument{%
  \selectlanguage{ngerman}%
  \renewcommand*{\freffigname}{\figurename}
  \renewcommand*{\freftabname}{\tablename}
  \selectversion{}
}

\providecommand{\svnfinal}{final}
\usepackage[eso-foot,notoday,\svnfinal]{svninfo}

\hyphenation{%
%  eventuelle Trennmusterdefinitionen, mit - als Trennstellenmarker  
}

%\setcounter{secnumdepth}{3}    %% Nummerierung bis zur Ebene subsubsection
%\setcounter{tocdepth}{3}       %% Inhaltsverzeichnis bis zur Ebene subsubsection 
%% Anpassungen für das Layout: Float-Pages
%\renewcommand{\floatpagefraction}{.7}  % vorher: .5

%% Gegen die meisten Overfull hboxes, aus
%% dctt, Axel Reichert, Message-ID: <a84us0$plqcm$7@ID-30533.news.dfncis.de>
\tolerance 1414
\hbadness 1414
\emergencystretch 1.5em
\hfuzz 0.3pt
\widowpenalty=10000
\vfuzz \hfuzz
\raggedbottom

% Definition Textkasten / Exkursbox
\RequirePackage{color}
\RequirePackage{hhline}
\RequirePackage{multirow}
\RequirePackage{colortbl}
\RequirePackage{eso-pic,calc}           %% Zum Zeichnen des Rahmens um die Seite
\RequirePackage[bordercolor=SteelBlue]{todonotes} %% Zum Malen der farbigen Boxen

\newlength{\iconwidth}
\setlength{\iconwidth}{1cm}

\definecolor{boxheadcol}{gray}{.6}
\definecolor{boxcol}{gray}{.9}

\newenvironment{displaybox}[2]{%
  \begin{center}
    \setlength\arrayrulewidth{0.75pt}%
    \arrayrulecolor{white}%
    \renewcommand{\arraystretch}{1.3}%
    \begin{tabular} {p{\iconwidth}p{\linewidth-4\tabcolsep-\iconwidth}}
      \multirow{2}{*}{#2}&\cellcolor{boxheadcol}\textbf{\sffamily\color{white}#1} \\%
      \hhline{~-}%
      &\cellcolor{boxcol}%
}{%
    \end{tabular}
  \end{center}
}

\newenvironment{Exkursbox}[1]{%
\begin{displaybox}{#1}{}\sffamily\small}%
{\end{displaybox}}

\newenvironment{Beispiel}[1]{%
\begin{displaybox}{#1}{}\sffamily\small}%
{\end{displaybox}}

%% Hier beginnt das eigentliche Dokument
\begin{document}

%% Die Eckdaten des Dokuments, wegen Babel erst nach \begin{document}
\author{Mathias Weidner}
\title{Fehlersuche bei Linux-Server und -Netzwerkproblemen}
%\subject{Masterarbeit}
%\publishers{Technische Universität Musterstadt\\
%  Fakultät für angewandte Wissenschaft}
%\date{24.\,12.~2006}

%% Titel daraus erzeugen
\maketitle

%% Zusammenfassung. Oft auch erst nach dem Inhaltsverzeichnis.
%\include{zusammenfassung}

%% Inhaltsverzeichnis
\tableofcontents 

%% Abbildungsverzeichnis
%\listoffigures

%% Tabellenverzeichnis
%\listoftables

%% tl-vorwort.tex
%% vim: set sw=2 ts=2 tw=78 et si:
\chapter{Vorwort}
\label{cha:vorwort}

\begin{abstractsec}
  Hier wird die Einführung stehen.
\end{abstractsec}
\begin{normaltext}
  Ich habe bereits viele Jahre mit Fehlersuche verbracht.
  Sei es als Kind beim Basteln, wenn etwas nicht so funktionierte, wie ich
  dachte. In meinem ersten Beruf bei der Reparatur von Telefonen und
  Telefonanlagen, die damals noch mit Relais funktionierten, so dass man ihnen
  im wahrsten Sinne des Wortes bei der Arbeit zusehen konnte. Nach dem
  Studium, als Softwareentwickler, beim Debuggen meiner oder fremder
  Programme. Oder später dann als Systemadministrator für UNIX und Netzwerke.

  Dabe war ich vor allem auf einige Dinge angewiesen um zum Erfolg zu kommen.
  Das erste und wichtigste ist ein zumindest grundlegendes Verständnis der
  Materie und die Fähigkeit und Möglichkeit, weitere nötige Kenntnisse zu
  erwerben.
  
  Das zweite ist eine einigermaßen strukturierte Vorgehensweise, die
  mich zumindest in den meisten Fällen schneller zum Ziel führt als andere
  Vorgehensweisen.

  \begin{Exkursbox}{Wissenschaftliches versus strukturiertes Vorgehen}
  Ich unterscheide hier zwischen dem wissenschaftlichen Vorgehen, einem
  methodischen Untersuchen aller Möglichkeiten zwecks Erkenntnisgewinn und dem
  zielgerichteten strukturierten Vorgehen, bei dem ich zunächst die
  Möglichkeiten untersuche, die vermutlich am ehesten zum Erfolg führen zwecks
  schnellstmöglicher Problembeseitigung.
  \end{Exkursbox}

  Eine dritte Voraussetzung ist die entsprechende geistige Einstellung. Ich
  musste sehr oft und werde vermutlich noch oft die Erfahrung machen, dass ich
  mir selbst bei der Lösung eines Problems im Weg stehe:

  \begin{itemize}
    \item Sei es, dass ich mir vor der Kenntnis aller relevanten Tatsachen
      bereits eine (dann eventuell falsche) Meinung gebildet hatte.
    \item Sei es, dass ich durch Druck von außen oder selbst erzeugten Druck
      besonders schnell sein wollte und dann beim oberflächlichen Hinsehen
      wichtige Details übersah.
    \item Sei es, dass ich angebotene Hilfe nicht annahm, weil ich es selbst
      schaffen wollte oder so sehr im Problem gefangen war, das ich das
      Hilfsangebot nicht wahrnahm.
    \item Sei es, dass ich kurz vor dem Ziel aufgab und einfach nicht den
      letzten Meter gegangen war.
  \end{itemize}

  In diesem Buch lege ich meine über die Jahre gesammelten Erfahrungen mit
  der Fehlersuche bei Linux-Servern und in IP-Netzwerken nieder.

  Ich habe diese beiden Themen kombiniert, weil zum Einen Linux-Server fast
  immer über IP-Netzwerke angesprochen werden und daher auf ein gut
  funktionierendes Netzwerk angewiesen sind und zum Anderen Linux-Rechner
  (da gehen auch Arbeitsstationen) für mich ideal zur Analyse von
  Netzwerkproblemen geeignet sind.

  Zwar benutze ich fast täglich Linux auf dem Desktop, allerdings sind hier
  die am häufigsten verwendeten Programme ein Terminal und ein Webbrowser, die
  in den meisten Fällen vorzüglich funktionieren, so dass ich kaum Gelegenheit
  hatte Erfahrungen mit der Fehlersuche am Linux-Desktop zu sammeln. Aus
  diesem Grund bleibt dieser Bereich aussen vor.
\end{normaltext}
\begin{notes}
\item Meine Motivation ist das mitunter erschreckende Unverständnis einiger
  meiner Kollegen beim einfachen Fehlersuchen. Ich extrapoliere, dass es auch
  anderen so gehen könnte.
\item Die Zielgruppe ist angehende (und gestandene?) Systemadministratoren für
  Linux und/oder IP-Netzwerke. Für diese muss ich die meisten Gedanken
  transparent machen und entsprechende Beispiele bringen.
\item Beim Text und Stil brauche ich mir keine persönlichen und
  institutionellen Beschränkungen auferlegen. Der Stil sollte dennoch
  weitgehend neutral sein, in den Beispielen ist die erste (und dritte)
  Person angemessen.
\item Ich werde grundlegende (strategische) Methoden, die ich bei der
  Fehlersuche verwende, vorstellen.
\item Ich werde einige (taktische) Programme vorstellen, die mir bei der
  Fehlersuche helfen.
\item Ich werde die operative Anwendung der grundlegenden Methoden und der
  Programme in einigen konkreten Fehlersituationen vorstellen.
\item Ich werde mich auf die Gebiete Linux-Server und (IPv4) Netzwerke
  beschränken. Grafische Oberflächen bleiben (zunächst) außen vor.
\item Danksagung
\end{notes}

\section*{Für wen ist dieses Buch}
\label{sec:fuerwen}

\begin{normaltext}
\end{normaltext}

\section*{Übersicht}
\label{sec:ubersicht}

\begin{normaltext}
  Dieses Buch ist in vier Teile gegliedert:
  \begin{itemize}
  \item Teil 1 beschäftigt sich mit dem grundlegenden Vorgehen bei der
    Fehleranalyse.
  \item In Teil 1 geht es vorwiegend um lokale Probleme an einem
    Linux-Rechner.
  \item Teil 3 befasst sich mit Netzwerkproblemen.
  \item In Teil 4 geht es um die Nachbereitung eines gelösten und die
    Vorbereitung auf das nächste Problems.
  \end{itemize}
\end{normaltext}

\section*{Danksagung}
\label{sec:problemstellung}

\begin{abstractsec}
  Wer alles geholfen hat.
\end{abstractsec}
\begin{normaltext}
  Diese haben alle geholfen.
\end{normaltext}

%%% Local Variables: 
%%% mode: latex
%%% TeX-master: "troubleshoot-linux"
%%% End: 


%\part{Grundlegende Herangehensweisen}
%%% tl-part1.tex
\part{Grundlegende Herangehensweisen}

%%% Local Variables: 
%%% mode: latex
%%% TeX-master: "troubleshoot-linux"
%%% End: 
%%% vim: set sw=2 ts=2 tw=78 et si:

%% tl-methoden.tex
\chapter{Methoden}
\label{cha:methoden}

\begin{abstractsec}
  Methoden in Form von klaren Handlungsanweisen erleichtern die Fehlersuche in
  dem Sie den Kopf von Routine entlasten und für das spezielle Problem der
  konkreten Fehlersuche entlasten.
\end{abstractsec}

\begin{notes}
\item Entscheidungsbaum
\end{notes}

%\section{Nicht verrückt machen lassen}
%\label{sec:nicht-verrueckt}

%\begin{abstractsec}
%  sonst kann man nicht klar denken.
%\end{abstractsec}
%\begin{normaltext}
%  Blablabla
%\end{normaltext}

%\subsection{Seiteneffekte}
%\label{sec:seiteneffekte}

%\section{Methoden}
%\label{sec:methoden}

%\begin{abstractsec}
%  Neben den in den folgenden Kapiteln vorgestellten und vorgeführten
%  Programmen sind gute Methoden ein wichtiger Bestandteil meines
%  Werkzeugkastens bei der Fehlersuche.
%\end{abstractsec}

%\begin{notes}
%\item Entscheidungsbaum
%\end{notes}

%\section{Heuristiken}
%\label{heuristiken}

%\begin{abstractsec}
%  Wenn gar nichts mehr geht, helfen vielleicht Heuristiken.
%\end{abstractsec}

%\begin{notes}
%\item zeitliche Korrelation
%\end{notes}

%%% Local Variables: 
%%% mode: latex
%%% TeX-master: "arbeit-hauptdatei"
%%% End: 
%%% vim: set sw=2 ts=2 tw=78 et si:

%% tl-heuristiken.tex
\chapter{Heuristiken}
\label{cha:heuristiken}

\begin{abstractsec}
  Heuristiken können mitunter einen Ausweg bieten, wenn alle bekannten
  Methoden nicht anwendbar sind oder versagt haben. Oder sie können helfen,
  langwierige strukturierte Methoden abzukürzen und zu beschleunigen.
\end{abstractsec}

\begin{normaltext}
  Heuristiken wende ich an, wenn alle meine bekannten Methoden nicht
  anwendbar sind oder versagt haben. Oder, wenn ich eine etwas langwierige
  strukturierte Methode abkürzen will um schneller zum Ziel zu kommen.

  \begin{Exkursbox}{Methoden und Heuristiken}
    % hier vielleicht noch die Definitionen aus dem Duden
    In diesem Buch spreche ich von Methode als klaren strukturierten
    Anweisungen, die im Normalfall immer zu einer Lösung gelangen.

    Bei Heuristiken denke ich auch an ein strukturiertes
    Vorgehen, dass aber den Erkenntnisgewinn zum Ziel hat und nicht
    notwendigerweise direkt zur Lösung des Problems führt.
  \end{Exkursbox}
\end{normaltext}

\begin{notes}
\item zeitliche Korrelation
\item Suchmaschine
\item Supportforen
\end{notes}

\section{Korrelation}
\label{sec:korrelation}

%\begin{abstractsec}
%  sonst kann man nicht klar denken.
%\end{abstractsec}

\begin{normaltext}
Mit einer Korrelation kann man eine Beziehung zwischen zwei oder mehreren
Merkmalen, Ereignissen oder Zuständen beschreiben. Wichtig ist dabei, zu
beachten, dass diese Beziehung kausal sein kann aber nicht muss. Im
mathematischen Sinne beschreibt die Korrelation einen statistischen
Zusammenhang im Gegensatz zur Proportionalität, die einen festes Verhältnis
beschreibt.
\end{normaltext}

\begin{notes}
\item Zeitliche Korrelation: Wann trat der Fehler erstmalig auf und was wurde
  zu dieser Zeit geändert?
\item geht auch nicht
\end{notes}

\section{Abkürzungen und Umwege}
\label{sec:abkuerzung-umweg}

\begin{abstractsec}
  Abkürzungen beschleunigen das Erreichen des Ziels. Umwege erhöhen die
  Ortskenntnis.
\end{abstractsec}

\begin{normaltext}

\subsection{Abkürzungen}
\label{sec:abkuerzungen}

Die Frage, wann ein Problem erstmals auftrat (oder bemerkt wurde) und was zu
dieser Zeit oder davor geändert wurde, kann eine Fehlersuche erheblich
abkürzen, wenn Sie einen Hinweis ergibt auf eine mögliche Ursache ergibt. Bei
dieser Fragestellung haben wir es mit einer Korrelation zu tun und dürfen
nicht vergessen, dass die Änderungen mit dem Problem zu tun haben können, aber
nicht notwendigerweise müssen.

\subsection{Umwege}
\label{sec:umwege}

\subsubsection{Probe und Gegenprobe}

Jeder Test, der erfolgreich ist, sollte durch eine geeignete Gegenprobe
evaluiert werden, die bestätigt, dass der Test auch versagen kann. Das gleiche
gilt für fehlgeschlagene Tests. Diese müssen evaluiert werden, ob sie
funktioniern können.

Wenn ein Verbindungsversuch auf einem Rechner fehlschlägt, versuche ich den
gleichen Versuch von einem anderen Rechner aus um nachzuweisen, dass er hätte
funktionieren können. Das hat mich schon häufig davor bewahrt, einen
Netzwerkfehler zu vermuten, wenn lediglich ein Paketfilter auf dem Zielrechner
die Verbindung unterbunden hatte.
\end{normaltext}

%\begin{notes}
%\item Entscheidungsbaum
%\end{notes}

%%% Local Variables: 
%%% mode: latex
%%% TeX-master: "arbeit-hauptdatei"
%%% End: 
%%% vim: set sw=2 ts=2 tw=78 et si:

%% tl-herangehen.tex
%% vim: set sw=2 ts=2 tw=78 et si:
\chapter{Herangehen}
\label{cha:herangehen}

\begin{abstractsec}
  In diesem Kapitel geht es weder um Methoden und Handlungsangweisungen für
  die Fehlersuche noch um gute Ratschläge, falls alle Methoden nicht zum Ziel
  führen. Hier geht es um Faktoren, die den Erfolg einer Fehlersuche eher
  inderekt, dafür aber umso stärker beeinflussen. Es geht um Einstellungen,
  Gewohnheiten, darum, wie mein Gehirn funktioniert und wie ich das
  gegebenenfalls ausnutzen kann um ein hartnäckiges Problem doch noch zu
  lösen.
\end{abstractsec}

\begin{normaltext}

\section{Prinzipien}
\label{sec:prinzipien}

\subsection*{Vier Kraftprinzipien}

\begin{quote}
Es ist gegen neun Uhr, gut gelaunt betrete ich die Büroräume, als mich der
Kollege am Operatingplatz mit leicht erhobener Stimme empfängt:

``Im Rathaus geht das INTERNET nicht!''

``Das ist aber schade.''

Eine dritte Kollegin ruft aus dem Hintergrund: ``Das habe ich gewusst, dass er
das jetzt sagt''.
\end{quote}

Was war denn da los? Proben die für eine Fernsehsendung?

Nun ich habe hier Prinzipien angewandt, die ich im Wing Tsun Kung Fu (WT)
gelernt habe und - entsprechend angepasst - bei Fehlersuchen aller Art gut
verwenden kann. Es gibt im WT vier Kraftprinzipien, die lauten:

\begin{enumerate}
  \item Mache Dich frei von Deiner eigenen Kraft.
  \item Mache Dich frei von der Kraft Deines Gegners.
  \item Nutze die Kraft Deines Gegners.
  \item Füge zur Kraft Deines Gegners Deine eigene Kraft hinzu.
\end{enumerate}

Das klingt alles sehr schön und esoterisch, aber was hat das mit Fehlersuche
in Computern und Computernetzen zu tun?

Die eigene Kraft, von der ich mich frei machen will, sind vorschnelle
Vermutungen, die ich aufstelle, die manchmal korrekt sind, aber mich oft nur
daran hindern, das eigentliche Problem wahrzunehmen.

Die Kraft meines Gegners, von der ich mich freimachen will, sind die
Vermutungen und voreiligen Diagnosen dessen, der mir das Problem mitteilt.
Diese können mich, genau wie meine eigenen, in die falsche Richtung locken.

Wie nutze ich nun die Kraft meines Gegners? Nachdem ich mich eigener
vorschneller Vermutungen und Diagnosen enthalten habe und die Aussagen meines
Gegenübers zwar zur Kenntnis genommen aber mir nicht unbedingt zu eigen
gemacht habe, evaluiere ich seine Angaben zum Beispiel anhand des
grundlegenden Entscheidungsbaumes um das wirkliche Ausmaß des Problems und die
nächsten Schritte zu bestimmen. Dabei frage ich gezielt und strukturiert nach.

Meine Kraft füge ich hinzu, indem ich aus den so gewonnenen Kenntnissen eine
Lösung des Problems herausarbeite.

Im eingangs erwähnten Gespräch wollte der Operator mir mit dem Tonfall und der
Schilderung wohl die besondere Dringlichkeit nahelegen und etwas Druck
aufbauen. Meine erste Reaktion war, diesen Druck in's Leere laufen zu lassen,
um den Kopf zum klaren Denken freizuhalten. Meine nächste Frage ging dann auch
danach, ob andere Websites aus dem Rathausnetz nicht erreichbar waren und ob
diese Websites aus unserem Netz erreichbar waren. So nutzte ich seine Kraft
und fügte meine - in diesem Fall das strukturierte Erfassen der Situation -
hinzu. Letztendlich stellte sich heraus, dass nur einige wenige Websites
betroffen waren und das - aus unserer Sicht - global, das heisst ausserhalb
undseres Einflussbereiches. Bis ich an meinem Platz ankam, waren alle Wogen
geglättet. Was blieb, sind ein paar Erinnerungen und ein Ticket im
Ticketsystem.
\end{normaltext}

\section{Zwei Bewusstseinsmodi}
\label{sec:zwei-bewusstseinsmodi}

\begin{abstractsec}
  Es gibt eine Unterscheidung des menschlichen Bewußstseins in einen
  sprachlich, analytischen Modus und einen visuellen, wahrnehmenden Modus.
  Obwohl die meisten Menschen beide Modi mehr oder weniger unbewusst
  verwenden, ist es gerade bei der Fehlersuche von Vorteil beide gezielt
  einsetzen zu können und gezielt zwischen beiden umschalten zu können.
\end{abstractsec}
\begin{normaltext}
  Es gibt eine Unterscheidung des menschlichen Bewußstseins in einen
  sprachlich, analytischen Modus und einen visuellen, wahrnehmenden Modus.
  Obwohl die meisten Menschen beide Modi mehr oder weniger unbewusst
  verwenden, ist es gerade bei der Fehlersuche von Vorteil beide gezielt
  einsetzen zu können und gezielt zwischen beiden umschalten zu können.

  Vor meinem Studium habe ich einen Beruf erlernt, in dem ich unter anderem
  Telefonvermittlungsanlagen reparieren können musste, die damals noch mit
  Relais funktionierten. Damals hatten wir zwei Arten dieser Anlagen. Die
  alten, mit Hebdrehwählern hatten nur relativ wenige Relais und wenn man in
  der Vermittlunsanlage beobachtete, was beim Abheben des Telefonhörers
  passierte, dann konnte man beinahe das nächste Relais ansagen, was betätigt
  wurde. Das heißt, diese Anlagen konnte man im sprachlich analytischen Modus
  analysieren.

  Die anderen, moderneren Anlagen, funktionierten mit Koordinatenschaltern.
  Das waren große elektromagnetische Schalter, die in einer Art Matrix die
  richtige Verbindung durchschalteten. Diese konnten eine Verbindung erheblich
  schneller durchschalten, als Hebdrehwähler und konnten auch die Tonsignale
  von Tastentelefonen ohne Umsetzungen in Zähltakte verarbeiten. Dafür
  benötigten sie erheblich mehr Relais zur Funktion und Fehler konnten dort
  nicht wie in den alten Anlagen analysiert werden, da fast immer mehrere
  Relais gleichzeitig betätigt wurden. Dafür lernten wir eine andere
  Herangehensweise. Zunächst setzten wir uns so, dass wir die ganze Anlage im
  Blick hatten. Dann beobachteten wir wiederholt das Gesamtbild der Anlage,
  wenn bestimmte Aktionen ausgeführt wurden. Anschließend verglichen wir die
  Relais, die sich bewegt hatten mit dem Schaltplan der Anlage und
  erarbeiteten uns analytisch Stück für Stück den Zusammenhang zwischen dem,
  was wir gesehen hatten und dem, was laut Schaltplan hätte passieren müssen.
  Am Anfang wiederholten wir das mit einer funktionsfähigen Anlage. Nach
  einiger Zeit, als wir ein Gefühl für die Funktionsweise der Anlage bekommen
  hatten, isolierte unser Lehrmeister einige Relaiskontakte mit farblosem
  schnelltrocknendem Kleber und unsere Aufgabe war es, herauszufinden, welche
  Kontakte schlecht waren. Später, als wir dadurch ein Gefühl dafür bekamen,
  welcher Fehler wie aussah, erhielten die Fortgeschrittenen die Aufgabe sich
  selbst Fehler auszudenken, die die anderen finden sollten. Natürlich setzten
  wir unseren Ehrgeiz dahin, Fehler einzubauen, die durch ihr Verhalten die
  anderen zunächst auf die falsche Spur schickten. Und natürlich kannten wir
  diese Anlagen am Ende in- und auswendig. Was wir dabei - spielerisch -
  übten, war ein Umschalten zwischen dem visuellen, wahrnehmenden Modus und
  dem sprachlich, analytischen Modus unseres Gehirns. Und, ja, heute könnte
  ich keine solcher Anlagen mehr reparieren, müsste alles erst wieder lernen.
  Was ich aber immer noch kann, ist relativ schnell zwischen dem Gesamtbild
  eines Problems und den Details zu wechseln.

  Dieser Exkurs in eine längst vergangene Zeit war nötig, um einen Aspekt
  hervorzuheben, der damals so wichtig ist, wie heute: das die unzähligen
  Details, die man bei der Fehlersuche wissen muss, für sich genommen nichts
  wert sind, wenn es nicht gelingt, diese in ein Gesamtbild einzufügen und
  dadurch richtig zu bewerten. Wenn ich in der Lage bin, das Gesamtbild zu
  sehen, kann ich für jedes Detail einschätzen, ob es korrekt, falsch oder
  irrelevant ist und mich dementsprechend dem nächsten zuwenden.
  Dazu ist es notwendig, beide Gehirnhälften oder Bewusstseinsmodi gezielt
  einzusetzen. In \cite{edwards:zeichnenlernen} beschreibt die Autorin am
  Beispiel des Zeichnenlernens, einige Übungen zum gezielten Umschalten der
  beiden Modi, die sie L-Modus und R-Modus nennt.
\end{normaltext}

\begin{notes}
\item Bisektion zur Beschleunigung: nicht den kompletten Entscheidungsbaum
  abarbeiten.
\end{notes}

%\subsection{Seiteneffekte}
%\label{sec:seiteneffekte}

%\section{Methoden}
%\label{sec:methoden}

%\begin{abstractsec}
%  Neben den in den folgenden Kapiteln vorgestellten und vorgeführten
%  Programmen sind gute Methoden ein wichtiger Bestandteil meines
%  Werkzeugkastens bei der Fehlersuche.
%\end{abstractsec}

%\begin{notes}
%\item Entscheidungsbaum
%\end{notes}

%\section{Heuristiken}
%\label{heuristiken}

%\begin{abstractsec}
%  Wenn gar nichts mehr geht, helfen vielleicht Heuristiken.
%\end{abstractsec}

%\begin{notes}
%\item zeitliche Korrelation
%\end{notes}

%%% Local Variables: 
%%% mode: latex
%%% TeX-master: "arbeit-hauptdatei"
%%% End: 


%\part{Lokale Probleme}
%%% tl-part2.tex
\part{Lokale Probleme}

%%% Local Variables: 
%%% mode: latex
%%% TeX-master: "troubleshoot-linux"
%%% End: 
%%% vim: set sw=2 ts=2 tw=78 et si:

%% tl-lokal-werkzeuge.tex
\chapter{Werkzeuge zur lokalen Fehlersuche}
\label{cha:lokal-werkzeuge}

\begin{abstractsec}
  Verschiedene Werkzeuge helfen mir lokale Probleme einzugrenzen. Hier stelle
  ich die Werkzeuge kurz vor, die ich in den nächsten drei Kapiteln zur
  Fehlersuche einsetze.
\end{abstractsec}

\begin{normaltext}
  Linux stellt mir eine Unmenge von Werkzeugen für die Fehlersuche zur
  Verfügung. Etliche davon kommen mir sowohl bei Total- oder Partialausfällen
  zu gute. Andere bei Performanceproblemen. Einige sind so nützlich, dass sie
  immer wieder bei den unterschiedlichsten Problemen zum Einsatz kommen. Da
  es mir schwerfällt, die einzelnen Werkzeuge bestimmten Kategorien
  zuzuordnen, stelle ich diese nachfolgend in alphabetischer Reihenfolge vor.
  Das erleichtert zumindest das Wiederfinden, wenn man mal eben etwas schnell
  nachschlagen möchte.
\end{normaltext}

\begin{notes}
\item Wie installiere ich Software?
\end{notes}

\section{shell}
\label{sec:lokal-werkzeuge-shell}
\begin{abstractsec}
  Ein wichtiges Hilfsmittel bei der Fehlersuche ist die Shell. Dies in
  zweierlei Hinsicht: zum einen starte ich in einer interaktiven Shell die
  Kommandos, mit denen ich den Fehler eingrenzen will, zum anderen verwende
  ich die Shell für simple, schnell zusammengestrickte Programme, die mich bei
  der Fehlersuche unterstützen.
\end{abstractsec}
\begin{normaltext}
  Ein wichtiges Hilfsmittel bei der Fehlersuche ist die Shell. Dies in
  zweierlei Hinsicht: zum einen starte ich in einer interaktiven Shell die
  Kommandos, mit denen ich den Fehler eingrenzen will, zum anderen verwende
  ich die Shell für simple, schnell zusammengestrickte Programme, die mich bei
  der Fehlersuche unterstützen.

  Bei der Problemeingrenzung bevorzuge ich eine Shell mit History-Funktion und
  Kommandozeilenergänzung (commandline completion).
  Die History-Funktion benutze ich vor allem Dingen, um bereits ausgeführte
  Befehle wieder hervorzuholen, gegebenenfalls geringfügig zu ändern und noch
  einmal auszuführen. Die Kommandozeilenergänzung beschleunigt den Zusammenbau
  von neuen Befehlen, in denen die Shell (meist nach Eingabe von \verb?<TAB>?)
  die Zeile komplettiert oder - bei mehreren möglichen Vervollständigungen -
  die nächsten Argumente vorschlägt. Das halte ich für unverzichtbar um bei
  der Fehlersuche den Gedankenflug nicht abreißen zu lassen.
  Ich bin an die Bash gewähnt, aber andere Shells können das meist ebensogut.

  Für Shell-Scripts bevorzuge ich als kleinsten gemeinsamen Nenner die Bourne
  Shell (\verb?/bin/sh?). Auf manchen Systemen ist \verb?/bin/sh? nur ein Link
  auf die Bash. Das ist kein Problem, weil diese als Bourne Again Shell in der
  Lage ist, Bourne Shell Skripts auszuführen. Da andererseits nicht jede
  Bourne Shell kompatible Shell in der Lage ist, Bash-Erweiterungen zu
  verstehen, beschränke ich mich hier nach Möglichkeit auf den kleinsten
  gemeinsamen Nenner.

  Ich will hier keine komplette Einführung in die Programmierung mit der
  Bourne Shell geben, sondern verweise stattdessen auf die Handbuchseiten.
  Stattdessen stelle ich ein Skript vor, das ich in dieser oder ähnlicher Form
  bereits mehrfach für die Fehlersuche verwendet habe und gehe dann auf die
  Einzelheiten ein.

  \subsection{Strace Invocator}
  Manchmal habe ich ein Problem mit einem Programm, bei dem ich auf strace
  (siehe Abschnitt \ref{sec:lokal-werkzeuge-strace}) zurückgreife,
  um das Programm bei der Arbeit zu beobachten.
  Wenn dieses Programm jedoch nicht von Kommandozeile,
  sondern von einem anderen Programm gestartet wird, greife ich auf folgenden
  Trick zurück: Ich benenne das Programm um indem ich an den Namen die Endung
  \verb?.orig? anhänge. Unter dem ursprünglichen Programmnamen platziere ich
  einen Link auf dieses Skript:
  \begin{verbatim}
 1:#!/bin/sh
 2:
 3:progname=$(basename $0)
 4:origin=$0.orig
 5:
 6:die () {
 7:  ev=$1
 8:  shift
 9:  echo $* 1>&2
10:  exit $ev
11:}
12:
13:[ -x $origin ] || die 2 "Can't execute $origin"
14:
15:tmpdir=$(mktemp -d /tmp/$progname.XXXXX)
16:
17:date > $tmpdir/invocation
18:
19:echo "#----- args -----" >> $tmpdir/invocation
20:for arg in "$@"; do echo $arg >> $tmpdir/invocation; done
21:
22:echo "#----- id -----" >> $tmpdir/invocation
23:id >> $tmpdir/invocation
24:
25:echo "#----- env -----" >> $tmpdir/invocation
26:env >> $tmpdir/invocation
27:
28:strace -f -o $tmpdir/strace.out $origin "$@"
  \end{verbatim}
  Zeile 1 zeigt an, dass das Skript mit der Bourne Shell läuft.

  In Zeile 3 bestimme ich den Namen des aufgerufenen Programms und in Zeile 4
  den Namen des eigentlichen Programms.

  In Zeile 6 bis 11 habe ich eine kleine Funktion, mit der ich das Skript mit
  Fehlermeldung und -code beenden kann.

  In Zeile 13 teste ich, ob das Originalprogramm da und ausführbar ist und
  breche andernfalls ab. Dazu nutze ich die oben definierte Funktion.

  In Zeile 15 lege ich ein temporäres Verzeichnis mit eindeutigem Namen an.
  Damit kann ich mich bei der Auswertung in aller Ruhe auf jeden einzelnen
  Aufruf des Programms konzentrieren.

  In den Zeilen 17 bis 26 halte ich verschiedene Informationen zum Aufruf
  fest. Und zwar die aktuelle Zeit, ide übergebenen Argumente, die
  Benutzer-Id, unter der der Prozess läuft und die Umgebungsvariablen.

  In Zeile 28 schließlich rufe ich via strace das Originalprogramm mit allen
  Argumenten auf. Diesen Aufruf kann ich noch modifizieren, wenn ich an der
  Standardeingabe für das Programm interessiert bin:
  \begin{verbatim}
28:tee $tmpdir/stdin | strace -f -o $tmpdir/strace.out $origin "$@"
  \end{verbatim}
  Oder, wenn ich sowohl an der Standardeingabe als auch an der Standardausgabe
  interessiert bin:
  \begin{verbatim}
28:tee $tmpdir/stdin \
29:| strace -f -o $tmpdir/strace.out $origin "$@" \
30:| tee $tmpdir/stdout
  \end{verbatim}
\end{normaltext}

\section{perl}
\label{sec:lokal-werkzeuge-perl}
\begin{abstractsec}
  Für knifflige Probleme, die ich mit den spezialisierten Werkzeugen nicht zu
  fassen kriege und denen mit einfacher Shell-Programmierung auch nicht
  beizukommen ist, benötige ich eine Programmiersprache, die mächtiger als die
  Shell ist, mit der ich aber trotzdem mit wenig Aufwand ein passendes
  Programm schreiben kann. Für mich ist das Perl.
\end{abstractsec}
\begin{normaltext}
  Zwischen den vielen Spezialwerkzeugen für die Fehlersuche und der Shell als
  Kommandozentrale benötige ich hin und wieder ein Werkzeug, mit dem sich auch
  kniffligere Probleme angehen lassen, die so vielleicht vorher noch gar nicht
  untersucht worden sind. Etwas, das sich etwa so schnell wie die Shell
  programmieren läßt, aber ausdrucksstärker ist und auch sehr komplexe
  Probleme angehen kann. Für micht ist das Perl. Für andere vielleicht Python,
  das in vielem Perl ähnlich nicht, aber auf keinem Fall in der Syntax.

  Die Programmiersprache Perl umfasst Konzepte von einfachen Werkzeugen wie sed
  oder awk bis hin zu anspruchsvollen Programmiersprachen wie C oder Lisp. Es
  gibt umfangreiche Fachliteratur sowohl offline als auch online sowie
  Communities, an die man sich bei Problemen wenden kann.

  Was Perl aber heraushebt gegenüber vielen anderen Skriptsprachen ist CPAN,
  das Comprehensive Perl Archive Network, ein so umfangreiches Reservoir an
  Softwaremodulen für fast alle erdenklichen Zwecke, das es möglich macht, die
  meisten Skripts auf wenige Zeilen zu beschränken. Die besten und meist
  verwendeten Module schaffen es mit der Zeit in die Standarddistribution und
  stehen dann nach der Installation von Perl gleich zur Verfügung. In vielen
  Fällen muß Perl auch gar nicht installiert werden, weil es bereits
  Bestandteil des Systems ist.

  Ganz besonders hilfreich bei der Problemlösung mit Perl ist das Perl
  Kochbuch von Tom Christiansen und Nathan Torkington. In diesem sind Lösungen
  für viele Probleme in Rezeptform aufbereitet und vor allen Dingen erläutert.
  Die Codebeispiele aus dem Kochbuch sind Online verfügbar, den meisten Wert
  zieht man jedoch aus den Erläuterungen im Buch.

  \subsection{HTTP Injector}

  Vor einiger Zeit hatte ich ein Problem, bei dem 502-Fehler von
  einem Webservice abhängig waren von der Zeit für die
  Anfrage. Der Betreiber des Webservices stritt das ab und um das
  Problem zu verifizieren benötigte ich die Möglichkeit HTTP-Anfragen gezielt
  zu verzögern.
  
  Ich kam mit Hilfe des Kochbuches zu folgendem Programm:
  \begin{verbatim}
01:#!/usr/bin/perl
02:use Getopt::Long;
03:use IO::Socket;
04:use Time::HiRes qw(sleep);
05:
06:my %opt = ( delay => 0 );
07:
08:GetOptions( \%opt, 'delay=i');
09:
10:my $server = shift;
11:my $port   = shift || 80;
12:
13:my $socket = IO::Socket::INET->new(PeerAddr => $server,
14:                                   PeerPort => $port,
15:                                   Proto    => 'tcp',
16:                                   Type     => SOCK_STREAM);
17:
18:my @in = <>;
19:my $del = $opt{delay} / ( 1.0 + scalar @in );
20:foreach (@in) {
21:    s/[\r\n]+$//;
22:    sleep $del;
23:    print $socket $_, "\r\n";
24:}
25:sleep $del;
26:print $socket "\r\n";
27:
28:while (my $line = <$socket>) {
29:    print $line;
30:}
  \end{verbatim}
  In den Zeilen 2-4 lade ich die benötigten Module. \verb?Getopt::Long? ist
  für die Verarbeitung der Kommandozeilenoptionen und sichert ab, dass ich mit
  \verb?--delay? einen Integerwert angebe. \verb?IO::Socket? stellt die
  Socketfunktionalität bereit, so dass ich diesen Socket wie eine Datei
  verwenden kann. \verb?Time::HiRes? stellt mir eine verbesserte
  \verb?sleep()? Funktion bereit, die mit Gleitkommazahlen zurechtkommt.

  In Zeile 6 stelle ich die Option \verb?--delay? auf den Wert 0 ein, falls
  sie nicht explizit angegeben wird. In Zeile 8 werden die Optionen
  eingelesen.

  Zeile 10 und 11 entnehmen den Server und gegebenenfalls den Port der
  Kommandozeile und in Zeile 13 öffne ich mit diesen Angaben den Socket.

  In Zeile 18 lese ich die gesamte Eingabe in ein Array ein. Dies benötige
  ich, da ich die Anzahl der Zeilen wissen muss, denn ich verzögere das Senden
  zeilenweise um jeweils einen Bruchteil der Gesamtverzögerung. Die Zeilen
  20-25 schließlich bereiten die Zeilenenden auf und senden die modifizierten
  Zeilen verzögert über den Socket. Zeile 26 schickt die Leerzeile, nach der
  der Server antwortet.

  In Zeile 28-30 liest das Skript die Antwort des Servers vom Socket und
  schreibt sie zur Standardausgabe.

  Dieses Skript kann ich nun wie folgt aufrufen:
  \begin{verbatim}
time ./http-injector.pl --delay 5 localhost 80 < request > reply

real  0m5.072s
user  0m0.056s
sys   0m0.012s
  \end{verbatim}
  Dabei steht in der Datei request die HTTP-Anfrage, die ich an den Server
  sende.
  Nach fünf Sekunden ist die Anfrage beim Server, und die Antwort landet in
  der Datei reply.

  Damit konnte ich nachweisen, dass dieselbe Anfrage einen Fehler
  lieferte, wenn sie mehr als drei Sekunden zur Übertragung brauchte und
  fehlerfrei beantwortet wurde, wenn sie weniger als drei Sekunden brauchte.
\end{normaltext}

\section{fuser}
\label{sec:lokal-werkzeuge-fuser}
\begin{abstractsec}
  Das Programm fuser setze ich ein, wenn ich schnell Informationen darüber
  haben will, welche Prozesse bestimmte Dateien oder Netzwerksockets geöffnet
  haben.
\end{abstractsec}
\begin{normaltext}
  Das Programm fuser setzte ich ein, wenn ich schnell Informationen darüber
  haben will, welche Prozesse bestimmte Dateien oder Netzwerksockets geöffnet
  haben, um sie dann mit anderen Programmen näher zu untersuchen.
  Zwar kann ich die ermittelten Prozesse dann gleich von fuser beenden lassen,
  aber in diesem Buch geht es vor allem um die Fehleranalyse und dafür wäre
  das dann meist doch etwas zu grobschlächtig.

  Was mich vor allem interessiert, sind die Prozesse und die Art und Weise,
  wie diese die betreffenden Dateien verwenden.
  Diese bekomme ich von fuser in einer Tabelle angezeigt. In der ersten Spalte
  steht die Datei, dahinter die PID. Mit der Option -v kann ich diese Ausgabe
  erweitern, so dass fuser für jeden Prozess den Benutzer (USER), die PID, den
  Zugriff (ACCESS) und den Namen des Prozesses (COMMAND) anzeigt.

  Von diesen ist ACCESS für viele Analysen interessant. Diese Spalte kann die
  folgenden Merkmale haben:
  \begin{description}
    \item[c] CWD, das Arbeitsverzeichnis des Prozesses,
    \item[e] Executable, die Datei wird als Programm ausgeführt
    \item[f] File, die Datei ist als normale Datei geöffnet
    \item[F] File, die Datei ist zum Schreiben geöffnet
    \item[r] Root, die Datei ist Wurzelverzeichnis
    \item[m] MMAP, die Datei ist in den Speicherbereich des Prozesses
      eingeblendet (zum Beispiel als Bibliothek)
  \end{description}
  Mit diesen Merkmalen bekomme ich heraus, wie ein Prozess eine Datei
  verwendet. Allerdings kann ich das nur für Dateien sehen, die noch im
  Dateisystem verlinkt sind. Dateien, die zwar geöffnet, aber nicht mehr im
  Dateisystem verlinkt sind, kann ich damit nicht finden. Dazu benötige ich
  andere Programme, wie zum Beispiel lsof. Mit der Option \verb?--mount?
  (\verb?-m?) bekomme ich aber zumindest die PID dieser Prozesse, und kann
  diese dann mit lsof näher untersuchen.

  Dazu muss ich folgende Besonderheit der Ausgabe von fuser beachten. Das
  Programm schreibt nur die PIDs an die Standardausgabe, alles andere kommt
  über die Fehlerausgabe. Damit kann ich die Ausgabe von fuser sehr bequem in
  Scripts weiterverarbeiten:
  \begin{verbatim}
for p in $(fuser -m /); do
    lsof -p $p
done
  \end{verbatim}
  Will ich die Ausgabe allerdings in einem Pager betrachten, oder
  dokumentieren, so schreibe ich:
  \begin{verbatim}
fuser -m / 2>&1 | less
  \end{verbatim}

  Neben den Prozessen, die auf bestimmte Dateien zugreifen, bin ich manchmal
  an allen Prozessen interessiert, die irgendeine Datei in einem Dateisystem
  geöffnet haben. Dafür verwende ich die Option \verb?--mount? (\verb?-m?).
  Das ist auch der einzige Weg, wie ich mit fuser Prozesse finden kann, die
  bereits im Dateisystem gelöschte Dateien noch offen halten. Welche Prozesse
  das konkret sind, kann ich zwar damit noch nicht sagen, aber das kann ich
  zum Beispiel ermitteln, wenn ich die mit fuser ermittelten Prozesse mit lsof
  näher betrachte.

  Außer bei Dateien in Dateisystemen kann ich mit fuser auch die Prozesse
  ermitteln, die bestimmte Sockets geöffnet haben. Dazu wähle ich den
  entsprechenden Namensraum mit der Option \verb?--namespace SPACE?
  (\verb?-n SPACE?) aus. Fuser kennt die folgenden Namensräume:
  \begin{description}
    \item[file] ist der Standardnamensraum, der nicht extra angegeben werden
      muss.
    \item[tcp] steht für TCP-Sockets
    \item[udp] steht für UDP-Sockets
  \end{description}
  Sockets werden nach dem folgenden Schema angegeben:
  \begin{verbatim}
[local port][,[remote host][,[remote port]]][/namespace]
  \end{verbatim}
  Den Namensraum kann ich angeben, wenn die Angabe eindeutig ist und ich
  diesen nicht explizit mit \verb?--namespace? angeben will.
  Die Komma sind wichtig. So zeigt \verb?fuser ssh/tcp? alle Prozesse, die mit
  dem lokalen Port 22 arbein, währen \verb?fuser ,,ssh/tcp? alle Prozesse mit
  abgehender SSH-Verbindung an.

  Mit der Option \verb?-4? beziehungsweise \verb?-6? kann ich die Ausgabe auf
  die entsprechende Version des Internetprotokolls eingrenzen.

  Eine Option, die ich eher selten anwende ist \verb?--kill? (\verb?-k?), mit
  der fuser ein Signal (ohne weitere Angaben: \verb?SIGKILL?) an die
  ermittelten Prozesse sendet. Das Signal kann ich mit einem vorangestellten
  Bindestrich (\verb?-?) angeben, eine Liste der Signale bekomme ich mit
  \verb?--list-signals? (\verb?-l?). Zum Beispiel könnte ich mit
  \begin{verbatim}
fuser -k -HUP 22/tcp
  \end{verbatim}
  alle SSH-Anmeldungen an diesem Rechner beenden. War ich selbst via SSH
  angemeldet, dann habe ich mich damit selbst hinausgeworfen. Oder, falls ich
  ein Dateisystem aushängen will, beende ich mit
  \begin{verbatim}
fuser -k -m /media/cdrom
  \end{verbatim}
  alle Prozesse, die auf die eingehängte CD-ROM zugreifen. Falls aber unter
  /media/cdrom kein Dateisystem eingehängt war, werden alle Prozesse, die das
  nächsthöhere Dateisystem (meist /) verwenden, beendet. Das kommt einem
  Ausschalten des Rechners schon sehr nahe. Darum gibt es, quasi als
  Sicherheitsgurt für solche Fälle, die Option \verb?--ismountpoint?
  (\verb?-M?), mit der alle Aktionen nur dann ausgeführt werden, wenn der
  angegebene Dateiname ein Mountpoint ist. Außerdem kann ich mit der Option
  \verb?-w? das Senden des Signals auf Prozesse einschränken, die eine Datei
  zum Schreiben geöffnet haben. Das ist dann interessant, wenn ich das Datei
  von read-write auf read-only umhängen will.

\end{normaltext}

\section{lsof}
\label{sec:lokal-werkzeuge-lsof}
\begin{abstractsec}
  Ein Werkzeug, dass in keinem Werkzeugkasten für die lokale Fehlersuche fehlen
  sollte, ist lsof. Dieses Programm zeigt Informationen zu Dateien, die von
  Prozessen geöffnet sind, an.
\end{abstractsec}
\begin{normaltext}
  Ein Werkzeug, dass in meinem Werkzeugkasten für die lokale Fehlersuche nicht
  fehlen darf, ist lsof. Dieses Programm zeigt Informationen zu Dateien, die
  von laufenden Prozessen geöffnet sind, an.

  Ich habe dieses Programm erfolgreich beim Untersuchen von Mount-Problemen
  eingesetzt.
  Auch beim Aufspüren und Untersuchen von Sicherheitsproblemen leistet es
  wertvolle Dienste.

  Außer für Linux gibt es dieses Programm auch für andere UNIX-Derivate, bei
  denen einige Optionen eine andere Bedeutung haben. Aus diesem Grund und weil
  ich hier nicht alle Optionen erläutern werde, ist ein Blick in die
  entsprechende Handbuchseite unumgänglich.

  Offene Dateien, die lsof auflistet können
  \begin{itemize}
    \item reguläre Dateien,
    \item Verzeichnisse,
    \item block- oder zeichenorientierte Spezialdateien,
    \item Verweise auf ausführbaren Code,
    \item Bibliotheken,
    \item UNIX- oder Netzwerksockets sein.
  \end{itemize}

  In einigen Aspekten überschneidet sich die Funktionialität von lsof mit der
  von netstat, welches ich weiter hinten in 
  Sektion \ref{sec:lokal-werkzeuge-netstat}
  beschreibe. Meist entscheide ich je nach vorliegendem Problem, zu welchem
  der Programme ich greife.

  Es ist möglich, von lsof statt einer einmaligen Ausgabe, automatisch in
  bestimmten Abständen neue Schnappschüsse der angeforderten Informationen zu
  erhalten und die Ausgabe dazu so umzuformen, dass sie gut von einem Skript
  oder sonstigem Programm überwacht werden kann.

  Rufe ich lsof ohne Optionen und Argumente auf, bekomme ich eine Liste aller
  Dateien, die alle laufenden Prozesse im Moment geöffnet haben. Bin ich nur
  an wenigen Dateien interessiert, gebe ich diese als Argumente auf der
  Kommandozeile an. Bin ich nur an bestimmten Aspekten oder an Dateien, die
  ich zwar selbst nicht genau kenne, aber deren Eigenschaften, so
  spezifiziere ich das mit Optionen. Einige dieser Optionen will ich hier
  vorstellen.
  
  Selektiere ich mit einer Option eine definierte Menge von
  Dateien, dann werden nur die Dateien, die dieser Selektion genögen,
  angezeigt. Gebe ich mehrere Selektoren an, dann werden alle Dateien
  angezeigt, die irgendeinem dieser Selektoren entsprechen. Das heißt, die
  Menge der angezeigten Dateien entspricht der ODER-Verknüpfung der einzelnen
  Selektoren. Dazu gibt es folgende Ausnahme: wenn ich mit einer Option
  bestimmte Dateien deselektiere (zum Beispiel durch vorangestelltes \verb?^?
  bei Auswahllisten), dann werden diese Dateien auch nicht angezeigt, wenn sie
  nach einem anderen Selektor drann wären. Außerdem kann ich die Verknüpfung
  der Selektionskriterien mit der Option \verb?-a? von ODER auf UND umstellen.
  Gebe ich mehrmals die gleiche Option mit verschiedenen Selektoren an, so
  werden diese vor der ODER- beziehungsweise UND-Verknüpfung zu einem Selektor
  zusammengefasst.
  Wenn ich zum Beispiel nur an allen Internetsockets interessiert bin, die von
  Prozessen mit UID xyz geöffnet sind, dann schreibe ich:
  \begin{verbatim}
    lsof -a -i -u xyz
  \end{verbatim}
  
  Nun zu den Optionen.

  Mit Option \verb?-c name? selektiere ich Prozesse, deren Name mit \verb?name?
  beginnt. Fängt \verb?name? selbst mit \verb?^? an, dann werden genau diese
  Prozesse ignoriert. Beginnt und endet \verb?name? mit einem Schrägstrich
  (\verb?/?), dann wird er als regulärer Ausdruck interpretiert.

  Mit der Option \verb?+d s? bekomme ich alle geöffneten Dateien direkt im
  Verzeichnis \verb?s?. Demgegenüber liefert \verb?+D s? auch die Dateien und
  Verzeichnisse in den Unterverzeichnissen von \verb?s?, die von Prozessen
  geöffnet sind. Beide Optionen kann ich mit \verb?-x? kombinieren, damit lsof
  symbolischen Links folgen und Mountpoints überqueren soll, was es ansonsten
  nicht machen würde.

  Die Option \verb?-d s? erwartet eine Liste von Dateideskriptoren (diese
  stehen in der Ausgabe in Spalte FD), die ich einschließen, oder mit \verb?^?
  ausschließen kann. Möchte ich zwar das Arbeitsverzeichnis, aber nicht die
  Standardeingabe, -ausgabe und -fehlerausgabe von Prozessen wissen, dann
  drücke ich das so aus:
  \begin{verbatim}
    lsof -d cwd,^0,^1,^2
  \end{verbatim}

  Mit der Option \verb?-i [m]? bekomme ich Internetsockets und zwar speziell
  für TCP oder UDP angezeigt. Optional kann ich diese mit dem Muster \verb?m?
  genauer spezifizieren. Dazu gebe ich \verb?m? in der folgenden Form
  \verb?[46][protocol][@hostname|hostaddr][:service|port]? an. Hierbei steht
  \begin{description}
    \item[4] für die Beschränkung auf IPv4
    \item[6] für die Beschränkung auf IPv6
    \item[protocol] für den Protokollnamen TCP oder UDP
    \item[hostname] für einen Internet-Hostnamen oder alternativ
    \item[hostaddr] für eine numerische Adresse
    \item[service] für einen Servicenamen aus /etc/services oder alternativ
    \item[port] für die Portnummer
  \end{description}
  Demgegenüber kann ich mit \verb?-U? UNIX-Domain-Sockets auswählen.

  Die Option \verb?-n? unterdrückt die Umwandlung von Netzadressen in Namen,
  \verb?-P? die Umwandlung von Portnummern in Servicenamen und schließlich
  \verb?-l? die Umwandlung von UID in Benutzernamen. Diese Optionen verwende
  ich, wenn ich mehr Klarheit haben will, oder wenn diese Umwandlung
  ihrerseits die Ausführung von lsof verzögert, weil DNS- oder NIS-Anfragen
  für die Auflösung notwendig sind.

  Mit der Option \verb?-u s? lassen sich die Prozesse nach UID oder
  Benutzernamen auswählen, während ich mit \verb?-p s? die Prozesse direkt
  nach PID auswählen kann.

  Starte ich lsof mit Otion \verb?-r [t]?, so liefert es die Informationen
  wiederholt in dem mit \verb?t? spezifizierten Zeitabstand (ohne Angabe 15
  Sekunden). Diese Option kann ich mit \verb?-F f? kombinieren um die Ausgabe
  für die einfachere Verarbeitung in einem Skript zu modifizieren.
\end{normaltext}

\section{netstat}
\label{sec:lokal-werkzeuge-netstat}
\begin{abstractsec}
  In den meisten Fällen setze ich netstat bei Netzwerkproblemen ein.
  Wenn ich jedoch Probleme mit UNIX-Sockets vermute oder den Prozess, der
  einen bestimmten Socket verwendet ermitteln will, hilft es mir auch bei der
  lokalen Fehlersuche.
\end{abstractsec}
\begin{normaltext}
  Ein Werkzeug, das sowohl bei der lokalen, als auch bei der Fehlersuche im
  Netzwerk behilflich sein kann, ist netstat. Auf den Aspekt
  Netzwerkfehlersuche gehe ich im Abschnitt \ref{sec:netz-werkzeuge-netstat}
  näher ein. Hier konzentriere ich mich auf die Fehlersuche bei lokalen
  Problemen.

  Dafür verwende ich vor allem die Optionen \verb?--protocol=unix?
  (alternativ: \verb?-A unix?) oder \verb?--unix? (\verb?-x?) um mir die
  UNIX-Sockets ausgeben zu lassen.
  Bin ich an den Prozessen interessiert, die die Netzwerksockets verwenden,
  kann ich diese stattdessen mit \verb?--inet?, \verb?--ipx?, \verb?--tcp?,
  \ldots selektieren. Das ist ausführlicher in Abschnitt
  \ref{sec:netz-werkzeuge-netstat} beschrieben.

  Mit \verb?--program? (\verb?-p?) erhalte ich die PID und den Namen des
  Prozesses, der den Socket benutzt. Dafür benötige ich Superuser-Privilegien.
  Mit dieser PID kann ich dann zum Beispiel den Prozess mit strace näher
  betrachten.

  Normalerweise zeigt netstat nur aktive, das heißt verbundene Sockets an. Mit
  der Option \verb?--listening? (\verb?-l?) kann ich dagegen nur die Sockets
  ausgeben lassen, die auf eine Verbindung warten oder mit \verb?--all?
  (\verb?-a?) alle.

  Mehr Informationen kann ich bekommen, wenn ich zusätzlich die Optionen
  \verb?--verbose? (\verb?-v?) oder \verb?--extend? (\verb?-e?) angebe.

  Die Ausgabe von netstat kommt als Tabelle, deren Spalten die folgende
  Bedeutung haben:
  \begin{description}
    \item[RefCnt] Zeigt die Anzahl der Prozesse, die sich mit dem Socket
      verbunden haben.
    \item[Flags] geben zusätzliche Informationen zum Zustand des Sockets aus:
      \begin{description}
        \item[ACC] - der Socket wartet auf eine Verbindung
        \item[W] - der Socket wartet auf Daten
        \item[N] - der Socket hat im Moment nicht genug Platz zum Schreiben
      \end{description}
    \item[Typ] kann stehen für
      \begin{description}
        \item[DGRAM] für verbindungslose Sockets
        \item[STREAM] für verbundene Sockets
        \item[RAW] für rohe ungefilterte Sockets %XXX FIXME
        \item[RDM] für zuverlässig ausgelieferte Nachrichten (Reliable
          Delivered Messages)
        \item[SEQPACKETS] für nacheinander folgende Pakete %XXX FIXME
        \item[PACKET] für rohe Interface-Sockets %XXX FIXME
      \end{description}
    \item[State] kann für einen der folgenden Zustände des Sockets stehen:
      \begin{description}
        \item[Free] - nicht allozierte Sockets %XXX FIXME
        \item[Listening] - nicht verbundene Sockets
        \item[Connecting, Connected, Disconnected] - die Phasen einer
          Socketverbindung
        \item[(empty)] für unverbundene Sockets %XXX FIXME
      \end{description}
    \item[PID] enthält die Prozess-ID und den Namen des Prozesses
    \item[Path] zeigt den Pfad zum Socket vom Prozess aus an, relativ zum
      Arbeitsverzeichnis des Prozesses
  \end{description}
\end{normaltext}

\section{iproute}
\label{sec:lokal-werkzeuge-iproute}
\begin{abstractsec}
  Ein Werkzeug, dass in keinem Werkzeugkasten für die lokale Fehlersuche fehlen
  Iproute ist ein Softwarepaket, dass ich am häufigsten bei Netzwerkproblemen
  einsetze. Zwei Programme daraus, ip und ss, nutze ich auch bei lokalen
  Problemen, daher stelle ich sie bereits hier vor. Im Kapitel über Werkzeuge
  für Netzwerkprobleme (\ref{cha:netz-werkzeuge}) gehe ich noch etwas
  ausführlicher auf iproute ein.
\end{abstractsec}
\begin{normaltext}
  Ein Werkzeug, dass in keinem Werkzeugkasten für die lokale Fehlersuche fehlen
  Iproute ist ein Softwarepaket, dass ich am häufigsten bei Netzwerkproblemen
  einsetze. Zwei Programme daraus, ip und ss, nutze ich auch bei lokalen
  Problemen, daher stelle ich sie bereits hier vor. Diese beiden Programme
  überschneiden sich in der Funktionalität der Programme mit ifconfig, route,
  netstat und arp.
  
  Im Kapitel über Werkzeuge
  für Netzwerkprobleme (\ref{cha:netz-werkzeuge}) gehe ich noch etwas
  ausführlicher auf iproute ein. 

  \subsection*{ip}

  Das Programm \verb?ip? funktioniert nach dem Schema
  \begin{verbatim}
ip [optionen] objekt befehl
  \end{verbatim}
  Die komplette Syntax ist in der Handbuchseite beschrieben. Ich bin bei der
  lokalen Fehlersuche vor allem an folgenden Objekten interessiert:
  \begin{description}
    \item[link] Schnittstellenparameter
    \item[addr] IP-Adressen
    \item[neigh] ARP- und NDISC-Caches
    \item[route] IP-Routingtabellen
  \end{description}
  Zu allen Objekten kann ich mit
  \begin{verbatim}
ip objekt help
  \end{verbatim}
  eine Kurzübersicht der möglichen Befehle und Optionen bekommen.

  Bei der lokalen Fehlersuche verwende ich am häufigsten den Befehl show, um
  mir den momentanen Zustand anzusehen. Die anderen Objekte und Befehle
  benötige ich eher selten zur Fehlersuche.

  \subsection*{ss}

  Das Programm \verb?ss? (show sockets) zeigt Informationen über Sockets an.
  Dabei kann ich auswählen, ob ich mit
  \begin{description}
    \item[-t] TCP-Sockets, mit
    \item[-u] UDP-Sockets, oder mit
    \item[-x] UNIX-Sockets
  \end{description}
  betrachten will.

  Die Ausgabe kann ich mit folgenden Optionen beeinflussen:
  \begin{description}
    \item[-n] schaltet die Namensauflösung ab.
    \item[-r] schaltet die Namensauflösung ein.
    \item[-l] zeigt Listening Sockets anstelle von Connected Sockets.
    \item[-a] zeigt sowohl Listening als auch Connected Sockets.
    \item[-p] zeigt die Prozesse, die einen Socket verwenden.
  \end{description}

  Auch hier helfen die Handbuchseiten weiter.
\end{normaltext}

\section{strace}
\label{sec:lokal-werkzeuge-strace}

\begin{abstractsec}
  Mit strace kann ich die Interaktion eines Programmes mit dem Linux-Kernel
  beobachten. Dieses verwende ich zum Beispiel um unvorhergesehene
  Programmabrüche oder unerklärliches Verhalten eines Binärprogrammes zu
  analysieren.
\end{abstractsec}
\begin{normaltext}
  Strace ist ein Werkzeug, dass ich einsetze, wenn mir das Verhalten eines
  Programmes unklar ist. Wenn ich Probleme mit Zugriffsrechten vermute, aber
  keinen Anhaltspunkt in den Fehlermeldungen oder Systemprotokollen finde.
  Wenn ich mit einem Programm noch wenig Erfahrung habe, kaum Hilfe zu meinem
  Problem im Internet finde, aber das Problem trotzdem so schnell wie möglich
  beheben will.
  In \cite{guug:uptimes:2012.1/07} führt Harald König sehr gut in die Arbeit
  mit strace ein.
  
  Strace hilft mir, wenn ich beobachten will, wie ein Programm
  mit seiner Umgebung interagiert. Es setzt dazu an der Kernelschnittstelle
  an und protokolliert alle Systemaufrufe mit den Parametern und Ergebnissen.
  Diese werden im Protokoll ähnlich den Systemaufrufen in der
  Programmiersprache C dargestellt: das Ergebnis steht hinter einem
  Gleichheitszeichen am Ende der Zeile, die Systemzeit steht vor dem
  Systemaufruf. Wenn ich mehrere Prozesse beobachte und in dieselbe Datei
  protokolliere, steht auch noch die Prozess-ID am Anfang der Zeile.

  Um die Ausgabe von strace interpretieren zu können, ist es hilfreich die
  Sektion der Handbuchseiten zu installiert zu haben. Diese befinden sich bei
  Debian-basierten Systemen im Paket manpages-dev.

  Da strace Systemaufrufe der beobachteten Prozesse protokolliert, verlangsamt
  sich deren Ablauf, was zu zusätzlichen Problemen bei der Fehlersuche führen
  kann. Dessen muss ich mir beim Einsatz von strace immer bewusst sein.

  Ich persönlich setze strace bei der Fehlersuche meist auf eine der folgenden
  Weisen ein.

  Bei Problemen mit mit Programmen, die von Hand aufgerufen werden, starte ich
  das betreffende Programm wie gewohnt, allerdings setze ich am Anfang der
  Kommandozeile strace mit einigen Optionen ein. Aus

  \begin{verbatim}
  $ make xyz
  \end{verbatim}

  wird dann

  \begin{verbatim}
  $ strace -f -o make.strace make xyz
  \end{verbatim}

  Dabei bedeuten die Optionen

  \begin{description}
    \item[-f] Strace soll auch von make gestartete Programme verfolgen.
    \item[-o make.strace] Die Ausgabe soll in die Datei make.strace
      geschrieben werden.
  \end{description}

  Bei Problemen mit bereits gestarteten Prozessen verwende ich die Option
  {\bf -p PID} um mich mit dem betreffenden Prozess zu verbinden. Alle
  weiteren Optionen verwende ich wie gehabt, auf die Angabe des Programmnamens
  und der Programmparameter kann ich verzichten.

  Schwieriger ist es, wenn ein Programm ein zweites aufruft, dieses ein
  drittes und so weiter. Wenn ich nur an einem der Programme interessiert bin
  und nicht genau weiss, von welchem Prozess beziehungsweise Programm dieses
  gestartet wird, helfe ich mir mit einem Trick. Ich benenne das eigentliche
  Programm um und ersetze es durch ein Skript, welche alle mich
  interessierenden Werte wie zum Beispiel die Aufrufparameter, die Umgebung,
  die UID/GID, die Standardeingabe und/oder -ausgabe protokolliert und
  schließlich das Originalprogramm via strace aufruft.
  Damit bekomme ich meist genügend Informationen, um das Problem zu lösen.
  Natürlich darf ich am Ende nicht vergessen, das Skript wieder durch das
  Originalprogramm zu ersetzen.

  \begin{notes}
  \item Was für eine Kernelschnittstelle nutzt strace?
  \item Strace-Skript: Datum, Kommandozeile, Umgebung, Benutzer,
    Standardeingabe, Strace-Ausgabe
  \end{notes}
\end{normaltext}

\section{ltrace}
\label{sec:lokal-werkzeuge-ltrace}
\begin{abstractsec}
  Ltrace ist, ähnlich strace, ein Programm, mit dem ich einem Prozess bei der
  Arbeit zusehen kann. Im Gegensatz zu strace, welches nur die
  Kernel-Schnittstelle beobachtet, zeigt ltrace den Aufruf von
  Bibliotheksfunktionen.
\end{abstractsec}
\begin{normaltext}
  Ltrace ist, ähnlich strace
  (siehe Abschnitt \ref{sec:lokal-werkzeuge-strace}),
  ein Programm, mit dem ich einem Prozess bei der
  Arbeit zusehen kann. Im Gegensatz zu strace, welches nur die
  Kernel-Schnittstelle beobachtet, zeigt ltrace den Aufruf von
  Bibliotheksfunktionen.

  Beim Start eines Programmes via ltrace läßt dieses den Prozess laufen, bis
  es endet. Dabei fängt ltrace Aufrufe von Bibliotheksfunktionen durch den
  Prozess und Signale an den Prozess ab und zeigt sie auf STDERR an. Das ist
  in etwa das Gleiche, was auch strace macht. Da aber ltrace
  Bibliotheksaufrufe abfängt, ist die Ausgabe viel feiner granuliert und
  umfangreicher.

  Die folgenden Optionen sind für die Fehlersuche mit ltrace interessant. Für
  eine komplette Liste der Optionen verweise ich auf die Handbuchseite.
  \begin{description}
    \item[-S] Mit dieser Option zeigt ltrace zusätzlich zu den
      Bibliotheksaufrufen auch Systemaufrufe an der Kernelschnittstelle.
      Diesen wird in der Ausgabe \verb?SYS_? vorangestellt.
    \item[-L] Diese Option ist nur zusammen mit \verb?-S? sinnvoll, da sie die
      Ausgabe der Bibliotheksaufrufe unterdrückt. Mit beiden Optionen zusammen
      zeigt ltrace etwa das gleiche an wie strace, der einzige Unterschied ist
      das vorangestellte \verb?SYS_? bei ltrace.
    \item[-e expr] Damit kann ich die Ereignisse/Funktionen, die angezeigt
      werden sollen, einschränken. Funktionen, die ich nicht sehen will,
      kennzeichne ich mit vorangestelltem \verb?!?. So kann ich zum Beispiel
      mit \verb?ltrace -e malloc,free? nachschauen, ob angeforderter Speicher
      auch wieder freigegeben wird, was insbesondere bei langlaufenden
      Prozessen von Belang ist.
    \item[-f] Diese Option bewirkt, genau wie bei strace, das auch
      Kindprozesse mit beobachtet werden.
    \item[-o dateiname] Das bewirkt, ebenfalls wie bei strace, das die Ausgabe
      von ltrace in die angegebene Datei anstatt zu STDERR ausgegeben wird.
    \item[-p pid] Mit dieser Option kann ich, wie bei strace, einen bereits
      laufenden Prozess untersuchen.
    \item[-i] Mit dieser Option zeigt ltrace den Befehlszeiger zu jedem
      Funktionsaufruf.
    \item[-r] Damit fügt ltrace relative Zeitstempel in die Ausgabe, mit denen
      ich zum Beispiel die Verzögerungen durch Timeouts genauer eingrenzen
      kann.
    \item[-T] Mit dieser Option zeigt ltrace die Zeit, die für die
      Funktionsaufrufe benötigt wurde.
    \item[-l dateiname] Diese Option erlaubt mir die Beobachtung
      einzuschränken auf Funktionsaufrufe aus dieser Bibliothek. Ich muss den
      kompletten Pfad zur Bibliothek angeben. Welche Bibliotheken ein Programm
      verwendet, bekomme ich mit ldd heraus. Wenn ich an mehreren Bibliotheken
      interessiert bin, kann ich diese Option mehrfach angeben.
  \end{description}
  \begin{notes}
  \item Diskussion: wann ltrace, wann strace
  \end{notes}
\end{normaltext}

\section{GDB der GNU Debugger}
\label{sec:lokal-werkzeuge-gdb}
\begin{abstractsec}
  GDB ist ein sehr mächtiges Werkzeug, das eher in extrem schwierigen Fällen
  zum Einsatz kommt. Vorzugsweise, wenn ich nachträglich die Ursache eines
  Programmabsturzes ermitteln will oder wenn ich einen Programmfehler vermute
  und finden will.
\end{abstractsec}
\begin{normaltext}
  GDB ist ein sehr mächtiges Werkzeug, das eher in extrem schwierigen Fällen
  zum Einsatz kommt. Vorzugsweise, wenn ich nachträglich die Ursache eines
  Programmabsturzes ermitteln will oder wenn ich einen Programmfehler vermute
  und finden will.

  Um mit dem Debugger zu arbeiten benötige ich Zugriff auf die Quellen, aus
  denen das betreffende Programm übersetzt wurde.
  \footnote{Zwar ist es auch möglich, direkt die Maschinenbefehle zu verfolgen,
  aber das liegt definitiv jenseits des Horizonts dieses Buches.}
  Außerdem brauche ich die Symboltabellen des Programms damit der Debugger die
  Maschinenbefehle des Binärprogramms den Quellcodezeilen zuordnen kann. Diese
  bekomme ich, wenn zum Beispiel beim Übersetzen des Programms mit dem
  Compiler gcc die Option \verb?-g? angegeben wurde. Bei vielen Softwarpaketen
  kann ich die Symboltabellen über das entsprechende Paket mit der Option
  \verb?-dbg? (zum Beispiel \verb?avahi-dbg? für \verb?avahi?) installieren.

  Um einen Prozess postmortal zu analysieren, muss ich das System anweisen ein
  sogenanntes Corefile zu schreiben, das den Zustand des Prozesses beim
  Programmabsturz enthält. Dazu kann ich in der Bash mit dem Befehl
  \verb?ulimit -c size? die maximale Größe des Corefiles festlegen, die das
  Betriebssystem schreibt. Diese muss ich hoch genug wählen, damit das
  Corefile für den gewünschten Prozess reicht.

  Ich kann den GNU Debugger auf verschiedene Arten starten:
  \begin{description}
    \item[gdb options program core] So starte ich, wenn ich einen
      Prozess postmortal analysieren will (mit \verb?core?) oder, wenn ich
      ein Programm nur beim Ablauf verfolgen will (ohne \verb?core?). Dabei
      ist \verb?program? der Name der Programmdatei und \verb?core? der Name
      des Corefiles.
    \item[gdb options --args program arguments] Hier gebe ich dem
      Programm, dass ich beobachten will gleich die Kommandozeilenargumente
      beim Aufrauf des GDB mit.
    \item[gdbtui options] Das startet den GDB mit einer
      Textbenutzeroberfläche, bei der im oberen Teil der aktuelle Quelltext
      angezeigt wird und im unteren Teil die Befehle und Ausgaben des GDB.
  \end{description}

  Von den Optionen, die GDB beim Aufruf mitgegeben werden können, sind die
  folgenden für die Fehlersuche relevant, weitere können der Handbuchseite,
  der GDB-Texinfo-Datei oder \verb?gdb -help? entnommen werden:
  \begin{description}
    \item[-c file | -core file] Damit gebe ich das Corefile bei den Optionen
      an und brauche es nicht mehr nach dem Programmnamen anzugeben.
    \item[-e file | -exec file] Gibt das ausführbare Programm an.
    \item[-s file | -symbols file] Gibt die Datei mit den Symboltabellen an.
      Die Optionen \verb?-s? und \verb?-e? können auch zu \verb?-se?
      zusammengefasst werden, wenn die Symboltabellen noch in der
      Binärprogrammdatei enthalten sind.
    \item[-help] Listed alle Optionen mit einer kurzen Erläuterung auf.
  \end{description}

  Nachdem ich GDB gestartet habe, steuere ich den Ablauf der Sitzung mit
  Textbefehlen. Alle diese Befehle können soweit abgekürzt werden, wie sie
  noch eindeutig sind.
  Die wichtigsten dieser Befehle für die Fehlersuche sind:
  \begin{description}
    \item[break function | break file:function] So ziemlich als erstes rufe
      ich in einer Debuggingsitzung \verb?break main? auf, damit der Debugger
      an dieser Funktion anhält und ich anschließend das Programm in Ruhe
      analysieren kann. Vor einer Funktion kann ich, durch \verb?:? getrennt
      die Datei angeben, falls der Debugger momentan eine andere geladen hat.
    \item[run | run arglist] Damit starte ich das Programm im Debugger.
      Optional kann ich dem Prozess mit \verb?arglist? einige
      Kommandozeilenargumente mitgeben.
    \item[bt] Dieser Befehl zeigt den Programmstack an. Bei einer postmortalen
      Analyse eines Prozesses rufe ich diesen Befehl als erstes auf, um
      herauszubekommen, wo genau das Programm abgestürzt ist.
    \item[print expr] Mit \verb?print? lasse ich mir die Werte in den
      Variablen und verschiedene andere Datein ausgeben. Dabei kann
      \verb?expr? ein komplexer C-Ausdruck sein.
    \item[c] Mit \verb?c? (continue) läuft das Programm weiter bis zum
      nächsten Haltepunkt oder bis zum Programmende.
    \item[next] Dieser Befehl arbeitet die nächste Zeile im Quelltext ab.
      Dabei werden Funktionsaufrufe ausgeführt und übersprungen.
    \item[step] Auch dieser Befehl arbeitet die nächste Zeile im Quelltext ab,
      der Debugger folgt hier allerdings Funktionsaufrufen in das Innere der
      Funktion.
    \item[list | list function | list file:function] Zeigt die aktuelle
      Programmumgebung beziehungsweise die angegebene Funktion im Quelltext.
    \item[help befehl] Gibt die Hilfe zu dem angegebenen Befehl aus.
    \item[quit] Beendet die Debuggingsitzung.
  \end{description}

  GDB ist ein sehr mächtiges Werkzeug für die Fehlersuche und auch sehr
  komplex in der Anwendung. Da die Bedienung nicht einfach ist und es auch
  einiger Vorkehrungen für den erfolgreichen Einsatz bedarf, setze ich ihn
  eher selten ein, quasi als Ultima Ratio. Trotzdem ist es sinnvoll, sich
  gelegentlich hinzusetzen und testweise das eine oder andere Programm im
  Debugger zu beobachten und zu analysieren, damit es im Ernstfall, wenn man
  darauf angewiesen ist, einfacher von der Hand geht.
\end{normaltext}

\section{vmstat}
\label{sec:lokal-werkzeuge-vmstat}

\begin{abstractsec}
  Das Programm vmstat verwende ich, um bei lokalen Performanceengpässen einen
  Überblick über das Gesamtsystem zu bekommen. Es liefert mir statistische
  Informationen über Prozesse, Speicher, I/O, Platen- und CPU-Aktivitäten.
\end{abstractsec}
\begin{normaltext}
  Das Programm vmstat verwende ich, um bei lokalen Performanceengpässen einen
  Überblick über das Gesamtsystem zu bekommen. Es liefert mir statistische
  Informationen über Prozesse, Speicher, I/O, Platen- und CPU-Aktivitäten.
  Durch die kompakte Darstellung kann ich im Wiederholungsmodus  ein Gefühl
  für die normale Aktivität des Gesamtsystems bekommen und sehe dann, wenn ein
  Performanceengpass auftritt, manchmal sofort, ob es ein allgemeines
  Ressourcenproblem ist (zuwenig Speicher, zu schwache CPU, zu langsame
  Platte, \ldots) oder ob ich mich doch eher mit dem betreffenden Programm
  beschäftigen muss.

  Für die meisten Optionen sind keine speziellen Privilegien erforderlich, das
  heißt, ich kann mit vmstat auch als normaler Benutzer mal eben nachsehen,
  wie es dem System geht.
  
  Im Wiederholungsmodus zeigt das Programm in der ersten Zeile für alle Werte
  den Durchschnitt seit Systemstart an und in den folgenden Zeilen die Werte
  für die betreffende Abtastperiode. Diesen Modus verwende ich am häufigsten:
  \begin{verbatim}
vmstat [-a] [-n] [periode [anzahl]]
  \end{verbatim}
  Lasse ich die beiden Zahlen für \verb?periode? und \verb?anzahl? weg, dann
  bekomme ich nur die Durchschnittswerte seit Systemstart angezeigt. Das
  brauche ich eigentlich nur, wenn ich das System sehr gut kenne oder mehrere
  ähnliche Systeme vergleichen will.

  Gebe ich eine Zahl für \verb?periode? an, dann verwendet vmstat diese als Länge der
  Periode in Sekunden mit der es kontinuierlich die Werte der letzten Periode
  ausgibt. Auch hier steht in der ersten Zeile der Durchschnitt seit
  Systemstart, so dass ich gleich ab der zweiten Zeile sehen kann, womit das
  System besonders beschäftigt ist.

  Gebe ich auch eine Zahl für \verb?anzahl? an, dann beendet sich vmstat nach
  soviel Perioden, wie angegeben.

  Die Ausgabe von vmstat in diesem Modus bedeutet folgendes:
  \begin{description}
    \item[Procs] Unter \verb?r? steht hier die Anzahl der Prozesse, die laufen
      können und unter \verb?b? steht die Anzahl der Prozesse in
      uninteruptible sleep, das heißt der Prozesse, die im Kernelcode auf I/O
      warten.
    \item[Memory] Hier habe ich unter \verb?swpd? die Menge des verwendeten
      virtuellen Speichers (des ausgelagerten Hauptspeichers). Unter
      \verb?free? finde ich den unbenutzten Speicher, unter \verb?buff?
      Speicher, der für Datei- und Socketpuffer verwendet wird. Der unter
      \verb?cache? aufgeführte Speicher wird für zwischengespeicherte Daten
      von Blockdevices verwendet.

      Habe ich die Option \verb?-a? angegeben, wird statt \verb?buffer/cache?
      aktiver und inaktiver Speicher angezeigt.
    \item[IO] Unter des Spalte \verb?bi? finde ich die Anzahl der von
      Blockdevices gelesenen Blöcke (blocks in) und unter \verb?bo? die
      geschriebenen (blocks out).
    \item[System] Hier finde ich unter \verb?in? die Anzahl der Interrupts pro
      Sekunde und unter \verb?cs? die Anzahl der Kontextwechsel pro Sekunde.
    \item[CPU] Diese Spalten zeigen prozentual, wie die CPU ihre Zeit
      verbringt. Unter\verb?us? steht der Anteil, den die CPU mit Code in
      Benutzerprogrammen verbringt. Unter \verb?sy? steht die Zeit für das
      Abarbeiten von Systemaufrufen und unter \verb?id? die untätige (idle)
      Zeit. Bis zum Kernel 2.5.41 zählte auch die Zeit, in der die CPU auf I/O
      wartete hierzu, ab dann gab es dafür die Spalte \verb?wa?.
  \end{description}
  \begin{notes}
  \item Was bedeutet aktive / inactive bei Memory?
  \end{notes}

  Im Wiederholungsmodus kann ich mit Option \verb?-a? die Anzeige des Speichers von
  buffer/cache auf inactive/active umschalten und mit \verb?-n? die
  wiederholte Anzeige der Kopfzeilen abschalten. Letztere sind insbesondere
  bei länger laufender Ausgabe praktisch, wenn ich nicht genau im Kopf habe,
  welche Spalte was anzeigt. Daher verwende ich \verb?-n? so gut wie nie.

  Mit Option \verb?-m?, die Superuserprivilegien erfordert, zeigt vmstat
  Informationen zum Slab Allocator an. Das ist ein Verfahren zur Verwaltung
  des Arbeitsspeichers. Weitere Informationen hierzu finden sich in
  \cite{Bonwick:1994:SAO:1267257.1267263}.

  Mit Option \verb?-d? zeigt vmstat einmalig Diskstatistiken für alle Disk
  Devices (das können auch RAM-Disks und Loop Devices sein) an.
  Dabei bedeuten:
  \begin{description}
    \item[Reads/Writes] für Lese-/Schreib-Zugriffe:
      \begin{description}
        \item[total] alle erfolgreichen Zugriffe
        \item[merged] gruppierte Zugriffe, die in einem I/O-Vorgang resultieren
        \item[sectors] erfolgreich gelesene/geschriebene Sektoren
        \item[ms] Anzahl der lesend/schreiben verbrachten Millisekunden
      \end{description}
    \item[I/O] Unter \verb?cur? die gerade laufende I/O und unter \verb?s? die
      Sekunden, die das System mit I/O verbracht hat.
  \end{description}

  Mit Option \verb?-D? zeigt vmstat einmalig zusammengefasste Statistiken zu
  allen Disks.

  Und mit \verb?-p partition? schließlich zeigt vmstat einmalig Statistiken fü
  diese Partition. Hierbei bedeuten in der Ausgabe:
  \begin{description}
    \item[reads] die Gesamtzahl der Lesezugriffe auf diese Partition
    \item[read sectors] die Gesamtzahl der gelesenen Sektoren dieser Partition
    \item[writes] die Gesamtzahl der Schreibzugriffe auf diese Partition
    \item[requested writes] die Gesamtzahl der Schreibanforderungen für diese
      Partition
  \end{description}
\end{normaltext}

\section{sysstat}
\label{sec:lokal-werkzeuge-sysstat}

\begin{abstractsec}
  Sysstat ist ein Werkzeug zur Einschätzung der Systemperformance.
\end{abstractsec}
\begin{normaltext}
  Sysstat ist ein Werkzeug zur Einschätzung der Systemperformance.

  Üblicherweise wird via cron aller 10 Minuten ein Skript gespeichert, das die
  Systemstatistiken sichert. Diese Statistiken können dann mit verschiedenen
  Auswertewerkzeugen ausgelesen werden.

  Alternativ können diese Werkzeuge auch selbst periodisch die
  Performancedaten sammeln und als aktuelle Schnappschüsse ausgeben.

  Zur Auswertung mit sysstat stehen die folgenden Werkzeuge zur Verfügung:
  \begin{description}
    \item[cifsiostat] liefert CIFS Statistiken
    \item[iostat] liefert CPU- und I/O-Statistiken für Geräte und Partitionen
    \item[mpstat] liefert (multi-)prozessorbezogene Statistiken
    \item[nfsiostat] liefert NFS-bezogene I/O-Statistiken
    \item[pidstat] liefert Statistiken über Linux-Tasks (-Prozesse)
    \item[sar, sadf] Sar sammelt, berichtet und sichert Informationen zu
      Systemaktivitäten, sadf gibt die von sar gesammelten Daten in
      verschiedenen Formaten aus.
  \end{description}

  Die Auswertung der Statistikdaten mit sar wurde bereits in
  \cite{Loukides:1996:SPT:547780} beschrieben. Da sich die Software seitdem
  weiterentwickelt hat, ist ein Blick in die Handbuchseiten unerlässlich.

  Für die grafische Auswertung gibt es verschiedene Programme. Ein
  Programm, sargraph, wird als Beispiel mit sysstat verteilt. Dieses wertet
  die XML-Ausgabe von sadf auf und übergibt sie an gnuplot zur Darstellung.
\end{normaltext}

\section{acct}
\label{sec:lokal-werkzeuge-acct}
\begin{abstractsec}
  Abrechnungsprogramme spielen eine wichtige Rolle im Gesamtbild der
  Performanceoptimierung. Sie liefern einen einfachen Weg herauszufinden, was
  ein Rechner macht. Welche Anwendungen laufen, wieviel Systemzeit verbrauchen
  diese Programme, wie stark belasten sie das System, und so weiter. Wenn man
  die Programme und ihr Verhalten im Großen und Ganzen kennt, kann man sich
  an die Strategie machen, um die Performance zu optimieren. Das Programmpaket
  acct liefert diese Abrechnungsdaten.
\end{abstractsec}
\begin{normaltext}
  Abrechnungsprogramme spielen eine wichtige Rolle im Gesamtbild der
  Performanceoptimierung. Sie liefern einen einfachen Weg herauszufinden, was
  ein Rechner macht. Welche Anwendungen laufen, wieviel Systemzeit verbrauchen
  diese Programme, wie stark belasten sie das System, und so weiter. Wenn man
  die Programme und ihr Verhalten im Großen und Ganzen kennt, kann man sich
  an die Strategie machen, um die Performance zu optimieren. Das Programmpaket
  acct liefert diese Abrechnungsdaten.

  Natürlich belastet das Führen der Statistiken das System zusätzlich. Und
  zusätzlichen Plattenplatz benötigen die Statistiken auch. Andererseits: wenn
  das System schon deutlich auf das Einschalten der Statistikerfassung
  reagiert, arbeitet es bereits in einem Bereich nahe der Belastungsgrenze und
  eine Analyse der Systemperformance und entsprechende Maßnahmen sind schon
  längst fällig.

  Ist das Paket acct installiert, so wird das Accounting meist automatisch
  beim Systemstart via accton eingeschaltet. Will man es deaktivieren, reicht
  es meist nicht, das über das Startscript auszuschalten, da durch cron in
  regelmäßigen Abständen die Protokolle komprimiert werden und dazu das
  Accounting ab und wieder angeschaltet wird. Um das Accounting zu
  deaktivieren ändert man bei Debian-Systemen in /etc/default/acct die
  Variable \verb?ACCT_ENABLE? auf 0.

  Die Abrechnugsdaten werden mit dem Programm sa ausgewertet. Dieses Programm
  gibt eine Tabelle mit einer Zeile für jedes Programm, mit der Anzahl der
  Aufrufe des Programms in der ersten Spalte und die folgenden Bezeichnungen
  in den anderen Spalten enthält:
  \begin{description}
    \item[cpu] die Summe von von CPU-System- und -Userzeit in CPU-Minuten
    \item[re] die ``wirkliche'' Laufzeit des Programms
    \item[k] der durchschnittliche Speicherverbrauch in KByte. Der
      Durchschnitt basiert auf der CPU-Zeit des Programms.
    \item[avio] die durchschnittliche Anzahl von I/O-Operationen pro
      Programmaufruf
    \item[tio] die Gesamtzahl der I/O-Operationen
    \item[k*sec] das Integral über den Speicher und die CPU-Zeit
    \item[s] die Systemzeit
    \item[u] die Benutzerzeit
  \end{description}
  Ganz rechts, ohne Bezeichnung steht der Programmname.  
  Ist dieser mit einem Asterisk ('*') gekennzeichnet, ist das Programm als
  Daemon gelaufen. Das heißt, es hat \verb?fork()? aufgerufen, aber nicht
  \verb?exec()?. Daemon-Prozesse sammeln durch ihre lange Laufzeit auch sehr
  viel CPU-Zeit an und verfälschen damit die Werte für den durchschnittlichen
  Speicherverbrauch.

  Die Tabelle ist nach der CPU-Zeit absteigend sortiert.

  Programme, die nur einmalig gelaufen sind, werden in der Zeile {\bf
  \verb?***other*?} zusammengefasst.

  Mit den folgenden Optionen kann ich die Ausgabe von sa modifizieren, die
  Handbuchseite kennt noch mehr davon:
  \begin{description}
    \item[-a | --list-all-names] Zeigt alle Programme (fasst keine Programme unter
      \verb?***other*? zusammen).
    \item[-b | --sort-sys-user-div-calls] Sortiert die Aufrufe nach der
      CPU-Zeit geteilt durch die Anzahl der Aufrufe.
    \item[-d | --sort-avio] Sortiert nach der durchschnittlichen Anzahl der
      I/O-Operationen.
    \item[-D | --sort-tio] Sortiert nach der Gesamtzahl der I/O-Operationen.
    \item[-i | --dont-read-summary-file] Ignoriert die Auswertungsdatei. Mit
      dieser Option zeigt sa die Prozesse seit dem letzten Aufruf von 
      \verb?sa -s?.
    \item[-k | --sort-cpu-avmem] Sortiert nach dem durchschnittlichen
      Speicherverbrauch. Dieser Report identifiziert die größten
      Speichernutzer.
    \item[-n | --sort-num-calls] Sortiert nach der Anzahl der Aufrufe,
      identifiziert die am häufigsten aufgerufenen Programme.
    \item[-r | --reverse-sort] Dreht die Sortierreihenfolge um.
    \item[-s | --merge] Fasst die aktuellen Accountingdaten in der
      Auswertungsdatei zusammen.
    \item[-t | --print-ratio] Zeigt das Verhältnis von Laufzeit zu CPU-Zeit,
      identifiziert Programme mit sehr viel Leerlauf.
  \end{description}
\end{normaltext}

\section{bonnie++}
\label{sec:lokal-werkzeuge-bonnie}
\begin{abstractsec}
  Bei jeglicher Art von Performance Tuning ist es essentiell, eine
  Bestandsaufnahme vor den Tuning-Maßnahmen und danach zu machen, um sich von
  der Wirkung des Tunings zu überzeugen.
  Bonnie++ ist ein Programm, mit dem ich die Performance der Lese- und
  Schreiboperationen im Dateisystem in Zahlen ausdrücken und damit dann
  vergleichen kann.
\end{abstractsec}
\begin{normaltext}
  Bei jeglicher Art von Performance Tuning ist es essentiell, eine
  Bestandsaufnahme vor den Tuning-Maßnahmen und danach zu machen, um sich von
  der Wirkung des Tunings zu überzeugen.
  Bonnie++ ist ein Programm, mit dem ich die Performance der Lese- und
  Schreiboperationen im Dateisystem in Zahlen ausdrücken und damit dann
  vergleichen kann.

  Dabei gibt das Programm für jeden Test, den es durchführt, zwei Kennzahlen
  aus: die geschaffte Arbeit (je mehr, um so besser) und die dafür benötigte
  CPU-Zeit (je weniger, umso besser).

  Die Tests teilen sich grob in zwei Abschnitte, die man gegebenenfalls auch
  überspringen kann. In einem Abschnitt testet bonnie++ den I/O-Durchsatz mit
  relativ großen Dateien, wie er ähnlich auch bei Datenbankanwendungen
  vorkommt. Im anderen Abschnitt geht es um das Erzeugen, Lesen und Löschen
  vieler relativ kleiner Dateien, wie es ähnlich auf Proxy-, Mail- und
  News-Servern vorkommt.

  In den meisten Fällen ist man eher daran interessiert, das I/O-Verhalten bei
  einzelnen Dateisystemen zu beobachten. Bei bestimmten Problemen möchte man
  jedoch den Einfluss von gleichzeitigen Dateizugriffen bewerten. Zu diesem
  Zweck ist es möglich, mehrere bonnie++ Prozesse synchron zu starten.

  Die Ausgabe von bonnie++ kommt, wie schon beim Vorgängerprogramm bonnie als
  achtzigspaltiger Text. Zusätzlich gibt bonnie++ die Werte auch noch als
  kommaseparierte Werte (CSV) aus, die einfacher weiterverarbeitet werden
  können und mehr als 80 Zeichen pro Zeile einnehmen können. Für diese
  CSV-Daten gibt es zwei Programme (\verb?bon_csv2html?, \verb?bon_csv2txt?),
  die die Daten für die HTML-Ausgabe beziehungsweise das bekannte Textformat
  aufbereiten. In deren Handbuchseiten sind die Felder der CSV-Daten
  beschrieben.

  Eine Warnung will ich noch vorwegschicken, bevor ich mich den Optionen
  zu wende, die, wie üblich, in den Handbuchseiten ausführlicher beschrieben
  werden.
  {\bf Bonnie++ sollte niemals auf aktiven Produktionsmaschinen laufen, da die
  Performance durch die Tests sehr stark beeinträchtigt wird.}

  \subsection*{Optionen}
  \begin{description}
    \item[-d dir] Das Verzeichnis, in dem die Testdateien angelegt werden.
      Ohne Angabe dieser Option werden die Testdateien im aktuellen
      Verzeichnis angelegt.
    \item[-s size] Die Größe der Dateien für die I/O-Performance-Tests. Mit
      einer Größe von 0 wird dieser Test übersprungen.
    \item[-n number] Die Anzahl der Dateien für den Dateierzeugungstest. Die
      Anzahl wird als Vielfache von 1024 angegeben. Ist die Anzahl 0, wird
      dieser Test übersprungen. Per Default werden leere Dateien angelegt. Es
      ist möglich, die maximale und minimale Größe der Dateien und die Anzahl
      der Verzeichnisse durch Doppelpunkt getrennt gemeinsam mit der Anzahl
      anzugeben. Details stehen in den Handbuchseiten.
    \item[-x number] Anzahl der Testläufe. Damit ist es möglich,
      hintereinander weg mehrere Tests zu machen, die Ergebnisse kommen
      kontinuierlich als CSV-Daten.
    \item[-u user] Der Benutzer, unter dem der bonnie++ Prozess laufen soll.
      Bonnie++ kann als normaler Nutzer gestartet werden. Startet man es als
      root, gibt man mit dieser Option besser einen anderen Benutzer vor, um
      Fehler am Dateisystem zu vermeiden.
    \item[-q] Mit dieser Option gibt bonnie++ nur die CSV-Daten an STDOUT aus.
      Dateien angelegt.
    \item[-f | -f size] Fast Mode Control, überspringt den zeichenweisen
      I/O-Test, wenn kein Parameter ansonsten gibt es die Testgröße für
      zeichenweisen I/O-Test vor (default 20M)
    \item[-b] Keine Pufferung der Schreiboperationen, das heisst es wird
      \verb?fsync()? nach jedem Schreiben aufgerufen.
    \item[-q] Quiet Mode. Es wird ein Teil der Ausgabe unterdrückt, an STDOUT
      werden nur die CSV-Daten ausgegeben, alles andere an STDERR. Damit ist
      es einfacher, die Ausgabe weiter zu verarbeiten.
    \item[-p number] Die Anzahl der der Prozesse, für die Semaphore
      reseerviert werden sollen. Alle Prozesse, die Semaphore mit Option
      \verb?-ys? verwenden, starten synchron.
    \item[-y s] Durch Semaphor synchronisiert starten
    \item[-y p] Mit Prompt synchronisieren. Bonnie++ startet erst, wenn
      \verb?<RETURN>? eingegeben wurde.
    \item[-D] Direct-I/O (\verb?O_DIRECT?) für Massen-I/O-Tests verwenden.
    \item[-z seed] Die Startzahl für den Zufallsgenerator angeben, um den
      gleichen Test zu wiederholen.
    \item[-Z file] Zufallsdaten aus der angegebenen Datei verwenden.
  \end{description}

  \subsection*{Synchrone Tests}
  Um mehrere Prozesse mit bonnie++ synchron zu starten, kann ich wie folgt
  vorgehen:
  \begin{verbatim}
$ bonnie++ -p3
$ bonnie++ [weitere Optionen] -ys > out1 &
$ bonnie++ [weitere Optionen] -ys > out2 &
$ bonnie++ [weitere Optionen] -ys > out3 &
  \end{verbatim}

  \subsection*{Signale}
  Bonnie++ kann mitunter recht lange laufen, vor allem wenn es mit Option
  \verb?-x?  seine Tests wiederholt.

  Mit SIGINT kann man die Ausführung abbrechen, bonnie++ räumt dann wieder
  auf, das heißt, es entfernt die temporären Dateien und Verzeichnisse. Das
  kann auch etwas dauern. Durch wiederholtes Senden von SIGINT bricht es auch
  das Aufräumen ab.

  SIGHUP wird ignoriert, das heißt, wenn es im Hintergrund läuft, bricht es
  auch nicht nach Abmelden vom Terminal ab.
\end{normaltext}

\section{hdparm}
\label{sec:lokal-werkzeuge-hdparm}
\begin{abstractsec}
  Mit hdparm kann ich von der Kommandozeile aus Einfluß nehmen auf die
  Schnittstelle zur Festplatte des Rechners. Ich verwende das Programm zum
  Feintuning der Festplattenzugriffe aber auch zum Verifizieren und Beheben
  von Plattenfehlern.
\end{abstractsec}
\begin{normaltext}
  Mit hdparm kann ich von der Kommandozeile aus Einfluß nehmen auf die
  Schnittstelle zur Festplatte des Rechners. Ich verwende das Programm zum
  Feintuning der Festplattenzugriffe aber auch zum Verifizieren und Beheben
  von Plattenfehlern. Es arbeitet mit der Kernelschnittstelle des SATA-, PATA-
  und SAS-Subsystems zusammen. Auch bei manchen USB-Festplatten kann ich
  Parameter mit hdparm verändern. Für einige der Optionen benötige ich eine
  aktuelle Kernelversion.

  Allgemein sieht der Aufruf von hdparm wie folgt aus:
  \begin{verbatim}
hdparm [optionen] [geraet ..]
  \end{verbatim}
  Es gibt Optionen, mit denen ich Parameter sowohl abfragen als auch setzen
  kann (get/set). Bei diesen Optionen gilt, dass sie ohne weiteres Argument
  den Parameter abfragen und mit zusätzlichem Argument den Wert entsprechend
  setzen.

  Rufe ich hdparm ganz ohne Parameter auf, verhält es sich,
  als hätte ich die Optionen \verb?-acdgkmur? angegeben.

  In der Datei \verb?/etc/hdparm.conf? kann ich die Default-Konfiguration für
  hdparm systemweit hinterlegen.

  Nachfolgend beschreibe ich die, aus meiner Sicht, wichtigsten Optionen.
  Weitere Informationen gibt es, wie fast immer, in den Handbuchseiten.
  \begin{description}
    \item[-a] (get/set) Anzahl der Sektoren für das Vorauslesen (read-ahead)
      im Dateisystem. Damit verbessert sich die Performance beim sequentiellen
      Lesen großer Dateien. Es gibt noch eine, davon separate,
      Read-Ahead-Funktion der Festplatte, diese Funktion unterstützt.
    \item[-A] (get/set) Ein- oder Ausschalten der Vorauslesefunktion der
      Festplatte. Mit \verb?-A0? wird es ausgeschaltet, mit \verb?-A1?
      eingeschaltet.
    \item[-B] (get/set) Einstellen Advanced Power Management (APM)
      Eigenschaften der Platte, soweit diese das unterstützt. Gültig sind
      Werte von 1 (meiste Energieeinsparung) bis 254 (höchste
      I/O-Performance), mit dem Wert 255 wird es ganz abgeschaltet. Werte von
      1-127 erlauben einen Spin-Down der Festplatte, Werte von 128-254
      erlauben das nicht.
    \item[-c] (get/set) 32-Bit-Support für (E)IDE ein- oder ausschalten.
      Mit 0 wird dieser Ausgeschaltet, 1 schaltet ihn ein und 3 schaltet den
      32-bit-Support mit speziellem Sync.

      Es geht hierbei um den Transfer via PCI oder VLB zum Hostadapter. Das
      Kabel zur Festplatte hat immer 16 Bit.
    \item[--fibmap dateiname] Gibt eien Liste von Blockextents
      (Sektorbereichen), den die Datei auf der Platte belegt.

      Damit kann ich mir die Fragmentierung einer Datei auf der Platte
      ansehen, oder die Blöcke für Fehlertests bestimmen.

      Wenn diese Option verwendet wird, muss sie die einzige sein.
    \item[-g] Zeigt die Laufwerksgeometrie (Zylinder, Köpfe, Sektoren), die
      Größe des Gerätes in Sektoren und den Startoffset von Beginn der Platte.
    \item[-i] Zeigt Identifizierungsinformationen, die die Kerneltreiber
      während des Systemstarts und der Konfiguration gesammelt haben.
    \item[-I] Zeigt die Identifizierungsinformationen, die das Laufwerk
      liefert.
    \item[-m] (get/set) Die Anzahl der Sektoren für multiple Sektor I/O. Mit
      dem Wert 0 wird das abgeschaltet. Die meisten IDE-Festplatten erlauben
      die Übertragung von mehreren Sektoren pro Interrupt. Damit läßt sich der
      System Overhead für Disk I/O um typisch 30 bis 50 Prozent reduzieren. Es
      kann allerdings in seltenen Fällen zu massiven Dateisystemfehlern
      führen.
    \item[--read-sector sektornummer] Liest den angegebenen Sektor und
      schreibt den Inhalt in Hex-Darstellung zum Standardausgang. Die
      Sektornummer muss als Dezimalzahl angegeben werden. Hdparm führt einen
      Low-Level-Read für diesen Sektor  aus. Damit kann diese Funktion als
      definitiver Test, ob ein Sektor schlecht ist, genommen werden. Um den
      Sektor durch die Platte ersetzen zu lassen, kann man die Option
      \verb?--write-sector? verwenden.
    \item[-t] Nimmt die Zeit von Lesezugriffen für Benchmarks und
      Vergleichsmessungen. Um brauchbare Werte zu erhalten, muss man diese
      Funktion mindestens zwei mal bei einem ansonsten inaktiven System (keine
      anderen aktiven Prozesse) und genügend freiem Hauptspeicher ausführen. 
      Diese Funktion zeigt, wie schnell Daten ohne den Overhead des
      Dateisystems gelesen werden können.
    \item[-T] Nimmt die Zeit von Cache-Read-Zugriffen. Auch diese Funktion
      sollte mindestens zwei mal bei ansonsten inaktivem System wiederholt
      werden. Das zeigt den Durchsatz von Prozessor, Cache und RAM.
    \item[--write-sector sektornummer] Schreibt Nullen in den Sektor. {\bf
      Sehr gefährlich!} Irrt man
      sich in der Sektornummer, werden möglicherweise vitale Informationen des
      Dateisystems überschrieben. Die Sektornummer wird als Dezimalzahl
      angegeben.
      Diese Funktion kann zum Anstoßen der automatischen
      Reparatur des Sektors durch die Festplatte verwendet werden.
      Sinnvollerweise vergewissert man sich vorher mit der Option
      \verb?--read-sector?, das man es wirklich mit einem defekten Sektor zu
      tun hat.
    \item[-W] (get/set) Zeigt beziehungsweise modifiziert das Write Caching
      des IDE/SATA Laufwerks. 1 bedeutet, dass es eingeschaltet ist.
    \item[-z] Zwingt den Kernel, die Partitionstabelle neu zu lesen.
  \end{description}
  In der Datei /etc/hdparm.conf kann die Default-Konfiguration für hdparm
  hinterlegt werden. Diese Datei ist gut kommentiert, so dass ich mir weitere
  Erläuterungen hier spare.
\end{normaltext}

%%% Local Variables: 
%%% mode: latex
%%% TeX-master: "arbeit-hauptdatei"
%%% End: 
%%% vim: set sw=2 ts=2 tw=78 et si:

%% tl-lokal-bootprobleme.tex
\chapter{Bootprobleme}
\label{cha:bootprobleme}

\begin{abstractsec}
  Probleme beim Booten des Rechners betrachte ich als lokalen Totalausfall.
\end{abstractsec}

\begin{notes}
\item Partitionen von Festplattenimage einhängen
%\item geht auch nicht
%\item und auch das
\end{notes}

\section{Bootprobleme bei virtuellen Maschinen}
\label{sec:lokal-bootprobleme-vm}

\subsection*{Partitionen von Festplattenimages einhängen}
\label{sec:mount-hdimage-partition}

Bei Problemen mit dem Start von virtuellen Maschinen ist es oft notwendig,
wenigstens die Bootpartition der betroffenen VM zu untersuchen.

Manchmal kann
ich dazu das Festplattenimage der defekten VM einfach einer anderen VM
zuordnen und es von dieser aus untersuchen. Das erscheint zumindest bei
grafischen Oberflächen für die Administration als der einfachste Weg.
Allerdings muss ich dann immer zwischen dem Hostsystem beziehungsweise der
Administrationsoberfläche und der VM, mit der ich das Festplattenimage
untersuche wechseln und verliere viel Zeit beim Zuordnen, Ein- und Aushängen
und den Startversuchen.

Besser ist in meinen Augen, die Partitionen des betroffenen Images gleich im
Hostsystem für die Analyse und Störungsbeseitigung einzubinden und die
Vorteile der Shell für zügiges Arbeiten zu nutzen.

Nun habe ich aber das Problem, dass ich auf dem Hostsystem nicht, wie bei den
eigenen Festplatten, die Partitionen direkt zur Verfügung habe, sondern nur
das Komplettimage der Festplatte für die VM. Und dieses beginnt nicht mit dem
Dateisystem sondern mit der Partitionstabelle. Um die gewünschte Partition
einzubinden, muss ich dem Mount-Befehl den Offset der Partition mitgeben.

Den Offset kann ich mit dem Programm fdisk bestimmen. Dieses listet mit der
Option \verb?-l? die Partitionen und deren Offsets auf. Da wir letztere genau
bestimmen müssen, verwenden wir zusätzlich die Option \verb?-u?, damit fdisk
die Offsets als Anzahl von Sektoren zu je 512 Byte ausgibt:

\begin{verbatim}
# fdisk -l -u /dev/camion/ssh3 

Disk /dev/camion/ssh3: 4294 MB, 4294967296 bytes
255 heads, 63 sectors/track, 522 cylinders, total 8388608 sectors
Units = sectors of 1 * 512 = 512 bytes
Sector size (logical/physical): 512 bytes / 4096 bytes
I/O size (minimum/optimal): 4096 bytes / 4096 bytes
Disk identifier: 0x000c48f7

            Device Boot      Start         End      Blocks   Id  System
/dev/camion/ssh3p1            2048     7706623     3852288   83  Linux
/dev/camion/ssh3p2         7708670     8386559      338945    5  Extended
Partition 2 does not start on physical sector boundary.
/dev/camion/ssh3p5         7708672     8386559      338944   82  Linux swap
\end{verbatim}

Der Offset für die Systempartition \verb?ssh3p1? ist $512 * 2048$, also
1048576. Damit kann ich diese Partition im Hostsystem wie folgt einhängen:

\begin{verbatim}
# mount /dev/camion/ssh3 /tmp/mnt -o loop,offset=1048576
\end{verbatim}

Wenn ich fertig bin, hänge ich die Partition normal mit umount wieder aus.

Wichtig ist, dass ich die Partition im Hostsystem nur einhänge, wenn die VM
nicht läuft. Das ist beim Untersuchen von Bootproblemen meist gegeben. Bei
laufenden VMs habe ich mit LVM die Möglichkeit, einen Snapshot anzufertigen
und diesen Snapshot nur-lesend einzuhängen. Dabei muss ich aber bedenken, dass
Dateien, die in der VM geöffnet waren, eventuell in einem inkonsistenten
Zustand sind. Für Backups kann ich jedoch die VM kurz runterfahren, den
Snapshot anlegen und die VM gleich wieder starten.

%\begin{abstractsec}
%  Die Aufteilung der Umsetzung wird hier gegliedert in verschiedene Aspekte
%\end{abstractsec}
%\begin{normaltext}
%  Blablabla
%\end{normaltext}

%\subsection{Einschränkung der Umsetzung}
%\label{sec:einschr-der-umsetz}

%\section{Schnittstellen nach außen}
%\label{sec:schn-nach-au3en}

%\subsection{Schnittstelle zu A}
%\label{sec:schnittstelle-zu}

%\begin{notes}
%\item Syntax
%\item Semantik
%\end{notes}

%\subsection{Schnittstelle zu B}
%\label{sec:schnittstelle-zu-b}

%\begin{notes}
%\item Syntax
%\item Semantik
%\end{notes}


%%% Local Variables: 
%%% mode: latex
%%% TeX-master: "arbeit-hauptdatei"
%%% End: 
%%% vim: set sw=2 ts=2 tw=78 et si:

%% tl-lokal-programmfehler.tex
\chapter{Partielle Ausfälle}
\label{cha:partielle-ausfaelle}

\begin{abstractsec}
  Der Rechner läuft, aber ein Dienst will um's Verrecken nicht starten. Ein
  Programm bricht immer mit Fehlermeldung ab. Ein Verzeichnis läßt sich nicht
  einhängen. Das sind Beispiele für partielle Ausfälle auf dem lokalen System,
  dem Linux-Rechner.
\end{abstractsec}

\section{Mount-Fehler}
\label{sec:mount-fehler}

\begin{abstractsec}
  Eine Klasse für sich sind Fehler beim Ein- und Aushängen von Dateisystemen.
  Seien das Dateisysteme auf mobilen Datenträgern oder über das Netz bezogene
  Dateisysteme
\end{abstractsec}
%\begin{normaltext}
%  Blablabla
%\end{normaltext}

\subsection*{mount: / is busy}
\label{sec:mount-is-busy}
\begin{normaltext}
  Dieser Fehler tritt mitunter bei umount auf, aber auch, wenn ich ein
  Dateisystem von read-write auf read-only umhängen will.
  Die Meldung \verb?is busy? deutet es schon an, der Kernel, genauer gesagt
  das Dateisystem ist (noch) beschäftigt. Wenn ich das Dateisystem aushängen
  will, reicht ein Prozess, der auf irgendeine Art auf das Dateisystem
  zugreift. Will ich es read-only umhängen, muss ich nach Prozessen suchen,
  die Dateien zum Schreiben geöffnet haben. In \cite{weidner12:linuxkopflos}
  habe ich einen solchen Fall detailliert beschrieben, hier gehe ich nur kurz
  auf die Schritte ein um die betreffenden Prozesse und die geöffneten Dateien
  zu finden. Die nächsten Schritte sind dann vom konkreten Fall abhängig.

  \begin{Exkursbox}{Offene gelöschte Dateien}
    Übrigens, auch Dateilöschungen sind Schreibzugriffe. Wenn
    Systembibliotheken aktualisiert werden, dann wird die alte Datei mit dem
    Systembefehl \verb?unlink()? gelöscht und die neue Datei mit dem gleichen
    Namen gespeichert. Neu gestartete Prozesse verwenden dann die neue
    Bibliothek. Prozesse, die bereits vor der Aktualisierung liefen, arbeiten
    weiter mit der alten Bibliothek, weil sie diese ja vor der Aktualisierung
    geöffnet hatten.
    
    Abgesehen von den möglichen Sicherheitsproblemen, die die alte Bibliothek
    vielleicht noch hat und damit auch die alten Prozesse, kann ich nun das
    Dateisystem mit der alten Bibliothek nicht read-only umhängen. Denn die
    alte Bibliothek ist zwas nicht mehr im Dateisystem gelinkt, die
    betreffenden Blöcke können aber erst freigegeben werden, wenn der letzte
    Prozess, der sie verwendet, beendet ist. Und erst danach kann ich das
    Dateisystem read-only umhängen.

    Das gleiche gilt für Dateien, die geöffnet werden und unmittelbar darauf
    im Dateisystem gelöscht. Diese sind damit nur noch für den Prozess, der
    sie geöffnet hat, verwendbar und die zugehörigen Speicherblöcke werden
    erst mit Beenden dieses Prozesses im Dateisystem freigegeben.
  \end{Exkursbox}

  Prozesse, die in irgendeiner Weise auf ein Dateisystem zugreifen, finde ich
  am schnellsten mit:
\begin{verbatim}
# fuser -vm $mntpnt
\end{verbatim}
  Diesen Befehl rufe ich mit Superuserprivilegien auf, um alle Prozesse
  angezeigt zu bekommen.
  Dabei interessieren mich neben der PID und dem Prozessnamen (COMMAND) vor
  allem die Angaben unter ACCESS. Insbesondere Prozesse mit \verb?F? und
  \verb?m?, da diese es sind, die das read-only Umhängen des Dateisystems
  verhindern.
  Ich könnte diese Prozesse mit Option \verb?-k? gleich von fuser beenden
  lassen, aber besser ist es, erst mit \verb?lsof -p $pid? nachzuschauen,
  welcher Prozess das genau ist und welche Dateien er konkret offen hält.
  Auch kann ich erstmal mit \verb?pstree -p? abschätzen, ob ich vielleicht
  vitale Systemfunktionen beende, wenn ich den Prozess einfach abschieße.
  Handelt es sich um einen Systemdienst, wie SSH, dann ist meist der
  Listening-Daemon bereits neu gestartet und nur die Instanz, über die ich
  angemeldet bin, verwendet noch die alte Bibliothek. Dann reicht es mitunter,
  wenn ich mich ein zweites Mal anmelde und danach die alte Verbindung trenne.
\end{normaltext}

%\subsection{Einschränkung der Umsetzung}
%\label{sec:einschr-der-umsetz}

%\begin{notes}
%\item Was nicht geht
%\item geht auch nicht
%\item und auch das
%\end{notes}

%\section{Schnittstellen nach außen}
%\label{sec:schn-nach-au3en}

%\subsection{Schnittstelle zu A}
%\label{sec:schnittstelle-zu}

%\begin{notes}
%\item Syntax
%\item Semantik
%\end{notes}

%\subsection{Schnittstelle zu B}
%\label{sec:schnittstelle-zu-b}

%\begin{notes}
%\item Syntax
%\item Semantik
%\end{notes}


%%% Local Variables: 
%%% mode: latex
%%% TeX-master: "troubleshoot-linux"
%%% End: 
%%% vim: set sw=2 ts=2 tw=78 et si:

%% tl-lokal-performance.tex
\chapter{Performance des Rechner}
\label{cha:lokal-performance}

\begin{abstractsec}
  Der Rechner funktioniert zwar, aber alles erscheint z�h und in Zeitlupe
  abzulaufen. Jobs brauchen eine gef�hlte Ewigkeit, die Tastatur scheint mit
  8 Baud angeschlossen. Irgendwo sind Engp�sse, aber wo?
\end{abstractsec}

\begin{notes}
\item CPU?
\item Speicher?
\item Input/Output?
\end{notes}

\section{Input/Output-Performance}
\label{sec:input-output-performance}

%\begin{abstractsec}
%  Die Aufteilung der Umsetzung wird hier gegliedert in verschiedene Aspekte
%\end{abstractsec}
%\begin{normaltext}
%  Blablabla
%\end{normaltext}

\subsection{Festplatten}
\label{sec:performance-festplatten}

\begin{notes}
\item HD Performancemessung mit hdparm (ALIX performance with an
  internal disk)
\end{notes}

%\subsection{Einschr�nkung der Umsetzung}
%\label{sec:einschr-der-umsetz}

%\section{Schnittstellen nach au�en}
%\label{sec:schn-nach-au3en}

%\subsection{Schnittstelle zu A}
%\label{sec:schnittstelle-zu}

%\subsection{Schnittstelle zu B}
%\label{sec:schnittstelle-zu-b}

%\begin{notes}
%\item Syntax
%\item Semantik
%\end{notes}


%%% Local Variables: 
%%% mode: latex
%%% TeX-master: "troubleshoot-linux"
%%% End: 
%%% vim: set sw=2 ts=2 tw=78 et si:


%\part{Netzwerkprobleme}
%%% tl-part3.tex
\part{Netzwerkprobleme}

%%% Local Variables: 
%%% mode: latex
%%% TeX-master: "troubleshoot-linux"
%%% End: 
%%% vim: set sw=2 ts=2 tw=78 et si:

%% tl-netz-werkzeuge.tex
\chapter{Werkzeuge zur Netzwerkfehlersuche}
\label{cha:netz-werkzeuge}

\begin{abstractsec}
  Verschiedene Werkzeuge eignen sich zur Fehlersuche im Netz. Hier stelle ich
  diejenigen vor, die ich in den nächsten Kapiteln einsetzen will.
\end{abstractsec}

\begin{notes}
\item[iproute, bridge-utils, netstat, route] zur Abfrage,
  Manipulation der (Kernel-)Netzwerkschnittstellen und Kernelrouten
\item[mii-tools, eth-tools] für Hardware-Ethernet-Schnittstellen
\item[ping, traceroute, nmap] zur ersten Diagnose von Verbindungsproblemen
\item[tcpdump, wireshark, libtrace-tools] zur Diagnose von hartnäckigen
  Verbindungsproblemen
\item[quagga] zur Diagnose und Manipulation von Routingprotokollen und
  -tabellen
\item[telnet, nc, openssl, smbclient] zum Testen von Anwendungsprotokollen
\item[iperf, ttcp, nuttcp] zur Diagnose von Performanceproblemen
\end{notes}

\section{arp}
\label{sec:netz-werkzeuge-arp}
\begin{abstractsec}
  Das Programm arp dient der Anzeige und Manipulation des ARP-Caches des
  Kernels und wird bei Problemen in direkt angeschlossenen Netzwerksegmenten
  eingesetzt.
\end{abstractsec}
\begin{normaltext}
  Das Programm arp dient der Anzeige und Manipulation des ARP-Caches des
  Kernels und wird bei Problemen in direkt angeschlossenen Netzwerksegmenten
  eingesetzt.

  ARP, das (Ethernet) Address Resolution Protocol, dient der Zuordnung von
  Internetadressen zu Ethernet-Adressen. Es wurde in RFC826 beschrieben und in
  RFC5227 und RFC5494 aktualisiert.

  Der prinzipielle Aufruf ist:
  \begin{verbatim}
# arp [Optionen] [Rechnername]
  \end{verbatim}

  Bei Aufruf ohne Optionen zeigt arp die MAC-Adressen an, die Rechnername
  zugeordnet sind. Fehlt Rechnername, werden alle bekannten Adresszuordnungen
  angezeigt. Rechnername kann dabei ein Hostname sein, ein FQDN oder eine
  IP-Adresse.

  Die für die Fehlersuche wichtigsten Optionen sind:
  \begin{description}
    \item[-n] damit unterlässt das Programm die Namensauflösung bei der
      Anzeige
    \item[-d Rechnername] entfernt alle Einträge für Rechnername
    \item[-s Rechnername Hardwareadresse] setzt den Eintrag für Rechnername
      auf Hardwareadresse
  \end{description}
  Für weitergehende Informationen verweise ich wieder auf die Handbuchseite.
\end{normaltext}

\section{ifconfig}
\label{sec:netz-werkzeuge-ifconfig}
\begin{abstractsec}
  Das Programm ifconfig dient traditionell der Konfiguration der
  Netzwerkschnittstellen. Außerdem liefert es Informationen über den aktuellen
  Zustand und die Konfiguration der Netzwerkschnittstelle.
\end{abstractsec}
\begin{normaltext}
  Das Programm ifconfig dient traditionell der Konfiguration der
  Netzwerkschnittstellen. Außerdem liefert es Informationen über den aktuellen
  Zustand und die Konfiguration der Netzwerkschnittstelle.

  Es gibt drei prinzipielle Aufrufmöglichkeiten für ifconfig. Mit
  \begin{verbatim}
# ifconfig [-a]
  \end{verbatim}
  zeigt das Programm den Status der aktiven Schnittstellen (ohne \verb?-a?)
  beziehungsweise aller (aktiven und nichtaktiven) Schnittstellen (mit
  \verb?-a?) an.

  Durch Aufruf von
  \begin{verbatim}
# ifconfig $schnittstellenname
  \end{verbatim}
  zeigt das Programm den Zustand genau der angegebenen Schnittstelle an.

  In der dritten Form,
  \begin{verbatim}
# ifconfig $schnittstellenname $optionen
  \end{verbatim}
  wird die Schnittstelle konfiguriert.

  Die wichtigsten Optionen beim Troubleshooting sind
  \begin{description}
    \item[up] zum Aktivieren von Schnittstellen
    \item[down] zum Deaktivieren
    \item[mtn N] zum Setzen der Maximum Transfer Unit
    \item[netmask M] zum Setzen der Netzmaske
    \item[hw A] zum Setzen der Hardware-Adresse
  \end{description}

  Weitere Informationen liefert die Handbuchseite.

  Es ist möglich, an eine Netzwerkschnittstelle mehrere IP-Adressen zu binden.
  Das Programm ifconfig arbeitet jedoch mit einer Adresse pro Schnittstelle.
  Um weitere Adressen an diese Schnittstelle zu binden, fügt man an den
  Schnittstellennamen einen Doppelpunkt und eine Zahl an.

  In letzter Zeit werden die Netzwerkschnittstellen auch mit dem Programm ip
  vom Paket iproute2 konfiguriert, welches in Sektion
  \ref{sec:netz-werkzeuge-iproute} beschrieben ist.
  Für den schnellen Überblick über die aktuelle Schnittstellenkonfiguration
  gefällt mir die Ausgabe von \verb?ip addr show? besser, da sie kompakter
  ist, was sich insbesondere dann auszahlt, wenn ein Rechner mehrere
  Netzwerkschnittstellen hat, oder mehrere Adressen auf einer Schnittstelle.

  Hier habe ich die Ausgabe für einen Rechner:
  \begin{verbatim}
$ /sbin/ifconfig -a
eth0      Link encap:Ethernet  HWaddr 00:01:6c:6f:c5:d6  
          inet addr:192.0.2.5  Bcast:192.0.2.255  Mask:255.255.255.0
          inet6 addr: 2001:db8::201:6cff:fe6f:c5d6/64 Scope:Global
          inet6 addr: fe80::201:6cff:fe6f:c5d6/64 Scope:Link
          UP BROADCAST RUNNING MULTICAST  MTU:1500  Metric:1
          RX packets:12259377 errors:0 dropped:0 overruns:0 frame:0
          TX packets:13256939 errors:0 dropped:0 overruns:0 carrier:1
          collisions:0 txqueuelen:1000 
          RX bytes:3104599402 (2.8 GiB)  TX bytes:1370040423 (1.2 GiB)
          Interrupt:30 

eth0:0    Link encap:Ethernet  HWaddr 00:01:6c:6f:c5:d6  
          inet addr:192.0.2.246  Bcast:192.0.2.255  Mask:255.255.255.0
          UP BROADCAST RUNNING MULTICAST  MTU:1500  Metric:1
          Interrupt:30 

lo        Link encap:Local Loopback  
          inet addr:127.0.0.1  Mask:255.0.0.0
          inet6 addr: ::1/128 Scope:Host
          UP LOOPBACK RUNNING  MTU:16436  Metric:1
          RX packets:5386162 errors:0 dropped:0 overruns:0 frame:0
          TX packets:5386162 errors:0 dropped:0 overruns:0 carrier:0
          collisions:0 txqueuelen:0 
          RX bytes:668745964 (637.7 MiB)  TX bytes:668745964 (637.7 MiB)

wlan0     Link encap:Ethernet  HWaddr 0c:60:76:7c:3e:25  
          BROADCAST MULTICAST  MTU:1500  Metric:1
          RX packets:0 errors:0 dropped:0 overruns:0 frame:0
          TX packets:0 errors:0 dropped:0 overruns:0 carrier:0
          collisions:0 txqueuelen:1000 
          RX bytes:0 (0.0 B)  TX bytes:0 (0.0 B)
  \end{verbatim}
  Die Ausgabe von \verb?ip? ist demgegenüber wesentlich kompakter.
  \begin{verbatim}
$ ip addr show
1: lo: <LOOPBACK,UP,LOWER_UP> mtu 16436 qdisc noqueue state UNKNOWN 
    link/loopback 00:00:00:00:00:00 brd 00:00:00:00:00:00
    inet 127.0.0.1/8 scope host lo
    inet6 ::1/128 scope host 
       valid_lft forever preferred_lft forever
2: eth0: <BROADCAST,MULTICAST,UP,LOWER_UP> mtu 1500 qdisc pfifo_fast state UP qlen 1000
    link/ether 00:01:6c:6f:c5:d6 brd ff:ff:ff:ff:ff:ff
    inet 192.0.2.246/24 brd 192.0.2.255 scope global eth0:0
    inet 192.0.2.5/24 brd 192.0.2.255 scope global secondary eth0
    inet6 2001:db8:201:6cff:fe6f:c5d6/64 scope global dynamic 
       valid_lft 86345sec preferred_lft 14345sec
    inet6 fe80::201:6cff:fe6f:c5d6/64 scope link 
       valid_lft forever preferred_lft forever
3: wlan0: <BROADCAST,MULTICAST> mtu 1500 qdisc noop state DOWN qlen 1000
    link/ether 0c:60:76:7c:3e:25 brd ff:ff:ff:ff:ff:ff
  \end{verbatim}

\end{normaltext}

\section{iproute}
\label{sec:netz-werkzeuge-iproute}
\begin{abstractsec}
  Die Programme der iproute-Suite sind moderne Werkzeuge zur Anzeige und
  Konfiguration von Netzwerkschnittstellen, -routen und -verkehrskontrolle.

  Sie bieten gegenüber den Programmen der net-tools-Suite erweiterte
  Möglichkeiten.
\end{abstractsec}
\begin{normaltext}
  Die Programme der iproute-Suite sind moderne Werkzeuge zur Anzeige und
  Konfiguration von Netzwerkschnittstellen, -routen und -verkehrskontrolle.

  Sie bieten gegenüber den Programmen der net-tools-Suite erweiterte
  Möglichkeiten.

  Mit dem Kernel kommunizieren sie über die (rt)netlink-Schnittstelle.

  Die iproute-Suite umfasst unter anderem die folgenden Programme:
  \begin{description}
    \item[ip] zum Anzeigen und Konfigurieren von Netzwerkschnittstellen,
      -adressen, -routen, Policy-Regeln, ARP- und NDISC-Einträgen, IP-Tunneln
      und Multicast
    \item[ss] zur Anzeige von Socketstatistiken
    \item[tc] zur Anzeige und Konfiguration von Netzwerkverkehrskontrolle
      (traffic control)
    \item[arpd] einen Userspace-ARP-Daemon
    \item[*stat, rtacct] verschiedene Statistikwerkzeuge
  \end{description}

  Weitere und genauere Informationen stehen in den betreffenden Handbuchseiten
  und der zugehörigen Dokumentation.

  \subsection*{ip}

  Das Program ip dient zum Anzeigen / Manipulieren von Routen, Schnittstellen,
  Policy-Routing und Tunneln. Es ist das Programm aus der Suite, das ich am
  häufigsten interaktiv aufrufe.

  Allgemein sieht der Aufruf des Programms wie folgt aus:
  \begin{verbatim}
  ip [ Optionen ] Objekt [ Befehl [ Argumente ] ]
  \end{verbatim}

  Die Optionen sind allgemeine Modifikatoren, die das Verhalten des Programms
  ändern, wie zum Beispiel \verb?-4? und \verb?-6?, die die Adressfamilie auf
  IPv4 oder IPv6 einschränken.

  Das Objekt gibt an, worüber ich Informationen wünsche, beziehungsweise, was
  ich manipulieren will. Mögliche Objekte sind unter anderem:
  \begin{description}
    \item[link] Netzwerkschnittstellen
    \item[address] Protokoll-Adressen an einer Schnittstelle
    \item[neighbour] ARP- oder NDISC-Einträge
    \item[route] Einträge in der Kernel-Routingtabelle
    \item[rule] Regeln in der Policydatenbank
    \item[maddress] Multicast-Adressen
    \item[mroute] Multicast-Routingeinträge
    \item[tunnel] IP-Tunnel
    \item[monitor] Nachrichten auf der netlink-Schnittstelle des Kernels
  \end{description}

  Der Befehl schließlich gibt die Aktion an, die wir ausführen wollen und wird
  gegebenenfalls von passenden Argumenten gefolgt. Befehl und Argumente sind
  spezifisch für die entsprechenden Objekte.
\end{normaltext}

  \subsection*{ss}

  Das Programm ss (socket statistics) liefert Informationen und Statistiken zu
  Sockets. Ähnliche Informationen kann ich zum Beispiel auch mit netstat
  bekommen, jedoch habe ich bei ss mehr Möglichkeiten zur Filterung, was mir
  insbesondere bei Servern mit vielen Verbindungen zugute kommt.

  Der Aufruf ist wie folgt:
  \begin{verbatim}
  ss [ Optionen ] [ Filter ]
  \end{verbatim}

  Debei zeigt ss ohne Optionen die verbundenen TCP-Sockets an.

  Die wichtigsten Optionen sind unter anderen:
  \begin{description}
    \item[-a] um alle Sockets anzuzeigen
    \item[-l] um nur die Listening Sockets anzuzeigen (diese werden per
      Default ausgelassen)
    \item[-p] um die zugehörigen Prozesse anzuzeigen (dafür benötige ich
      Rootrechte)
    \item[-t | -u | -w | -x] um TCP-, UDP-, Raw- oder Unix-Sockets auszuwählen
  \end{description}

  Der Filter beim Aufruf von ss hat die allgemeine Form:
  \begin{verbatim}
  [ TCP-Status ] [ Ausdruck ]
  \end{verbatim}

  Um nach TCP-Status zu filtern, gebe ich das Schlüsselwort \verb?state? oder
  \verb?exclude? gefolgt von einem der Standard-TCP-Zustandsnamen oder einem
  der folgenden an:
  \begin{description}
    \item[all] für alle Zustände
    \item[bucket] für TCP-Minisockets (TIME-WAIT|SYN-RECV)
    \item[big] für alle außer den Minisockets
    \item[connected] für alle nicht geschlossenen und nicht lauschenden
      Sockets
    \item[synchronized] für alle verbundenen Sockets, die nicht im Zustand
      SYN-SENT sind
  \end{description}

  Falls weder ein \verb?state? noch ein \verb?exclude? Statement vorhanden
  ist, ist die Voreinstellung \verb?all? bei Option \verb?-a? und ansonsten
  alle außer \verb?listening?, \verb?syn-recv?, \verb?time-wait? und
  \verb?closed?.

  Mit dem (booleschen) Ausdruck kann ich nach Adressen und Ports filtern.

  Weitere Optionen und ausführlichere Informationen zur Filterung gibt es in
  der Handbuchseite und im Artikel ``SS Utility: Quick Intro'', der sich bei
  der Dokumentation des iproute-Pakets befindet.

  \subsection*{tc}

  Mit dem Programm tc (traffic control) kann ich die Einstellungen zur
  Verkehrssteuerung des Kernels ansehen und manipulieren.

  Dabei gilt es drei Arten von Objekten zu unterscheiden:
  \begin{description}
    \item[QDISC] (queueing discipline) beschreiben das
      Warteschlangenverhalten, das heißt, wie und in welcher Reihenfolge
      Datenpakete, die in eine QDISC eingereiht wurden, an den Treiber der
      Netzwerkkarte zum Senden übergeben werden. Wenn ein Datenpaket gesendet
      werden soll, wird es in die für das Interface konfigurierte QDISC
      eingereiht. Unmittelbar danach versucht der Kernel, so viele Pakete wie
      möglich an den Netzwerkadapter zu übergeben.
    \item[CLASS] (Klassen) können in QDISC enthalten sein und wiederum weitere
      QDISC enthalten. Datenpakete werden in den inneren QDISC eingereiht.
      Datenpakete, die an den Netzwerkadapter übergeben werden, können
      von jeder inneren QDISC kommen. Dadurch, dass bestimmte Klassen vor
      anderen drankommen, kann der Datenverkehr priorisiert werden.
    \item[FILTER] entscheiden, welcher Klasse ein Datenpaket bei einem
      klassenbasierten QDISC zugeordnet wird. Alle Filter einer Klasse werden
      aufgerufen, bis einer eine Entscheidung trifft.
  \end{description}

  \subsubsection*{Klassenlose QDISC}
  Die folgenden Klassenlosen QDISC stehen zur Verfügung:
  \begin{description}
    \item[fifo] ist die einfachste Form (First In, First Out). Die QDISC
      kann auf Paketebene (pfifo) oder Byteebene (bfifo) begrenzt werden.
    \item[pfifo\_fast] ist die Standard-QDISC für Advanced Router Kernel. Diese
      enthält eine dreireihige Warteschlange, die das TOS-Feld und die
      Priorität des Datenpakets beachten.
    \item[red] (Random Early Detection) simuliert eine überlastete Leitung,
      indem zufällig Datenpakete verworfen werden, wenn sich der Durchsatz der
      konfigurierten Bandbreite nähert.
    \item[sfq] (Stochastic Fairness Queueing) sortiert die wartenden
      Datenpakete um, so dass jede Sitzung reihum dran ist.
    \item[tbf] (Token Bucket Filter) ist geeignet, um den Traffic zu eineer
      präzise konfigurierten Bandbreite zu verlangsamen.
  \end{description}

  Klassenlose QDISC müssen an der Wurzel installiert werden:
  \begin{verbatim}
tc qdisc add dev $device root $disc $params
  \end{verbatim}

  \subsubsection*{Klassenbasierte QDISC}
  Die folgenden klassenbasierten QDISC stehen zur Verfügung:
  \begin{description}
    \item[cbq] (Class Based Queueing) bildet eine Hierarchie von Klassen, die
      sich einen Link teilen und kann sowohl priorisieren als auch den
      Durchsatz begrenzen.
    \item[htb] (Hierarchy Token Buffer) ermöglicht garantierte Bandbreiten für
      Klassen und erlaubt die Ausgabe von oberen Grenzen für das Teilen von
      Bandbreite zwischen Klassen. Es enthält begrenzende Elemente auf Basis
      von TBF und kann Klassen priorisieren.
    \item[prio] wird für Priorisierung ohne Begrenzung der Bandbreite
      verwendet.
  \end{description}

  \subsubsection*{Theorie}
  Die Klassen formen einen Baum, bei dem jede Klasse genau einen Vorfahren hat
  und mehrere Kinder haben kann.

  Manche QDISC erlauben zur Laufzeit Klassen hinzuzufügen (CBQ, HTB), andere
  nicht (PRIO). Erstere können beliebig viele (auch keine) Subklassen haben,
  in denen die Datenpakete einsortiert werden.

  Jede Klasse enthält genau ein Blatt-QDISC (per Default pfifo), welcher durch
  ein anderes ersetzt werden kann. Diese QDISC kann wiederum andere Klasse
  enthalten, die zunächst auch nur ein QDISC haben.

  Wenn ein Datenpaket in einer klassenbasierten QDISC ankommt, wird es genau
  einer der enthaltenen Klassen zugeordnet. Sind Filter für eine Klasse
  definiert, werden diese zuerst für die Klassifizierung herangezogen. Einige
  QDISC werten auch das TOS-Feld des IP-Headers aus.

  Jeder Knoten im Klassenbaum kann seine eigenen Filter haben. Filter in
  höheren Ebenen können direkt auf niedrigere Klassen verweisen.

  Wenn ein Paket nicht klassifiziert werden konnte, geht es in die
  Blatt-QDISC der Klasse.

  \subsubsection*{Namen}
  Alle QDISC, Klassen und Filter haben Ids, die automatisch bestimmt oder
  explizit spezifiziert werden. Diese Ids bestehen aus einer Haupt- und einer
  Nebennummer, getrennt durch Doppelpunkt.

  Ein QDISC, welche potentiell Kinder haben kann, bekommt eine Hauptnummer,
  die 'Handle' genannt wird und lässt die Nebennummer als Namensraum für die
  Klassen. Üblicherweise benennt man QDISC, die Kinder haben, explizit.

  Alle Klassen, die zur selben QDISC gehören, teilen sich die gleiche
  Hauptnummeer. Jede hat eine separate Nebennummer, die Class-Id genann wird
  un sich auf die QDISC (nicht die Elternklasse bezieht).

  Filter haben eine dreiteilige Filter-Id, die nur bei einer Filter-Hierarchie
  benötigt werden.

  \subsubsection*{tc Befehle}
  \begin{description}
    \item[add] Fügt eine QDISC, Klasse oder einen Filter an. Der Vorfahre
      (root oder Class-Id) muss angegeben werden. QDISC oder Filter können mit
      dem \verb?handle? Parameter benannt werden, Klassen mit \verb?classid?.
    \item[remove] entfernt eine QDISK
    \item[change] modifiziert eine Einheit am Orte
    \item[replace] gleichzeitiges \verb?remove? / \verb?add?, der neue Knoten
      wird gegebenenfalls neu erzeugt
    \item[link] gleichzeitiges \verb?remove? / \verb?add?, der neue Knoten
      muss bereits existieren.
  \end{description}

  Weitere Informationen finden sich in der Handbuchseite von \verb?tc?.
  \subsection*{lnstat / nstat / rtacct}
  Diese Programme liefern Netzwerkstatistiken. Nähere Informationen siehe
  Handbuchseiten.
\section{netstat}
\label{sec:netz-werkzeuge-netstat}
\begin{abstractsec}
  Mit netstat ist es möglich essentielle Informationen zur
  Netzwerkkonfiguration und zum aktuellen Zustand der Netzwerksockets und
  Verbindungen eines Rechners zu erfahren.
\end{abstractsec}
\begin{normaltext}
  Ein Werkzeug, dass ich auch bei der Analyse von lokalen Rechnerproblemen
  einsetze und dort bereits vorgestellt habe, ist netstat. Im Bereich
  Netzwerkprobleme kann es viele Informationen liefern, die mir, je nach
  Problemfall weiterhelfen können.

  \subsection*{Sockets}
  \label{sec:netz-werkzeuge-netstat-sockets}

  Rufe ich \verb?netstat? ohne Argumente auf, liefert es mir eine Liste der
  offenen und aktiven Sockets aller konfigurierten Adressfamilien, das
  heisst, der bestehenden Verbindungen.

  Meist interessieren mich nicht alle Adressfamilien, sondern nur ganz
  bestimmte. Dann kann ich diese zum Beispiel mit der Option
  \verb?--protocol=$familie? einschränken. Für \verb?$familie? kan ich in
  einer kommaseparierten Liste die folgenden angeben: \verb?unix?,
  \verb?inet?, \verb?ipx?, \verb?ax25?, \verb?netrom?, \verb?ddp?. Alternativ
  kann ich jeden gewünschten Familiennamen einzeln als Option übergeben:
  \verb?--unix?, \verb?--inet?, \ldots

  In diesem Teil der Buches interessiert mich vor allem die Familie
  \verb?--inet?. Diese kann ich weiter eingrenzen. Mit der Option \verb?-4?
  beziehungsweise \verb?-6? beschränke ich die Ausgabe auf die entsprechende
  Version des Internet Protokolls.

  Außerdem verwende ich
  \begin{description}
    \item[--raw oder -w] wenn ich an Raw-Sockets interessiert bin,
    \item[--tcp oder -t] für TCP-Sockets, und
    \item[--udp oder -u] für UDP-Sockets.
  \end{description}

  Bin ich nur daran interessiert, ob überhaupt ein Prozess an einem bestimmten
  Socket wartet, verwende ich die Option \verb?--listening? beziehungsweise
  \verb?-l?. Diese werden bei der normalen Ausgabe weggelassen. Will ich
  hingegen sowohl die aktiven als auch die lauschenden Sockets erfassen,
  verwende ich die Option \verb?--all? beziehungsweise \verb?-a?.

  \subsection*{Routen}
  \label{sec:netz-werkzeuge-netstat-routen}

  Wenn ich statt an den Sockets eher an den Routen interessiert bin, verwende
  ich die Option \verb?--routes? beziehungsweise \verb?-r?. Damit bekomme ich
  die gleiche Ausgabe, wei mit dem Befehl \verb?route -e?. Auch hier kann ich
  mit \verb?-4? oder \verb?-6? die Protokollversion einschränken.

  Füge ich die Option \verb?-C? hinzu, bekomme ich Informationen aus dem
  Routencache, mit der Option \verb?-F? stattdessen aus der Forwarding
  Information Base (der Routentabelle), aber das ist sowieso die
  Voreinstellung.

  \subsection*{Interfaces}
  \label{sec:netz-werkzeuge-netstat-interfaces}

  Mit der Option \verb?--interfaces? oder \verb?-i? kann ich Informationen
  über die Netzwerk-Interfaces bekommen.

  Ein einfaches \verb?netstat -i? liefert mir in einer übersichtlichen Tabelle
  zu jedem aktiven Interface unter anderem die MTU, die Anzahl der gesendeten
  und empfangenen Datenpakete sowie die Anzahl der Sende- beziehungweise
  Empfangsfehler.

  Kombiniere ich das mit \verb?-e?, bekomme ich die gleiche Ausgabe wie vom
  Program \verb?ifconfig?. Kombiniert mit \verb?-a? werden auch Interfaces
  angezeigt, die nicht im Zustand \verb?UP? sind.

  \subsection*{Multicast-Gruppen}
  \label{sec:netz-werkzeuge-netstat-groups}
  Die Option \verb?--groups? beziehungsweise \verb?-g? liefert mir
  Informationen zur Mitgliedschaft des Rechners in Multicast-Gruppen.

  Auch diese kann ich mit \verb?-4? oder \verb?-6? einschränken.

  \subsection*{Statistiken}
  \label{sec:netz-werkzeuge-netstat-statistics}

  Mit der Option \verb?--statistics? beziehungsweise \verb?-s? zeigt netstat
  zusammengefasste Statistiken für alle Protokolle.

  \subsection*{allgemeine Optionen}
  \label{sec:netz-werkzeuge-netstat-allgemein}

  Abschließen möchte ich diese kleine Vorstellung von netstat mit ein paar
  allgemeinen Optionen, mit denen ich die Ausgabe modifizieren kann.

  Die von mir wohl am meisten eingesetzte Option ist \verb?--numeric?, kurz
  \verb?-n?. Mit dieser Option zeigt netstat numerische statt symbolischer
  Informationen an und das ist insbesondere bei Netzwerkadressen ein immenser
  Geschwindigkeitsvorteil, da sonst unter Umständen etliche DNS-Anfragen mit
  den entsprechenden Verzögerungen gestellt werden, bevor die Ausgabe
  angezeigt werden kann. Natürlich kann man das auch selektiv einstellen mit
  \verb?--numeric-hosts?, \verb?--numeric-ports? und \verb?--numeric-users?.

  Mit der Option \verb?--verbose? oder \verb?-v? kann ich mehr Informationen
  bekommen, insbesondere zu nicht konfigurierten Adressfamilien.

  Ähnliches bietet die Option \verb?--extend? oder \verb?-e?, die zusätzliche
  Informationen zum Beispiel bei Interfaces liefert.

  Gebe ich die Option \verb?--continuos? oder \verb?-c? an, bekomme ich die
  Informationen aller Sekunde ausgegeben.
\end{normaltext}

\section{perl}
\label{sec:lokal-werkzeuge-perl}
\begin{abstractsec}
  Für knifflige Probleme, die ich mit den spezialisierten Werkzeugen nicht zu
  fassen kriege und denen mit einfacher Shell-Programmierung auch nicht
  beizukommen ist, benötige ich eine Programmiersprache, die mächtiger als die
  Shell ist, mit der ich aber trotzdem mit wenig Aufwand ein passendes
  Programm schreiben kann. Insbesondere durch die vielen verfügbaren Module
  auf CPAN kann ich damit relativ schnell eine Speziallösung für vertrackte
  Probleme zusammenbauen.
\end{abstractsec}
\begin{normaltext}
  Perl hatte ich als Werkzeug bereits im Abschnitt über die Werkzeuge zur
  lokalen Fehlersuche beschrieben. Durch die vielen einfach verfügbaren und
  meist sehr gut getesteten und dokumentierten Module auf CPAN ist Perl auch
  ein unentbehrliches Werkzeug für die Fehlersuche bei Netzwerkproblemen.

  Das Perl Kochbuch \cite{ChristiansenTorkington04de} hatte ich bereits
  erwähnt. Mit dessen Hilfe und den darin beschriebenen Modulen von CPAN war
  es mir zum Beispiel möglich ein Testprogramm für ein Timing-Problem bei
  einem Webservice zu schreiben.

  \subsection*{HTTP Injector}

  Vor einiger Zeit hatte ich ein Problem, bei dem 502-Fehler von
  einem Webservice abhängig waren von der Zeit für die
  Anfrage. Der Betreiber des Webservices stritt das ab und um das
  Problem zu verifizieren benötigte ich die Möglichkeit HTTP-Anfragen gezielt
  zu verzögern.
  
  Ich kam mit Hilfe des Kochbuches zu folgendem Programm:
  \begin{verbatim}
01:#!/usr/bin/perl
02:use Getopt::Long;
03:use IO::Socket;
04:use Time::HiRes qw(sleep);
05:
06:my %opt = ( delay => 0 );
07:
08:GetOptions( \%opt, 'delay=i');
09:
10:my $server = shift;
11:my $port   = shift || 80;
12:
13:my $socket = IO::Socket::INET->new(PeerAddr => $server,
14:                                   PeerPort => $port,
15:                                   Proto    => 'tcp',
16:                                   Type     => SOCK_STREAM);
17:
18:my @in = <>;
19:my $del = $opt{delay} / ( 1.0 + scalar @in );
20:foreach (@in) {
21:    s/[\r\n]+$//;
22:    sleep $del;
23:    print $socket $_, "\r\n";
24:}
25:sleep $del;
26:print $socket "\r\n";
27:
28:while (my $line = <$socket>) {
29:    print $line;
30:}
  \end{verbatim}
  In den Zeilen 2-4 lade ich die benötigten Module. \verb?Getopt::Long? ist
  für die Verarbeitung der Kommandozeilenoptionen und sichert ab, dass ich mit
  \verb?--delay? einen Integerwert angebe. \verb?IO::Socket? stellt die
  Socketfunktionalität bereit, so dass ich diesen Socket wie eine Datei
  verwenden kann. \verb?Time::HiRes? stellt mir eine verbesserte
  \verb?sleep()? Funktion bereit, die mit Gleitkommazahlen zurechtkommt.

  In Zeile 6 stelle ich die Option \verb?--delay? auf den Wert 0 ein, falls
  sie nicht explizit angegeben wird. In Zeile 8 werden die Optionen
  eingelesen.

  Zeile 10 und 11 entnehmen den Server und gegebenenfalls den Port der
  Kommandozeile und in Zeile 13 öffne ich mit diesen Angaben den Socket.

  In Zeile 18 lese ich die gesamte Eingabe in ein Array ein. Dies benötige
  ich, da ich die Anzahl der Zeilen wissen muss, denn ich verzögere das Senden
  zeilenweise um jeweils einen Bruchteil der Gesamtverzögerung. Die Zeilen
  20-25 schließlich bereiten die Zeilenenden auf und senden die modifizierten
  Zeilen verzögert über den Socket. Zeile 26 schickt die Leerzeile, nach der
  der Server antwortet.

  In Zeile 28-30 liest das Skript die Antwort des Servers vom Socket und
  schreibt sie zur Standardausgabe.

  Dieses Skript kann ich nun wie folgt aufrufen:
  \begin{verbatim}
time ./http-injector.pl --delay 5 localhost 80 < request > reply

real  0m5.072s
user  0m0.056s
sys   0m0.012s
  \end{verbatim}
  Dabei steht in der Datei request die HTTP-Anfrage, die ich an den Server
  sende.
  Nach fünf Sekunden ist die Anfrage beim Server, und die Antwort landet in
  der Datei reply.

  Damit konnte ich nachweisen, dass dieselbe Anfrage einen Fehler
  lieferte, wenn sie mehr als drei Sekunden zur Übertragung brauchte und
  fehlerfrei beantwortet wurde, wenn sie weniger als drei Sekunden brauchte.
\end{normaltext}

%%% Local Variables: 
%%% mode: latex
%%% TeX-master: "troubleshoot-linux"
%%% End: 
%%% vim: set sw=2 ts=2 tw=78 et si:

%% tl-netz-totalausfall.tex
\chapter{Totalausfall des Netzes}
\label{cha:netz-totalausfall}

\begin{abstractsec}
  Habe ich an einem Rechner überhaupt keine Netzverbindung, oder kann ich ein
  ganzes Netzsegment nicht erreichen, spreche ich von einem Totalausfall.
\end{abstractsec}

\section{Ein Rechner hat überhaupt keine Netzverbindung}
\label{sec:gar-kein-netz}

\begin{abstractsec}
  Dieser Abschnitt beschäftigt sich mit dem Totalausfall des Netzwerks an
  einem Rechner.
\end{abstractsec}
\begin{normaltext}
  Manchmal kommt es vor, dass ein Rechner überhaupt keine Netzwerkverbindung
  hat. Dann bleibt nichts übrig, als dem Problem methodisch auf den Grund zu
  gehen. Dazu verwende ich den folgenden Entscheidungsbaum, um das Problem in
  Teilprobleme zu zerlegen:

\begin{tikzpicture}
[every node/.style={fill=black!10,rounded corners,align=center},
grow=south, level distance=2cm,
level 1/.style={sibling distance=3cm},
level 2/.style={sibling distance=3cm},
]

\node{Funktioniert die Bitübertragungsschicht?}
  child{node[draw,fill=white]{Hardwareproblem}
  edge from parent node[fill=black!10]{nein}}
  child{node at +(1,-1) {Hat der Rechner korrekte IP-Adressen?}
    child{node[draw,fill=white]{Schnittstellenkonfiguration}
    edge from parent node{nein}}
    child{node at +(1,-1) {Funktionieren einfache Dienste?}
      child{node[draw,fill=white]{Paketfilterregeln}
      edge from parent node{nein}}
      child{node at +(1,-1) {Kein Totalausfall}
      edge from parent node{ja}}
    edge from parent node{ja}}
  edge from parent node{ja}};
\end{tikzpicture}

  Die allererste Frage geht dabei nach der physischen Netzverbindung, im
  OSI-Modell auch Bitübertragungsschicht genannt. Bei Hardware-Rechnern mit
  einigermaßen neuer Netzwerkkarte helfen mir die mii-tools beziehungsweise
  eth-tools, die Auskunft über die Verbindungsparameter zum Switch geben
  können. Habe ich es mit einer virtuellen Maschine zu tun, dann muss ich am
  Hostsystem die Netzwerkeinstellungen für diese VM kontrollieren.

  Auch das Programm ifconfig zeigt an, ob das betreffende Interface eine
  physikalische Verbindung hat (\verb?RUNNING?). Noch deutlicher zeigt  es
  \verb?ip link show? beziehungsweise \verb?ip addr show? (\verb?LOWER_UP?
  oder \verb?NO-CARRIER?).

  Habe ich diese Programme nicht zur Verfügung, kann ich mich vielleicht mit
  \verb?netstat -i? behelfen. Dieser Befehl zeigt eine Statistik der
  gesendeten und empfangenen Pakete an. Da in jedem Netzwerk ein gewisser
  Anteil an Broadcastpaketen unterwegs ist, sollte sich die Anzahl der
  empfangenen Datenpakete erhöhen, wenn ich den Befehl im Abstand von einigen
  Sekunden aufrufe. Um sicher zu gehen, kann ich auf einem anderen Rechner im
  gleichen Netzsegment zum Beispiel mit ping Broadcastpakete erzeugen.

  Wenn tcpdump installiert ist, kann ich auch mit
  \verb?tcpdump -n -i <schnittstelle>? auf Traffic warten. Wenn ich damit
  Traffic sehe, beende ich das Programm nicht gleich, sondern warte, bis ich
  an Hand der Datenpakete entscheiden kann, ob die Schnittstelle im richtigen
  Segment angeschlossen ist.

  Kann ich keinen Traffic nachweisen, muss ich mir die Verkabelung ansehen,
  probeweise die Kabel tauschen, den Rechner an einen anderen Switch
  anschließen und/oder in ein anderes Netzsegment. Bei virtuellen Maschinen
  schaue ich auf dem Host nach, ob andere VM ähnliche Probleme haben, ob
  vielleicht ein falscher Netzwerkanschluß zugewiesen wurde oder kontrolliere
  die entsprechenden virtuellen Interfaces auf dem Host.

  Ein kurzer Test mit \verb?iptables -L? und \verb?ebtables -L? sagt mir, ob
  vielleicht Paketfilterregeln die Verbindung unterdrücken.

  Erst wenn die Bitübertragungsschicht funktioniert, hat es überhaupt Sinn,
  sich dem nächsten Problem, der IP-Konfiguration zuzuwenden. Diese muss zum
  angeschlossenen Netzsegment passen. Eventuell ist es besser, für die Dauer
  der Fehlersuche, alle Regeln zu entfernen und die Policies auf \verb?ACCEPT?
  zu stellen.

  Habe ich vorher die Funktionalität der Schnittstelle mit tcpdump
  kontrolliert, dann habe ich eventuell schon genügend Pakete gesehen, die mir
  zeigen, ob die Schnittstelle im richtigen Segment angeschlossen ist.
  Ansonsten kann ich das jetzt nachholen.

  Habe ich tcpdump nicht an Board, weiß aber, dass DHCP im Netzwerk zur
  Verfügung steht, kann ich versuchen darüber eine IP-Adresse zu bekommen.
  Dazu stelle ich temporär die Schnittstellenkonfiguration um. Falls ich es
  noch nicht getan habe, deaktiviere ich jetzt alle Paketfilter.

\begin{Exkursbox}{Konfiguration der Netzwerkschnittstelle}
  Bei Debian: /etc/network/interfaces

  Bei Redhat: /etc/sysconfig/network-scripts/*
\end{Exkursbox}

  Kann ich DHCP nicht nutzen, so könnte ich noch versuchen mit Zero
  Configuration Networking zu anderen Rechnern in meinem Segment zu bekommen.
  Das hängt davon ab, wie das betreffende Netzwerk administriert ist.

  Habe ich nicht die Möglichkeit für Tests mit DHCP/Zero Configuration
  Networking, kann ich versuchen mit Ping bekannte und aktive Stationen in
  meinem Netzsegment zu erreichen. Antworten die Stationen nicht, aber ihre
  IP-Adressen tauchen korrekt im ARP-Cache auf (Test mit \verb?arp -an?), dann
  liegt das Problem eher an zu restriktiven Paketfiltereinstellungen der
  anderen Rechner und ich würde diese dort testweise lockern beziehungsweise
  auf den anderen Rechnern mit tcpdump oder Wireshark nachschauen, ob ich
  Datenpakete des problematischen Rechners sehen kann. Bei VMs kann ich mit
  tcpdump am zugehörigen virtuellen Interface des Hosts nachschauen.

\begin{Exkursbox}{Sperren von ICMP durch Paketfilter}
  Verweis auf übermäßiges Beschränken von ICMP-Verkehr und die Auswirkungen
  auf den Netzverkehr.
\end{Exkursbox}

Habe ich mich davon überzeugt, dass der Rechner im richtigen Netzsegment
angeschlossen ist und Stationen in diesem erreichen kann, muss ich mich nur
noch vergewissern, dass er Rechner in anderen Netzen erreichen kann.

Bei der Überprüfung der Schnittstellenkonfiguration, habe ich mich gleich mit
vergewissert, dass der Rechner das richtige Gateway kennt. Ich überzeuge mich
davon, dass die konfigurierte Konfiguration auch aktiv ist. Insbesondere
überprüfe ich:

\begin{itemize}
  \item Stimmen IP-Adresse und Netzmaske? (\verb?ip addr show?)
  \item Stimmen Gateway und Routen? (\verb?ip route show?)
  \item Kann ich das Gateway erreichen? (\verb?ping $gateway?)
\end{itemize}

\begin{Exkursbox}{Netzmasken}
  Hier in Tabellenform oder ähnlich eine Übersicht zu IPv4-Netzmasken.
\end{Exkursbox}

Anschließend versuche ich mit ping einen Rechner in einem anderen
Netzsegment zu erreichen, der zum Beispiel von anderen Stationen
aus diesem Segment erreicht werden kann. Kann ich trotz korrekter
IP-Konfiguration und deaktiviertem Paketfilter auf dem Problemrechner keine
Rechner in anderen Segmenten erreichen, muss ich auf dem Gateway nachsehen, ob
die Datenpakete des Problemrechners dort vorbeikommen. Sehe ich diese am
Gateway, kann ich den Problemrechner bei der Fehlersuche hinter mir lassen,
vielleicht noch einen Dauerping einschalten und muss mich bei der Problemsuche
dem Netz zuwenden.

\end{normaltext}

%\subsection{Seiteneffekte}
%\label{sec:seiteneffekte}

%\subsection{Einschränkung der Umsetzung}
%\label{sec:einschr-der-umsetz}

%\begin{notes}
%\item Was nicht geht
%\item geht auch nicht
%\item und auch das
%\end{notes}

\section{Ein oder mehrere Netzsegmente sind nicht erreichbar}
\label{sec:ausfall-netzsegment}

\begin{normaltext}
  Beim Ausfall eines oder mehrerer Netzsegmente unterscheide ich zunächst, ob
  ich mich in einem der betroffenden Segmente befinde, oder nicht.

  Bin ich direkt in dem betroffenen Netzsegment - zum Beispiel, wenn der
  Ausfall mir von außerhalb gemeldet wird - untersuche ich zunächst, ob ich
  innerhalb des Netzsegmentes, vor allem zum Gateway Verbindung habe.
  Dabei kann ich gegebenenfalls auf das Vorgehen aus
  \ref{sec:gar-kein-netz} zurückgreifen.
  Anschließend versuche ich Kontakt zu dem Netzsegment zu bekommen, aus dem
  die Meldung kam. Bekomme ich keinen Kontakt, vermute ich ein Routingproblem.

  Bin ich außerhalb der betroffenen Netzsegmente, schaue ich als erstes, ob
  die Routen dorthin in Ordnung sind. Dabei kann mir das Programm traceroute
  helfen. Genauere Auskunft bekomme ich auf den Gateways oder anderen
  Rechnern, die an den Routingprotokollen teilnehmen. Fehlen die Routen, oder
  weisen sie in die falsche Richtung, muss ich das Routingproblem untersuchen.
  Sind die Routen in Ordnung, aber ich bekomme keine Verbindung, versuche ich
  vom Gateway des betreffenden Netzsegmentes eine Verbindung zu den Rechnern
  zu bekommen. Ist das möglich, dann handelt es sich möglicherweise um ein
  Konfigurationsproblem der betreffenden Rechner - dieses kann auch durch
  fehlerhafte Angaben vom DHCP-Server kommen.
\end{normaltext}

\begin{notes}
\item Entscheidungsbaum!
\end{notes}

\subsection{Routingprobleme}
\label{sec:routingprobleme}

\begin{notes}
\item Routingprotokolle
\item OSPF: BDR hatte falschen Neighbor (openbsd, GeNUScreen)
\item Routersoftware quagga
\end{notes}

\subsection{Rückroute fehlt}
\label{sec:rueckroute-fehlt}

\begin{notes}
\item Problem: Routen scheinen zu stimmen, keine Antwortpakete
\item Analyse: traceroute, analyse des letzten Hops und desjenigen danach
\item Ursache: Rückroute fehlt
\end{notes}

%\subsection{Schnittstelle zu B}
%\label{sec:schnittstelle-zu-b}

%\begin{notes}
%\item Syntax
%\item Semantik
%\end{notes}


%%% Local Variables: 
%%% mode: latex
%%% TeX-master: "arbeit-hauptdatei"
%%% End: 
%%% vim: set sw=2 ts=2 tw=78 et si:

%% tl-herangehen.tex
\chapter{Partieller Ausfall im Netz}
\label{cha:netz-teilausfall}

\begin{abstractsec}
  Wenn nur ein oder mehrere Dienste eines Servers nicht funktionieren, oder
  ein Server in einem ansonsten gut erreichbaren Netzsegment, handelt es sich
  um einen partiellen Ausfall im Netz.
\end{abstractsec}

%\section{Konzept der Umsetzung}
%\label{sec:konz-der-umsetz}

%\begin{abstractsec}
%  Die Aufteilung der Umsetzung wird hier gegliedert in verschiedene Aspekte
%\end{abstractsec}
%\begin{normaltext}
%  Blablabla
%\end{normaltext}

%\subsection{Seiteneffekte}
%\label{sec:seiteneffekte}

%\subsection{Einschr�nkung der Umsetzung}
%\label{sec:einschr-der-umsetz}

%\begin{notes}
%\item Was nicht geht
%\item geht auch nicht
%\item und auch das
%\end{notes}

%\section{Schnittstellen nach au�en}
%\label{sec:schn-nach-au3en}

%\subsection{Schnittstelle zu A}
%\label{sec:schnittstelle-zu}

%\begin{notes}
%\item Syntax
%\item Semantik
%\end{notes}

%\subsection{Schnittstelle zu B}
%\label{sec:schnittstelle-zu-b}

%\begin{notes}
%\item Syntax
%\item Semantik
%\end{notes}


%%% Local Variables: 
%%% mode: latex
%%% TeX-master: "arbeit-hauptdatei"
%%% End: 
%%% vim: set sw=2 ts=2 tw=78 et si:

%% tl-netz-performance.tex
\chapter{Netzwerkperformance}
\label{cha:netz-performance}

\begin{abstractsec}
  Manchmal scheint alles in Ordnung zu sein und trotzdem sind die Kunden nicht
  zufrieden. Performance ist ein heikles Thema, weil jeder seine eigene
  Vorstellung davon hat, was ausreichende Performance ist.
\end{abstractsec}

\begin{notes}
\item Baseline, ein Ausgangspunkt, der gemessen werden kann und als Referenz
  dient.
\item Bufferbloat
\item Netalyzr
\end{notes}

\section{Netzwerkbandbreite}
\label{sec:netz-performance-bandbreite}
\begin{abstractsec}
  Die Bandbreite einzelner Netzsegmente bestimme ich normalerweise nicht erst
  im Fehlerfall, sondern vorher, am Besten gleich nach Inbetriebnahme eines
  Abschnitts. Dann habe ich bei Problemen einen Referenzwert, der mir bei der
  Eingrenzung des Problems nützlich sein kann.
\end{abstractsec}
\begin{normaltext}
  Die Bandbreite einzelner Netzsegmente bestimme ich normalerweise nicht erst
  im Fehlerfall, sondern vorher, am Besten gleich nach Inbetriebnahme eines
  Abschnitts. Dann habe ich bei Problemen einen Referenzwert, der mir bei der
  Eingrenzung des Problems nützlich sein kann.

  \subsection{Bestimmung der Bandbreite mit ping}
  Wie man mit Ping die Bandbreite bestimmen kann, ist sehr gut in
  \cite{sloan2001} Kapitel 4. Path Characteristics beschrieben.

  Das geht für beliebige Streckenabschnitte einer Verbindung und zwar wie
  folgt:
  \begin{enumerate}
    \item Mit Ping bestimmt man die RTT zum vorderen und hinteren Ende des
      Netzsegments. Die Differenzen zwischen den beiden Zeiten eliminieren die
      Einflüsse der anderen Netzkomponenten.
    \item Jetzt bestimmt man die RTT zu den beiden Enden mit größeren
      Datenpaketen und bestimmt wieder die Differenz.
    \item Die Differenz zwischen den beiden Zeitdifferenzen aus den ersten
      beiden Schritten ist die Zeit, die für die zusätzlichen Daten benötigt
      wird.
    \item Die Bitrate ist 16 mal der Differenz der Paketgrößen geteilt
      durch die Differenzzeit aus Schritt 3 (die magische Zahl 16 kommt daher,
      dass wir mit 8 Bit pro Byte rechnen und die die Messungen die doppelte
      Zeit, nämlich für Hin- und Rückweg, enthalten, aber nur die einfache
      Bitrate bestimmen wollen)
  \end{enumerate}
  Wenn ich mehrere Pingpakete pro Einzelmessung sende, dann verwende ich
  jeweils die geringste gemessene RTT, da diese vermutlich die geringsten
  Störeinflüsse enthält.
\end{normaltext}

\section{Lasttests}
\label{sec:netz-performance-lasttests}
\begin{abstractsec}
  Die reine Kenntnis von Bandbreite und Latenz von Netzwerken reicht nicht
  aus, um das Verhalten unter realen Bedingungen zu Beschreiben. Insbesondere
  durch Pufferung von Datenpaketen, die nicht gleich gesendet werden können,
  ändert sich die Latenz einer Übertragungsstrecke erheblich. Eine
  Möglichkeit, derartige Probleme zu diagnostizieren sind Lasttests.
\end{abstractsec}
\begin{normaltext}
  Zusätzlich zu den Latenzzeiten, die durch die Datenübertragung und das
  reine Umkopieren in den Routern und Switches entstehen, gibt es
  Verzögerungen, die durch die Pufferung von Datenpaketen verursacht werden.
  Diese entstehen dadurch, dass ein Netzwerkgerät auf der Sendeseite die Daten
  nicht so schnell los wird, wie sie auf der Empfangsseite ankommen. Das
  passiert meist beim Übergang von schnellen auf langsamere Medien, aber auch,
  wenn Datenpakete aus mehreren Richtungen ankommen und in die selbe Richtung
  abgehen. Tritt dieser Effekt nur kurzzeitig auf, dann wirken die Puffer
  positiv, da keines der Datenpaket verloren geht, sondern nur etwas später
  ankommt. Kommen jedoch ständig mehr Daten an, als gesendet werden können,
  dann werden erst die Puffer gefüllt, bevor Daten verworfen werden. Ohne
  aktives Puffermangement hat dann jedes ankommende Datenpaket so viele andere
  Datenpakete vor sich, wie in den Puffer passen. Die durch den Puffer
  verursachte Latenz beträgt dann Puffergröße geteilt durch Sendebandbreite.
  Leider verhindert die durch den Puffer erhöhte Latenzzeit, dass sich das
  TCP-Protokoll an die gerade mögliche Bandbreite anpassen kann. Die
  wirksamste  Abhilfe ist Adaptive Queue Management, ein Notbehelf kann
  Trafficshaping sein, wobei dieses den Nachteil hat, dass man als
  Netzadministrator die Sendebandbreite kennen muss um den Traffic manuell auf
  die entsprechende Rate zu begrenzen.

  Die reine Kenntnis von Bandbreite und Latenz von Netzwerken reicht also nicht
  aus, um das Verhalten unter realen Bedingungen zu Beschreiben. Insbesondere
  durch Pufferung von Datenpaketen, die nicht gleich gesendet werden können,
  ändert sich die Latenz einer Übertragungsstrecke erheblich. Eine
  Möglichkeit, derartige Probleme zu diagnostizieren sind Lasttests.

  Das heisst, ich erzeuge künstlichen einen starken Datenverkehr und
  beobachte, wie sich die anderen Netzparameter verhalten. Das betrifft zum
  einen die Latenz und zum anderen die verfügbare Bandbreite.
  \subsection{Lasttest mit Ping}
  Mit dem Befehl
  \begin{verbatim}
# ping -f rechnername
  \end{verbatim}
  sendet Ping Datenpakete so schnell es geht zum Zielrechner. Dazu benötige
  ich Superuserrechte.

  Diesen Aufruf verwende ich, um auf einem Segment Netzwerklast zu erzeugen.
  Zusätzlich zur normalen Statistik (min/avg/max/mdev) zeigt Ping am Ende zwei
  Werte: IPG und EWMA.

  IPG (Inter Packet Gap) ist die Zeit zwischen dem Senden zweier Datenpakete.
  Für Ethernet ist die Minimalzeit auf die Zeit festgelegt, in der 96 Bit
  übertragen werden. Das sind 9,6 µs für 10 MBit/s Ethernet und 9,6 ns für 10
  GBit/s. Diese Zeit wird automatisch vom Ethernetadapter an jedes Datenpaket
  angehängt. Das angezeigte IPG kann ich als Maß verwenden, um abzuschätzen,
  wie effizient die Kombination Betriebssystem, Netzwerkkarte, Netzwerk Daten
  senden kann.

  EWMA steht für Exponential Weighted Moving Average. Bei diesem Durchschnitt
  werden die letzten  RTT-Zeiten höher gewichtet als ältere. Im Normalfall
  sollte dieser gleich dem Mittelwert für RTT sein. Weicht er signifikant ab,
  deutet das auf einen Trend hin. Dafür benötigt man aber einen länger
  laufenden Ping und da EWMA am Ende ausgegeben wird, wird der Trend erst am
  Ende dieser Messung offenbar.
\end{normaltext}


%\begin{abstractsec}
%  Die Aufteilung der Umsetzung wird hier gegliedert in verschiedene Aspekte
%\end{abstractsec}
%\begin{normaltext}
%  Blablabla
%\end{normaltext}

%\section{Schnittstellen nach außen}
%\label{sec:schn-nach-au3en}

%\subsection{Schnittstelle zu A}
%\label{sec:schnittstelle-zu}

%\begin{notes}
%\item Syntax
%\item Semantik
%\end{notes}

%\subsection{Schnittstelle zu B}
%\label{sec:schnittstelle-zu-b}

%\begin{notes}
%\item Syntax
%\item Semantik
%\end{notes}


%%% Local Variables: 
%%% mode: latex
%%% TeX-master: "arbeit-hauptdatei"
%%% End: 
%%% vim: set sw=2 ts=2 tw=78 et si:


%\part{Vorbereitung auf das nächste Problem}
%%% tl-part4.tex
\part{Vorbereitung auf das nächste Problem}

%%% Local Variables: 
%%% mode: latex
%%% TeX-master: "troubleshoot-linux"
%%% End: 
%%% vim: set sw=2 ts=2 tw=78 et si:

%% tl-ausblick.tex
\chapter{Nach der Fehlersuche ist vor der Fehlersuche}
\label{cha:ausblick}

\begin{abstractsec}
  Nach dem letzten Fehler wird es einen nächsten geben. Was kann ich tun, um
  zu vermeiden, dass es noch einmal den selben Fehler geben wird.
\end{abstractsec}

\begin{normaltext}
  Es heißt, Voraussagen sind schwierig, besonders für die Zukunft. Für die
  Vergangenheit ist es einfacher.
  Manchen Leuten scheint es gegeben, häufiger als andere richtige Voraussagen
  zu machen und damit Probleme schneller auf die Schliche zu kommen.
  Beobachtet man diese Leute genauer, stellt man vielleicht fest, das sie
  keine Voraussagen über die Zukunft machen, sondern über die Vergangenheit.
  Der Volksmund nennt dieses Phänomen auch Erfahrung.
\end{normaltext}

\section{Was habe ich gelernt}
\label{sec:was-habe-ich-gelernt}

\begin{notes}
\item fünf mal warum
\end{notes}

\section{Monitoring}
\label{sec:monitoring}

\begin{notes}
\item Kann ich den Fehler durch Monitoring vermeiden?
\end{notes}

%%% Local Variables: 
%%% mode: latex
%%% TeX-master: "arbeit-hauptdatei"
%%% End: 
%%% vim: set sw=2 ts=2 tw=78 et si:


\bibliographystyle{babalpha}
\bibliography{literatur}

\end{document}

%%% Local Variables: 
%%% mode: latex
%%% TeX-master: t
%%% End: 
