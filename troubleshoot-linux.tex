%% Hauptdatei für das Buch Troubleshooting Linux Servers
%%
\documentclass[%
%draft,                        %% um später den Zeilenumbruch zu
%kontrollieren
%a4paper,                       %% DIN A4-Papier
a5paper,                       %% DIN A5-Papier
BCOR4mm,                       %% Bindekorrektur 4 mm
DIVcalc,                       %% Satzspiegel berechnen
11pt,                          %% Schriftgröße
bibtotoc,                      %% Literaturverzeichnis ins Inhaltsverzeichnis
idxtotoc,                      %% Index ins Inhaltsverzeichnis
tablecaptionabove,             %% Tabellenüber``= statt unterschriften
]{scrbook}                     %% KOMA-Skript Book als Klasse

\usepackage{versions}          %% bedingte Kompilierung
\usepackage{ifthen}            %% für \ifthenelse
\usepackage[utf8]{inputenc}    %% UTF8 muss sein

%% bedingte Kompilierung via Makefile
\newcommand{\enableversions}[4]{
  \ifthenelse{\equal{#1}{on}}{
    \newenvironment{abstractsec}{\begin{quotation}\noindent\textbf{Zusammenfassung:}}{\end{quotation}}
  }{\excludeversion{abstractsec}}
  \ifthenelse{\equal{#2}{on}}{\includeversion{normaltext}
  }{\excludeversion{normaltext}}
  \ifthenelse{\equal{#3}{on}}{
    \newenvironment{notes}{\par\hrulefill{}\emph{Ungeordnete partielle
        Ideensammlung (nicht im Druck):}\hrulefill{}\begin{itemize*}}{\end{itemize*}}
  }{\excludeversion{notes}}
  \ifthenelse{\equal{#4}{on}}{\def\svnfinal{final}\includeversion{finalversion}
  }{\def\svnfinal{}\excludeversion{finalversion}}
}
\newcommand{\versionabstracts}{\enableversions{on}{off}{off}{off}}
\newcommand{\versionnotes}{\enableversions{off}{off}{on}{off}}
\newcommand{\versionabstractstext}{\enableversions{on}{on}{off}{off}}
\newcommand{\versiontext}{\enableversions{off}{on}{off}{off}}
\newcommand{\versiontextnotes}{\enableversions{off}{on}{on}{off}}
\newcommand{\versionfinal}{\enableversions{off}{on}{off}{on}}

%% Wählen, was erzeugt werden soll:
\providecommand{\selectversion}{
%\versionabstracts
%\versionnotes
%\versionabstractstext
%\versiontext
%\versiontextnotes
\versionfinal
}

\AtBeginDocument{%
%  \selectlanguage{ngerman}%
%  \renewcommand*{\freffigname}{\figurename}
%  \renewcommand*{\freftabname}{\tablename}
  \selectversion{}
}

%% Hier beginnt das eigentliche Dokument
\begin{document}

%% Die Eckdaten des Dokuments, wegen Babel erst nach \begin{document}
\author{Mathias Weidner}
\title{Troubleshooting Linux Servers}

\maketitle

%% tl-vorwort.tex
%% vim: set sw=2 ts=2 tw=78 et si:
\chapter{Vorwort}
\label{cha:vorwort}

\begin{abstractsec}
  Hier wird die Einf�hrung stehen.
\end{abstractsec}
\begin{normaltext}
  In diesem Buch lege ich meine �ber die Jahre gesammelten Erfahrungen mit
  der Fehlersuche bei Linux-Servern und in IP-Netzwerken nieder.

  Ich habe diese beiden Themen kombiniert, weil zum Einen Linux-Server fast
  immer �ber IP-Netzwerke angesprochen werden und daher auf ein gut
  funktionierendes Netzwerk angewiesen sind und zum Anderen Linux-Rechner
  (da gehen auch Arbeitsstationen) f�r mich ideal zur Analyse von
  Netzwerkproblemen geeignet sind.

  Zwar benutze ich fast t�glich Linux auf dem Desktop, allerdings sind hier
  die am h�ufigsten verwendeten Programme ein Terminal und ein Webbrowser, die
  in den meisten F�llen vorz�glich funktionieren, so dass ich kaum Gelegenheit
  hatte Erfahrungen mit der Fehlersuche am Linux-Desktop zu sammeln. Aus
  diesem Grund bleibt dieser Bereich aussen vor.
\end{normaltext}
\begin{notes}
\item Motivation
\item Aufbau
\item Danksagung
\end{notes}

\section*{F�r wen ist dieses Buch}
\label{sec:fuerwen}

\begin{normaltext}
\end{normaltext}

\section*{�bersicht}
\label{sec:ubersicht}

\begin{normaltext}
  Dieses Buch ist in vier Teile gegliedert:
  \begin{itemize}
  \item Teil 1 besch�ftigt sich mit dem grundlegenden Vorgehen bei der
    Fehleranalyse.
  \item In Teil 1 geht es vorwiegend um lokale Probleme an einem
    Linux-Rechner.
  \item Teil 3 befasst sich mit Netzwerkproblemen.
  \item In Teil 4 geht es um die Nachbereitung eines gel�sten und die
    Vorbereitung auf das n�chste Problems.
  \end{itemize}
\end{normaltext}

\section*{Danksagung}
\label{sec:problemstellung}

\begin{abstractsec}
  Wer alles geholfen hat.
\end{abstractsec}
\begin{normaltext}
  Diese haben alle geholfen.
\end{normaltext}

%%% Local Variables: 
%%% mode: latex
%%% TeX-master: "troubleshoot-linux"
%%% End: 

%% allgemeiner Teil
\include{tl-wie-vorgehen}
\include{tl-dokumentation}
%% lokale Probleme
\include{tl-klassifikation-lokal}
%% tl-lokal-werkzeuge.tex
\chapter{Werkzeuge}
\label{cha:lokal-werkzeuge}

\begin{abstractsec}
  Verschiedene Werkzeuge helfen mir lokale Probleme einzugrenzen. Hier stelle
  ich die Werkzeuge kurz vor, die ich in den n�chsten drei Kapiteln zur
  Fehlersuche einsetze.
\end{abstractsec}

\begin{normaltext}
  Linux stellt mir eine Unmenge von Werkzeugen f�r die Fehlersuche zur
  Verf�gung. Etliche davon kommen mir sowohl bei Total- oder Partialausf�llen
  zu gute. Andere bei Performanceproblemen. Einige sind so n�tzlich, dass sie
  immer wieder bei den unterschiedlichsten Problemen zum Einsatz kommen. Da
  es mir schwerf�llt, die einzelnen Werkzeuge bestimmten Kategorien
  zuzuordnen, stelle ich diese nachfolgend in alphabetischer Reihenfolge vor.
  Das erleichtert zumindest das Wiederfinden, wenn man mal eben etwas schnell
  nachschlagen m�chte.
\end{normaltext}

\begin{notes}
\item hdparm
\item fuser
\item gdb
\item ifconfig
\item iproute
\item lsof
\item netstat
\item perl
\item shell
\item strace
\end{notes}

%\section{Konzept der Umsetzung}
%\label{sec:konz-der-umsetz}

%\begin{abstractsec}
%  Die Aufteilung der Umsetzung wird hier gegliedert in verschiedene Aspekte
%\end{abstractsec}
%\begin{normaltext}
%  Blablabla
%\end{normaltext}

%\subsection{Seiteneffekte}
%\label{sec:seiteneffekte}

%\subsection{Einschr�nkung der Umsetzung}
%\label{sec:einschr-der-umsetz}

%\begin{notes}
%\item Was nicht geht
%\item geht auch nicht
%\item und auch das
%\end{notes}

%\section{Schnittstellen nach au�en}
%\label{sec:schn-nach-au3en}

%\subsection{Schnittstelle zu A}
%\label{sec:schnittstelle-zu}

%\subsection{Schnittstelle zu B}
%\label{sec:schnittstelle-zu-b}

%\begin{notes}
%\item Syntax
%\item Semantik
%\end{notes}


%%% Local Variables: 
%%% mode: latex
%%% TeX-master: "arbeit-hauptdatei"
%%% End: 
%%% vim: set sw=2 ts=2 tw=78 et si:

\include{tl-programmprobleme}
\include{tl-mount-probleme}
\include{tl-dateiprobleme}
%% Netzwerkprobleme
\include{tl-klassifikation-netzwerk}
\include{tl-totalausfall}
\include{tl-teilweiser-ausfall}
%% tl-netz-performance.tex
\chapter{Netzwerkperformance}
\label{cha:netz-performance}

\begin{abstractsec}
  Manchmal scheint alles in Ordnung zu sein und trotzdem sind die Kunden nicht
  zufrieden. Performance ist ein heikles Thema, weil jeder seine eigene
  Vorstellung davon hat, was ausreichende Performance ist.
\end{abstractsec}

\begin{notes}
\item Baseline, ein Ausgangspunkt, der gemessen werden kann und als Referenz
  dient.
\end{notes}

%\section{Konzept der Umsetzung}
%\label{sec:konz-der-umsetz}

%\begin{abstractsec}
%  Die Aufteilung der Umsetzung wird hier gegliedert in verschiedene Aspekte
%\end{abstractsec}
%\begin{normaltext}
%  Blablabla
%\end{normaltext}

%\section{Schnittstellen nach außen}
%\label{sec:schn-nach-au3en}

%\subsection{Schnittstelle zu A}
%\label{sec:schnittstelle-zu}

%\begin{notes}
%\item Syntax
%\item Semantik
%\end{notes}

%\subsection{Schnittstelle zu B}
%\label{sec:schnittstelle-zu-b}

%\begin{notes}
%\item Syntax
%\item Semantik
%\end{notes}


%%% Local Variables: 
%%% mode: latex
%%% TeX-master: "arbeit-hauptdatei"
%%% End: 
%%% vim: set sw=2 ts=2 tw=78 et si:

\include{tl-netz-hilfsmittel}
%% Vorbereitung auf das nächste Problem
\include{tl-lernen}
\include{tl-syslog}
\include{tl-monitoring}

%% Das Literaturverzeichnis
\nocite{springerlink:10.1007/s00287-011-0541-z,%
guug:uptimes:2012.1/07,%
guug:uptimes:2012.1/09%
}
\bibliographystyle{alpha}
\bibliography{literatur}

\end{document}

%%% Local Variables: 
%%% mode: latex
%%% TeX-master: t
%%% End: 
