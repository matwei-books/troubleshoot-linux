%% tl-heuristiken.tex
\chapter{Heuristiken}
\label{cha:heuristiken}

\begin{abstractsec}
  Heuristiken können mitunter einen Ausweg bieten, wenn alle bekannten
  Methoden nicht anwendbar sind oder versagt haben.
\end{abstractsec}

\begin{notes}
\item zeitliche Korrelation
\item Suchmaschine
\item Supportforen
\end{notes}

%\section{Nicht verrückt machen lassen}
%\label{sec:nicht-verrueckt}

%\begin{abstractsec}
%  sonst kann man nicht klar denken.
%\end{abstractsec}
%\begin{normaltext}
%  Blablabla
%\end{normaltext}

%\subsection{Seiteneffekte}
%\label{sec:seiteneffekte}

%\begin{notes}
%\item Was nicht geht
%\item geht auch nicht
%\end{notes}

%\section{Methoden}
%\label{sec:methoden}

%\begin{abstractsec}
%  Neben den in den folgenden Kapiteln vorgestellten und vorgeführten
%  Programmen sind gute Methoden ein wichtiger Bestandteil meines
%  Werkzeugkastens bei der Fehlersuche.
%\end{abstractsec}

%\begin{notes}
%\item Entscheidungsbaum
%\end{notes}

%%% Local Variables: 
%%% mode: latex
%%% TeX-master: "arbeit-hauptdatei"
%%% End: 
%%% vim: set sw=2 ts=2 tw=78 et si:
