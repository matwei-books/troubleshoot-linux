%% tl-heuristiken.tex
\chapter{Heuristiken}
\label{cha:heuristiken}

\begin{abstractsec}
  Heuristiken können mitunter einen Ausweg bieten, wenn alle bekannten
  Methoden nicht anwendbar sind oder versagt haben. Oder sie können helfen,
  langwierige strukturierte Methoden abzukürzen und zu beschleunigen.
\end{abstractsec}

\begin{normaltext}
  Heuristiken wende ich an, wenn alle meine bekannten Methoden nicht
  anwendbar sind oder versagt haben. Oder, wenn ich eine etwas langwierige
  strukturierte Methode abkürzen will um schneller zum Ziel zu kommen.

  \begin{Exkursbox}{Methoden und Heuristiken}
    % hier vielleicht noch die Definitionen aus dem Duden
    In diesem Buch spreche ich von Methode als klaren strukturierten
    Anweisungen, die im Normalfall immer zu einer Lösung gelangen.

    Bei Heuristiken denke ich auch an ein strukturiertes
    Vorgehen, dass aber den Erkenntnisgewinn zum Ziel hat und nicht
    notwendigerweise direkt zur Lösung des Problems führt.
  \end{Exkursbox}
\end{normaltext}

\begin{notes}
\item zeitliche Korrelation
\item Suchmaschine
\item Supportforen
\end{notes}

\section{Korrelation}
\label{sec:korrelation}

%\begin{abstractsec}
%  sonst kann man nicht klar denken.
%\end{abstractsec}

\begin{normaltext}
Mit einer Korrelation kann man eine Beziehung zwischen zwei oder mehreren
Merkmalen, Ereignissen oder Zuständen beschreiben. Wichtig ist dabei, zu
beachten, dass diese Beziehung kausal sein kann aber nicht muss. Im
mathematischen Sinne beschreibt die Korrelation einen statistischen
Zusammenhang im Gegensatz zur Proportionalität, die einen festes Verhältnis
beschreibt.
\end{normaltext}

\begin{notes}
\item Zeitliche Korrelation: Wann trat der Fehler erstmalig auf und was wurde
  zu dieser Zeit geändert?
\item geht auch nicht
\end{notes}

\section{Abkürzungen und Umwege}
\label{sec:abkuerzung-umweg}

\begin{abstractsec}
  Abkürzungen beschleunigen das Erreichen des Ziels. Umwege erhöhen die
  Ortskenntnis.
\end{abstractsec}

\begin{normaltext}

\subsection{Abkürzungen}
\label{sec:abkuerzungen}

Die Frage, wann ein Problem erstmals auftrat (oder bemerkt wurde) und was zu
dieser Zeit oder davor geändert wurde, kann eine Fehlersuche erheblich
abkürzen, wenn Sie einen Hinweis ergibt auf eine mögliche Ursache ergibt. Bei
dieser Fragestellung haben wir es mit einer Korrelation zu tun und dürfen
nicht vergessen, dass die Änderungen mit dem Problem zu tun haben können, aber
nicht notwendigerweise müssen.

\subsection{Umwege}
\label{sec:umwege}

\subsubsection{Probe und Gegenprobe}

Jeder Test, der erfolgreich ist, sollte durch eine geeignete Gegenprobe
evaluiert werden, die bestätigt, dass der Test auch versagen kann. Das gleiche
gilt für fehlgeschlagene Tests. Diese müssen evaluiert werden, ob sie
funktioniern können.

Wenn ein Verbindungsversuch auf einem Rechner fehlschlägt, versuche ich den
gleichen Versuch von einem anderen Rechner aus um nachzuweisen, dass er hätte
funktionieren können. Das hat mich schon häufig davor bewahrt, einen
Netzwerkfehler zu vermuten, wenn lediglich ein Paketfilter auf dem Zielrechner
die Verbindung unterbunden hatte.
\end{normaltext}

%\begin{notes}
%\item Entscheidungsbaum
%\end{notes}

%%% Local Variables: 
%%% mode: latex
%%% TeX-master: "arbeit-hauptdatei"
%%% End: 
%%% vim: set sw=2 ts=2 tw=78 et si:
