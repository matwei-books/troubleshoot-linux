%% tl-netz-performance.tex
\chapter{Netzwerkperformance}
\label{cha:netz-performance}

\begin{abstractsec}
  Manchmal scheint alles in Ordnung zu sein und trotzdem sind die Kunden nicht
  zufrieden. Performance ist ein heikles Thema, weil jeder seine eigene
  Vorstellung davon hat, was ausreichende Performance ist.
\end{abstractsec}

\begin{notes}
\item Baseline, ein Ausgangspunkt, der gemessen werden kann und als Referenz
  dient.
\item Bufferbloat
\item Netalyzr
\end{notes}

\section{Netzwerkbandbreite}
\label{sec:netz-performance-bandbreite}
\begin{abstractsec}
  Die Bandbreite einzelner Netzsegmente bestimme ich normalerweise nicht erst
  im Fehlerfall, sondern vorher, am Besten gleich nach Inbetriebnahme eines
  Abschnitts. Dann habe ich bei Problemen einen Referenzwert, der mir bei der
  Eingrenzung des Problems nützlich sein kann.
\end{abstractsec}
\begin{normaltext}
  Die Bandbreite einzelner Netzsegmente bestimme ich normalerweise nicht erst
  im Fehlerfall, sondern vorher, am Besten gleich nach Inbetriebnahme eines
  Abschnitts. Dann habe ich bei Problemen einen Referenzwert, der mir bei der
  Eingrenzung des Problems nützlich sein kann.

  \subsection{Bestimmung der Bandbreite mit ping}
  Wie man mit Ping die Bandbreite bestimmen kann, ist sehr gut in
  \cite{sloan2001} Kapitel 4. Path Characteristics beschrieben.

  Das geht für beliebige Streckenabschnitte einer Verbindung und zwar wie
  folgt:
  \begin{enumerate}
    \item Mit Ping bestimmt man die RTT zum vorderen und hinteren Ende des
      Netzsegments. Die Differenzen zwischen den beiden Zeiten eliminieren die
      Einflüsse der anderen Netzkomponenten.
    \item Jetzt bestimmt man die RTT zu den beiden Enden mit größeren
      Datenpaketen und bestimmt wieder die Differenz.
    \item Die Differenz zwischen den beiden Zeitdifferenzen aus den ersten
      beiden Schritten ist die Zeit, die für die zusätzlichen Daten benötigt
      wird.
    \item Die Bitrate ist 16 mal der Differenz der Paketgrößen geteilt
      durch die Differenzzeit aus Schritt 3 (die magische Zahl 16 kommt daher,
      dass wir mit 8 Bit pro Byte rechnen und die die Messungen die doppelte
      Zeit, nämlich für Hin- und Rückweg, enthalten, aber nur die einfache
      Bitrate bestimmen wollen)
  \end{enumerate}
  Wenn ich mehrere Pingpakete pro Einzelmessung sende, dann verwende ich
  jeweils die geringste gemessene RTT, da diese vermutlich die geringsten
  Störeinflüsse enthält.
\end{normaltext}

\section{Lasttests}
\label{sec:netz-performance-lasttests}
\begin{abstractsec}
  Die reine Kenntnis von Bandbreite und Latenz von Netzwerken reicht nicht
  aus, um das Verhalten unter realen Bedingungen zu Beschreiben. Insbesondere
  durch Pufferung von Datenpaketen, die nicht gleich gesendet werden können,
  ändert sich die Latenz einer Übertragungsstrecke erheblich. Eine
  Möglichkeit, derartige Probleme zu diagnostizieren sind Lasttests.
\end{abstractsec}
\begin{normaltext}
  Zusätzlich zu den Latenzzeiten, die durch die Datenübertragung und das
  reine Umkopieren in den Routern und Switches entstehen, gibt es
  Verzögerungen, die durch die Pufferung von Datenpaketen verursacht werden.
  Diese entstehen dadurch, dass ein Netzwerkgerät auf der Sendeseite die Daten
  nicht so schnell los wird, wie sie auf der Empfangsseite ankommen. Das
  passiert meist beim Übergang von schnellen auf langsamere Medien, aber auch,
  wenn Datenpakete aus mehreren Richtungen ankommen und in die selbe Richtung
  abgehen. Tritt dieser Effekt nur kurzzeitig auf, dann wirken die Puffer
  positiv, da keines der Datenpaket verloren geht, sondern nur etwas später
  ankommt. Kommen jedoch ständig mehr Daten an, als gesendet werden können,
  dann werden erst die Puffer gefüllt, bevor Daten verworfen werden. Ohne
  aktives Puffermangement hat dann jedes ankommende Datenpaket so viele andere
  Datenpakete vor sich, wie in den Puffer passen. Die durch den Puffer
  verursachte Latenz beträgt dann Puffergröße geteilt durch Sendebandbreite.
  Leider verhindert die durch den Puffer erhöhte Latenzzeit, dass sich das
  TCP-Protokoll an die gerade mögliche Bandbreite anpassen kann. Die
  wirksamste  Abhilfe ist Adaptive Queue Management, ein Notbehelf kann
  Trafficshaping sein, wobei dieses den Nachteil hat, dass man als
  Netzadministrator die Sendebandbreite kennen muss um den Traffic manuell auf
  die entsprechende Rate zu begrenzen.

  Die reine Kenntnis von Bandbreite und Latenz von Netzwerken reicht also nicht
  aus, um das Verhalten unter realen Bedingungen zu Beschreiben. Insbesondere
  durch Pufferung von Datenpaketen, die nicht gleich gesendet werden können,
  ändert sich die Latenz einer Übertragungsstrecke erheblich. Eine
  Möglichkeit, derartige Probleme zu diagnostizieren sind Lasttests.

  Das heisst, ich erzeuge künstlichen einen starken Datenverkehr und
  beobachte, wie sich die anderen Netzparameter verhalten. Das betrifft zum
  einen die Latenz und zum anderen die verfügbare Bandbreite.
  \subsection{Lasttest mit Ping}
  Mit dem Befehl
  \begin{verbatim}
# ping -f rechnername
  \end{verbatim}
  sendet Ping Datenpakete so schnell es geht zum Zielrechner. Dazu benötige
  ich Superuserrechte.

  Diesen Aufruf verwende ich, um auf einem Segment Netzwerklast zu erzeugen.
  Zusätzlich zur normalen Statistik (min/avg/max/mdev) zeigt Ping am Ende zwei
  Werte: IPG und EWMA.

  IPG (Inter Packet Gap) ist die Zeit zwischen dem Senden zweier Datenpakete.
  Für Ethernet ist die Minimalzeit auf die Zeit festgelegt, in der 96 Bit
  übertragen werden. Das sind 9,6 µs für 10 MBit/s Ethernet und 9,6 ns für 10
  GBit/s. Diese Zeit wird automatisch vom Ethernetadapter an jedes Datenpaket
  angehängt. Das angezeigte IPG kann ich als Maß verwenden, um abzuschätzen,
  wie effizient die Kombination Betriebssystem, Netzwerkkarte, Netzwerk Daten
  senden kann.

  EWMA steht für Exponential Weighted Moving Average. Bei diesem Durchschnitt
  werden die letzten  RTT-Zeiten höher gewichtet als ältere. Im Normalfall
  sollte dieser gleich dem Mittelwert für RTT sein. Weicht er signifikant ab,
  deutet das auf einen Trend hin. Dafür benötigt man aber einen länger
  laufenden Ping und da EWMA am Ende ausgegeben wird, wird der Trend erst am
  Ende dieser Messung offenbar.
\end{normaltext}


%\begin{abstractsec}
%  Die Aufteilung der Umsetzung wird hier gegliedert in verschiedene Aspekte
%\end{abstractsec}
%\begin{normaltext}
%  Blablabla
%\end{normaltext}

%\section{Schnittstellen nach außen}
%\label{sec:schn-nach-au3en}

%\subsection{Schnittstelle zu A}
%\label{sec:schnittstelle-zu}

%\begin{notes}
%\item Syntax
%\item Semantik
%\end{notes}

%\subsection{Schnittstelle zu B}
%\label{sec:schnittstelle-zu-b}

%\begin{notes}
%\item Syntax
%\item Semantik
%\end{notes}


%%% Local Variables: 
%%% mode: latex
%%% TeX-master: "arbeit-hauptdatei"
%%% End: 
%%% vim: set sw=2 ts=2 tw=78 et si:
