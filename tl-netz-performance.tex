%% tl-netz-performance.tex
\chapter{Netzwerkperformance}
\label{cha:netz-performance}

\begin{abstractsec}
  Manchmal scheint alles in Ordnung zu sein und trotzdem sind die Kunden nicht
  zufrieden. Performance ist ein heikles Thema, weil jeder seine eigene
  Vorstellung davon hat, was ausreichende Performance ist.
\end{abstractsec}

\begin{notes}
\item Baseline, ein Ausgangspunkt, der gemessen werden kann und als Referenz
  dient.
\item Bufferbloat
\item Netalyzr
\end{notes}

%\section{Konzept der Umsetzung}
%\label{sec:konz-der-umsetz}

%\begin{abstractsec}
%  Die Aufteilung der Umsetzung wird hier gegliedert in verschiedene Aspekte
%\end{abstractsec}
%\begin{normaltext}
%  Blablabla
%\end{normaltext}

%\section{Schnittstellen nach außen}
%\label{sec:schn-nach-au3en}

%\subsection{Schnittstelle zu A}
%\label{sec:schnittstelle-zu}

%\begin{notes}
%\item Syntax
%\item Semantik
%\end{notes}

%\subsection{Schnittstelle zu B}
%\label{sec:schnittstelle-zu-b}

%\begin{notes}
%\item Syntax
%\item Semantik
%\end{notes}


%%% Local Variables: 
%%% mode: latex
%%% TeX-master: "arbeit-hauptdatei"
%%% End: 
%%% vim: set sw=2 ts=2 tw=78 et si:
