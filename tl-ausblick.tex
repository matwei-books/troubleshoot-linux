%% tl-ausblick.tex
\chapter{Nach der Fehlersuche ist vor der Fehlersuche}
\label{cha:ausblick}

\begin{abstractsec}
  Nach dem letzten Fehler wird es einen nächsten geben. Was kann ich tun, um
  zu vermeiden, dass es noch einmal den selben Fehler geben wird.
\end{abstractsec}

\begin{normaltext}
  Es heißt, Voraussagen sind schwierig, besonders für die Zukunft. Für die
  Vergangenheit ist es einfacher.
  Manchen Leuten scheint es gegeben, häufiger als andere richtige Voraussagen
  zu machen und damit Probleme schneller auf die Schliche zu kommen.
  Beobachtet man diese Leute genauer, stellt man vielleicht fest, das sie
  keine Voraussagen über die Zukunft machen, sondern über die Vergangenheit.
  Der Volksmund nennt dieses Phänomen auch Erfahrung.
\end{normaltext}

\section{Was habe ich gelernt}
\label{sec:was-habe-ich-gelernt}

\begin{notes}
\item Problem: ein Auslöser, mehrere Ursachen
  
  Nachbereitung: fünf mal warum, System umgestalten, so dass Auslöser erkannt
  wird (Monitoring), aber nicht zum Problem führt
\end{notes}

\section{Monitoring}
\label{sec:monitoring}

\begin{notes}
\item Kann ich den Fehler durch Monitoring vermeiden?
\end{notes}

%%% Local Variables: 
%%% mode: latex
%%% TeX-master: "arbeit-hauptdatei"
%%% End: 
%%% vim: set sw=2 ts=2 tw=78 et si:
