%% tl-herangehen.tex
%% vim: set sw=2 ts=2 tw=78 et si:
\chapter{Herangehen}
\label{cha:herangehen}

\begin{abstractsec}
  In diesem Kapitel geht es weder um Methoden und Handlungsangweisungen für
  die Fehlersuche noch um gute Ratschläge, falls alle Methoden nicht zum Ziel
  führen. Hier geht es um Faktoren, die den Erfolg einer Fehlersuche eher
  inderekt, dafür aber umso stärker beeinflussen. Es geht um Einstellungen,
  Gewohnheiten, darum, wie mein Gehirn funktioniert und wie ich das
  gegebenenfalls ausnutzen kann um ein hartnäckiges Problem doch noch zu
  lösen.
\end{abstractsec}

\begin{normaltext}

\section{Prinzipien}
\label{sec:prinzipien}

\subsection{Vier Kraftprinzipien}

\begin{quote}
Es ist gegen neun Uhr, gut gelaunt betrete ich die Büroräume, als mich der
Kollege am Operatingplatz mit leicht erhobener Stimme empfängt:

``Im Rathaus geht das INTERNET nicht!''

``Das ist aber schade.''

Eine dritte Kollegin ruft aus dem Hintergrund: ``Das habe ich gewusst, dass er
das jetzt sagt''.
\end{quote}

Was war denn da los? Proben die für eine Fernsehsendung?

Nun ich habe hier Prinzipien angewandt, die ich im Wing Tsun Kung Fu (WT)
gelernt habe und - entsprechend angepasst - bei Fehlersuchen aller Art gut
verwenden kann. Es gibt im WT vier Kraftprinzipien, die lauten:

\begin{enumerate}
  \item Mache Dich frei von Deiner eigenen Kraft.
  \item Mache Dich frei von der Kraft Deines Gegners.
  \item Nutze die Kraft Deines Gegners.
  \item Füge zur Kraft Deines Gegners Deine eigene Kraft hinzu.
\end{enumerate}

Das klingt alles sehr schön und esoterisch, aber was hat das mit Fehlersuche
in Computern und Computernetzen zu tun?

Die eigene Kraft, von der ich mich frei machen will, sind vorschnelle
Vermutungen, die ich aufstelle, die manchmal korrekt sind aber mich oft nur
daran hindern, das eigentliche Problem wahrzunehmen.

Die Kraft meines Gegners, von der ich mich freimachen will, sind die
Vermutungen und voreiligen Diagnosen dessen, der mir das Problem mitteilt.
Diese können mich, genau wie meine eigenen, in die falsche Richtung locken.

Wie nutze ich nun die Kraft meines Gegners? Nachdem ich mich eigener
vorschneller Vermutungen und Diagnosen enthalten habe und die Aussagen meines
Gegenübers zwar zur Kenntnis genommen aber mir nicht unbedingt zu eigen
gemacht habe, evaluiere ich seine Angaben zum Beispiel anhand des
grundlegenden Entscheidungsbaumes um das wirkliche Ausmaß des Problems und die
nächsten Schritte zu bestimmen. Dabei frage ich gezielt und strukturiert nach.

Meine Kraft füge ich hinzu, indem ich aus den so gewonnenen Kenntnissen eine
Lösung des Problems herausarbeite.

Im eingangs erwähnten Gespräch wollte der Operator mr mit dem Tonfall und der
Schilderung wohl die besondere Dringlichkeit nahelegen und etwas Druck
aufbauen. Meine erste Reaktion war, diesen Druck in's Leere laufen zu lassen,
um den Kopf zum klaren Denken freizuhalten. Meine nächste Frage ging dan auch
danach, ob andere Websites aus dem Rathausnetz nicht erreichbar waren und ob
diese Websites aus unserem Netz erreichbar waren. So nutzte ich seine Kraft
und fügte meine - in diesem Fall das strukturierte Erfassen der Situation -
hinzu. Letztendlich stellte sich heraus, dass nur einige wenige Websites
betroffen waren und das - aus unserer Sicht - global, das heisst ausserhalb
undseres Einflussbereiches. Bis ich an meinem Platz ankam, waren alle Wogen
geglättet. Was blieb, sind ein paar Erinnerungen und ein Ticket im
Ticketsystem.
\end{normaltext}

\section{Zwei Bewusstseinsmodi}
\label{sec:zwei-bewusstseinsmodi}

\begin{abstractsec}
  Es gibt eine Unterscheidung des menschlichen Bewußstseins in einen
  sprachlich, analytischen Modus und einen visuellen, wahrnehmenden Modus.
  Obwohl die meisten Menschen beide Modi mehr oder weniger unbewusst
  verwenden, ist es gerade bei der Fehlersuche von Vorteil beide gezielt
  einsetzen zu können und gezielt zwischen beiden umschalten zu können.
\end{abstractsec}
\begin{normaltext}
  Es gibt eine Unterscheidung des menschlichen Bewußstseins in einen
  sprachlich, analytischen Modus und einen visuellen, wahrnehmenden Modus.
  Obwohl die meisten Menschen beide Modi mehr oder weniger unbewusst
  verwenden, ist es gerade bei der Fehlersuche von Vorteil beide gezielt
  einsetzen zu können und gezielt zwischen beiden umschalten zu können.

  Vor meinem Studium habe ich einen Beruf erlernt, in dem ich unter anderem
  Telefonvermittlungsanlagen reparieren können musste, die damals noch mit
  Relais funktionierten. Damals hatten wir zwei Arten dieser Anlagen. Die
  alten, mit Hebdrehwählern hatten nur relativ wenige Relais und wenn man in
  der Vermittlunsanlage beobachtete, was beim Abheben des Telefonhörers
  passierte, dann konnte man beinahe das nächste Relais ansagen, was betätigt
  wurde. Das heißt, diese Anlagen konnte man im sprachlich analytischen Modus
  analysieren.

  Die anderen, moderneren Anlagen, funktionierten mit Koordinatenschaltern.
  Das waren große elektromagnetische Schalter, die in einer Art Matrix die
  richtige Verbindung durchschalteten. Diese konnten eine Verbindung erheblich
  schneller durchschalten, als Hebdrehwähler und konnten auch die Tonsignale
  von Tastentelefonen ohne Umsetzungen in Zähltakte verarbeiten. Dafür
  benötigten sie erheblich mehr Relais zur Funktion und Fehler konnten dort
  nicht wie in den alten Anlagen analysiert werden, da fast immer mehrere
  Relais gleichzeitig betätigt wurden. Dafür lernten wir eine andere
  Herangehensweise. Zunächst setzten wir uns so, dass wir die ganze Anlage im
  Blick hatten. Dann beobachteten wir wiederholt das Gesamtbild der Anlage,
  wenn bestimmte Aktionen ausgeführt wurden. Anschließend verglichen wir die
  Relais, die sich bewegt hatten mit dem Schaltplan der Anlage und
  erarbeiteten uns analytisch Stück für Stück den Zusammenhang zwischen dem,
  was wir gesehen hatten und dem, was laut Schaltplan hätte passieren müssen.
  Am Anfang wiederholten wir das mit einer funktionsfähigen Anlage. Nach
  einiger Zeit, als wir ein Gefühl für die Funktionsweise der Anlage bekommen
  hatten, isolierte unser Lehrmeister einige Relaiskontakte mit farblosem
  schnelltrocknendem Kleber und unsere Aufgabe war es, herauszufinden, welche
  Kontakte schlecht waren. Später, als wir dadurch ein Gefühl dafür bekamen,
  welcher Fehler wie aussah, erhielten die Fortgeschrittenen die Aufgabe sich
  selbst Fehler auszudenken, die die anderen finden sollten. Natürlich setzten
  wir unseren Ehrgeiz dahin, Fehler einzubauen, die durch ihr Verhalten die
  anderen zunächst auf die falsche Spur schickten. Und natürlich kannten wir
  diese Anlagen am Ende in- und auswendig. Was wir dabei - spielerisch -
  übten, war ein Umschalten zwischen dem visuellen, wahrnehmenden Modus und
  dem sprachlich, analytischen Modus unseres Gehirns. Und, ja, heute könnte
  ich keine solcher Anlagen mehr reparieren, müsste alles erst wieder lernen.
  Was ich aber immer noch kann, ist relativ schnell zwischen dem Gesamtbild
  eines Problems und den Details zu wechseln.

  Dieser Exkurs in eine längst vergangene Zeit war nötig, um einen Aspekt
  hervorzuheben, der damals so wichtig ist, wie heute: das die unzähligen
  Details, die man bei der Fehlersuche wissen muss, für sich genommen nichts
  wert sind, wenn es nicht gelingt, diese in ein Gesamtbild einzufügen und
  dadurch richtig zu bewerten. Wenn ich in der Lage bin, das Gesamtbild zu
  sehen, kann ich für jedes Detail einschätzen, ob es korrekt, falsch oder
  irrelevant ist und mich dementsprechend dem nächsten zuwenden.
  Dazu ist es notwendig, beide Gehirnhälften oder Bewusstseinsmodi gezielt
  einzusetzen. In \cite{edwards:zeichnenlernen} beschreibt die Autorin am
  Beispiel des Zeichnenlernens, einige Übungen zum gezielten Umschalten der
  beiden Modi, die sie L-Modus und R-Modus nennt.
\end{normaltext}

\begin{notes}
\item Bisektion zur Beschleunigung: nicht den kompletten Entscheidungsbaum
  abarbeiten.
\end{notes}

%\subsection{Seiteneffekte}
%\label{sec:seiteneffekte}

%\section{Methoden}
%\label{sec:methoden}

%\begin{abstractsec}
%  Neben den in den folgenden Kapiteln vorgestellten und vorgeführten
%  Programmen sind gute Methoden ein wichtiger Bestandteil meines
%  Werkzeugkastens bei der Fehlersuche.
%\end{abstractsec}

%\begin{notes}
%\item Entscheidungsbaum
%\end{notes}

%\section{Heuristiken}
%\label{heuristiken}

%\begin{abstractsec}
%  Wenn gar nichts mehr geht, helfen vielleicht Heuristiken.
%\end{abstractsec}

%\begin{notes}
%\item zeitliche Korrelation
%\end{notes}

%%% Local Variables: 
%%% mode: latex
%%% TeX-master: "arbeit-hauptdatei"
%%% End: 
