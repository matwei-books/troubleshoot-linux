%% tl-herangehen.tex
%% vim: set sw=2 ts=2 tw=78 et si:
\chapter{Herangehen}
\label{cha:herangehen}

\begin{abstractsec}
  In diesem Kapitel geht es weder um Methoden und Handlungsangweisungen f�r
  die Fehlersuche noch um gute Ratschl�ge, falls alle Methoden nicht zum Ziel
  f�hren. Hier geht es um Faktoren, die den Erfolg einer Fehlersuche eher
  inderekt, daf�r aber umso st�rker beeinflussen. Es geht um Einstellungen,
  Gewohnheiten, darum, wie mein Gehirn funktioniert und wie ich das
  gegebenenfalls ausnutzen kann um ein hartn�ckiges Problem doch noch zu
  l�sen.
\end{abstractsec}

%\section{Nicht verr�ckt machen lassen}
%\label{sec:nicht-verrueckt}

%\begin{abstractsec}
%  sonst kann man nicht klar denken.
%\end{abstractsec}
%\begin{normaltext}
%  Blablabla
%\end{normaltext}

%\subsection{Seiteneffekte}
%\label{sec:seiteneffekte}

%\begin{notes}
%\item Was nicht geht
%\item geht auch nicht
%\end{notes}

%\section{Methoden}
%\label{sec:methoden}

%\begin{abstractsec}
%  Neben den in den folgenden Kapiteln vorgestellten und vorgef�hrten
%  Programmen sind gute Methoden ein wichtiger Bestandteil meines
%  Werkzeugkastens bei der Fehlersuche.
%\end{abstractsec}

%\begin{notes}
%\item Entscheidungsbaum
%\end{notes}

%\section{Heuristiken}
%\label{heuristiken}

%\begin{abstractsec}
%  Wenn gar nichts mehr geht, helfen vielleicht Heuristiken.
%\end{abstractsec}

%\begin{notes}
%\item zeitliche Korrelation
%\end{notes}

%%% Local Variables: 
%%% mode: latex
%%% TeX-master: "arbeit-hauptdatei"
%%% End: 
