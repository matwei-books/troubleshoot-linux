%% tl-herangehen.tex
%% vim: set sw=2 ts=2 tw=78 et si:
\chapter{Herangehen}
\label{cha:herangehen}

\begin{abstractsec}
  In diesem Kapitel geht es weder um Methoden und Handlungsangweisungen für
  die Fehlersuche noch um gute Ratschläge, falls alle Methoden nicht zum Ziel
  führen. Hier geht es um Faktoren, die den Erfolg einer Fehlersuche eher
  inderekt, dafür aber umso stärker beeinflussen. Es geht um Einstellungen,
  Gewohnheiten, darum, wie mein Gehirn funktioniert und wie ich das
  gegebenenfalls ausnutzen kann um ein hartnäckiges Problem doch noch zu
  lösen.
\end{abstractsec}

\begin{normaltext}

\section{Prinzipien}
\label{sec:prinzipien}

\subsection{Vier Kraftprinzipien}

\begin{quote}
Es ist gegen neun Uhr, gut gelaunt betrete ich die Büroräume, als mich der
Kollege am Operatingplatz mit leicht erhobener Stimme empfängt:

``Im Rathaus geht das INTERNET nicht!''

``Das ist aber schade.''

Eine dritte Kollegin ruft aus dem Hintergrund: ``Das habe ich gewusst, dass er
das jetzt sagt''.
\end{quote}

Was war denn da los? Proben die für eine Fernsehsendung?

Nun ich habe hier Prinzipien angewandt, die ich im Wing Tsun Kung Fu (WT)
gelernt habe und - entsprechend angepasst - bei Fehlersuchen aller Art gut
verwenden kann. Es gibt im WT vier Kraftprinzipien, die lauten:

\begin{enumerate}
  \item Mache Dich frei von Deiner eigenen Kraft.
  \item Mache Dich frei von der Kraft Deines Gegners.
  \item Nutze die Kraft Deines Gegners.
  \item Füge zur Kraft Deines Gegners Deine eigene Kraft hinzu.
\end{enumerate}

Das klingt alles sehr schön und esoterisch, aber was hat das mit Fehlersuche
in Computern und Computernetzen zu tun?

Die eigene Kraft, von der ich mich frei machen will, sind vorschnelle
Vermutungen, die ich aufstelle, die manchmal korrekt sind aber mich oft nur
daran hindern, das eigentliche Problem wahrzunehmen.

Die Kraft meines Gegners, von der ich mich freimachen will, sind die
Vermutungen und voreiligen Diagnosen dessen, der mir das Problem mitteilt.
Diese können mich, genau wie meine eigenen, in die falsche Richtung locken.

Wie nutze ich nun die Kraft meines Gegners? Nachdem ich mich eigener
vorschneller Vermutungen und Diagnosen enthalten habe und die Aussagen meines
Gegenübers zwar zur Kenntnis genommen aber mir nicht unbedingt zu eigen
gemacht habe, evaluiere ich seine Angaben zum Beispiel anhand des
grundlegenden Entscheidungsbaumes um das wirkliche Ausmaß des Problems und die
nächsten Schritte zu bestimmen. Dabei frage ich gezielt und strukturiert nach.

Meine Kraft füge ich hinzu, indem ich aus den so gewonnenen Kenntnissen eine
Lösung des Problems herausarbeite.

Im eingangs erwähnten Gespräch wollte der Operator mr mit dem Tonfall und der
Schilderung wohl die besondere Dringlichkeit nahelegen und etwas Druck
aufbauen. Meine erste Reaktion war, diesen Druck in's Leere laufen zu lassen,
um den Kopf zum klaren Denken freizuhalten. Meine nächste Frage ging dan auch
danach, ob andere Websites aus dem Rathausnetz nicht erreichbar waren und ob
diese Websites aus unserem Netz erreichbar waren. So nutzte ich seine Kraft
und fügte meine - in diesem Fall das strukturierte Erfassen der Situation -
hinzu. Letztendlich stellte sich heraus, dass nur einige wenige Websites
betroffen waren und das - aus unserer Sicht - global, das heisst ausserhalb
undseres Einflussbereiches. Bis ich an meinem Platz ankam, waren alle Wogen
geglättet. Was blieb, sind ein paar Erinnerungen und ein Ticket im
Ticketsystem.

\end{normaltext}

\begin{notes}
\item Bisektion zur Beschleunigung: nicht den kompletten Entscheidungsbaum
  abarbeiten.
\end{notes}

%\section{Nicht verrückt machen lassen}
%\label{sec:nicht-verrueckt}

%\begin{abstractsec}
%  sonst kann man nicht klar denken.
%\end{abstractsec}
%\begin{normaltext}
%  Blablabla
%\end{normaltext}

%\subsection{Seiteneffekte}
%\label{sec:seiteneffekte}

%\section{Methoden}
%\label{sec:methoden}

%\begin{abstractsec}
%  Neben den in den folgenden Kapiteln vorgestellten und vorgeführten
%  Programmen sind gute Methoden ein wichtiger Bestandteil meines
%  Werkzeugkastens bei der Fehlersuche.
%\end{abstractsec}

%\begin{notes}
%\item Entscheidungsbaum
%\end{notes}

%\section{Heuristiken}
%\label{heuristiken}

%\begin{abstractsec}
%  Wenn gar nichts mehr geht, helfen vielleicht Heuristiken.
%\end{abstractsec}

%\begin{notes}
%\item zeitliche Korrelation
%\end{notes}

%%% Local Variables: 
%%% mode: latex
%%% TeX-master: "arbeit-hauptdatei"
%%% End: 
