%% tl-herangehen.tex
%% vim: set sw=2 ts=2 tw=78 et si:
\chapter{Herangehen}
\label{cha:herangehen}

\begin{abstractsec}
  In diesem Kapitel geht es weder um Methoden und Handlungsangweisungen für
  die Fehlersuche noch um gute Ratschläge, falls alle Methoden nicht zum Ziel
  führen. Hier geht es um Faktoren, die den Erfolg einer Fehlersuche eher
  inderekt, dafür aber umso stärker beeinflussen. Es geht um Einstellungen,
  Gewohnheiten, darum, wie mein Gehirn funktioniert und wie ich das
  gegebenenfalls ausnutzen kann um ein hartnäckiges Problem doch noch zu
  lösen.
\end{abstractsec}

\begin{notes}
\item Bisektion zur Beschleunigung: nicht den kompletten Entscheidungsbaum
  abarbeiten.
\end{notes}

%\section{Nicht verrückt machen lassen}
%\label{sec:nicht-verrueckt}

%\begin{abstractsec}
%  sonst kann man nicht klar denken.
%\end{abstractsec}
%\begin{normaltext}
%  Blablabla
%\end{normaltext}

%\subsection{Seiteneffekte}
%\label{sec:seiteneffekte}

%\section{Methoden}
%\label{sec:methoden}

%\begin{abstractsec}
%  Neben den in den folgenden Kapiteln vorgestellten und vorgeführten
%  Programmen sind gute Methoden ein wichtiger Bestandteil meines
%  Werkzeugkastens bei der Fehlersuche.
%\end{abstractsec}

%\begin{notes}
%\item Entscheidungsbaum
%\end{notes}

%\section{Heuristiken}
%\label{heuristiken}

%\begin{abstractsec}
%  Wenn gar nichts mehr geht, helfen vielleicht Heuristiken.
%\end{abstractsec}

%\begin{notes}
%\item zeitliche Korrelation
%\end{notes}

%%% Local Variables: 
%%% mode: latex
%%% TeX-master: "arbeit-hauptdatei"
%%% End: 
