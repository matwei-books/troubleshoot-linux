%% tl-netz-totalausfall.tex
\chapter{Totalausfall des Netzes}
\label{cha:netz-totalausfall}

\begin{abstractsec}
  Habe ich an einem Rechner überhaupt keine Netzverbindung, oder kann ich ein
  ganzes Netzsegment nicht erreichen, spreche ich von einem Totalausfall.
\end{abstractsec}

\section{Ein Rechner hat überhaupt keine Netzverbindung}
\label{sec:gar-kein-netz}

%\begin{abstractsec}
%  Die Aufteilung der Umsetzung wird hier gegliedert in verschiedene Aspekte
%\end{abstractsec}
%\begin{normaltext}
%  Blablabla
%\end{normaltext}

%\subsection{Seiteneffekte}
%\label{sec:seiteneffekte}

%\subsection{Einschränkung der Umsetzung}
%\label{sec:einschr-der-umsetz}

%\begin{notes}
%\item Was nicht geht
%\item geht auch nicht
%\item und auch das
%\end{notes}

\section{Ein oder mehrere Netzsegmente sind nicht erreichbar}
\label{sec:ausfall-netzsegment}

\begin{notes}
\item Entscheidungsbaum!
\end{notes}

\subsection{Routen zu diesen Netzen fehlen}
\label{sec:routen fehlen}

\begin{notes}
\item Routingprotokolle
\item OSPF: BDR hatte falschen Neighbor (openbsd, GeNUScreen)
\item Routersoftware quagga
\end{notes}

\subsection{Rückroute fehlt}
\label{sec:rueckroute-fehlt}

\begin{notes}
\item Problem: Routen scheinen zu stimmen, keine Antwortpakete
\item Analyse: traceroute, analyse des letzten Hops und desjenigen danach
\item Ursache: Rückroute fehlt
\end{notes}

%\subsection{Schnittstelle zu B}
%\label{sec:schnittstelle-zu-b}

%\begin{notes}
%\item Syntax
%\item Semantik
%\end{notes}


%%% Local Variables: 
%%% mode: latex
%%% TeX-master: "arbeit-hauptdatei"
%%% End: 
%%% vim: set sw=2 ts=2 tw=78 et si:
